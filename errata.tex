\documentclass[a5paper,11pt]{article}

\usepackage[margin=1.75cm]{geometry}

\usepackage[ngerman]{babel}

\newcommand{\stylepath}{./langsci/styles/}	
\usepackage{langsci/styles/langsci-cgloss}
\usepackage{langsci/styles/langsci-gb4e}

\usepackage{ifthen}
\usepackage{tocloft}
\usepackage{enumitem}
\usepackage{hyperref}
\usepackage{graphicx}
\usepackage[linguistics]{forest}
\usepackage{unicode-math}
\usepackage{fontspec}
\usepackage{setspace}
\usepackage{xcolor}
\usepackage{multirow}
\usepackage{tikz}
\usetikzlibrary{positioning,arrows.meta,cd}
\tikzset{>=latex}

\usepackage{longtable}

\newcounter{chapter}

\usepackage{libertine}
\newcommand{\fontpath}{./langsci/fonts/}
\setmonofont[
	Ligatures={TeX},Scale=MatchLowercase,
	BoldFont = DejaVuSansMono-Bold.ttf ,
	SlantedFont = DejaVuSansMono-Oblique.ttf ,				
	BoldSlantedFont = DejaVuSansMono-BoldOblique.ttf 		
	]{DejaVuSansMono.ttf}
  \setsansfont[
	  Scale=MatchLowercase,
	  Path=\fontpath,
	  BoldFont = Arimo-Bold_B.ttf ,
	  ItalicFont = Arimo-Italic_B.ttf ,				
	  BoldItalicFont = Arimo-BoldItalic_B.ttf 		
	  ]{Arimo_B.ttf}
  \setmainfont[
	  Ligatures={TeX,Common},
	  Path=\fontpath,
	  PunctuationSpace=0,
	  Numbers={Proportional},
	  BoldFont = LinLibertine_RZ_B.otf ,
	  ItalicFont = LinLibertine_RI_B.otf ,
	  BoldItalicFont = LinLibertine_RZI_B.otf,
	  BoldSlantedFont = LinLibertine_RZ_B.otf,
	  SlantedFont    = LinLibertine_R_B.otf,
	  SlantedFeatures = {FakeSlant=0.25},
	  BoldSlantedFeatures = {FakeSlant=0.25},
	  SmallCapsFeatures = {FakeSlant=0},
	  ]{LinLibertine_R_B.otf}		

\begin{document}

% By LSP.
%\renewbibmacro*{index:name}[5]{%
%  \usebibmacro{index:entry}{#1}
%    {\iffieldundef{usera}{}{\thefield{usera}\actualoperator}\mkbibindexname{#2}{#3}{#4}{#5}}}


% By LSP.
\makeatletter
\def\blx@maxline{77}
\makeatother


% Mark for proof-reading/revision.
\newcommand{\Dirty}{\marginpar{\hl{DIRTY!}}}


% Fix line spacing in list environmens.
\setlist{noitemsep}


% Correct hyperref colors. The original ones would give you eye cancer.
\hypersetup{
  linkbordercolor  = {white}
%  , linkcolor        = {lsMidDarkBlue}
%  , anchorcolor      = {lsMidWine}
%  , citecolor        = {lsDarkGreenOne}
%  , menucolor        = {lsMidDarkBlue}
%  , urlcolor         = {lsDarkOrange}
%    , filecolor       = {}
%    , runcolor        = {}
}


% Use a better mono font, ideal for code.
% https://github.com/chrissimpkins/codeface/tree/master/fonts/inconsolata-g
% \setmonofont{Inconsolata-g}


% Use a math font that actually works! Requires unicode-math paackage.
% https://github.com/khaledhosny/libertinus
\setmathfont[Scale=MatchUppercase]{Libertinus Math}


\newenvironment{listLileveli}{\begin{enumerate}}{\end{enumerate}}
\newenvironment{listLilevelii}{\begin{enumerate}}{\end{enumerate}}
\newenvironment{listLileveliii}{\begin{enumerate}}{\end{enumerate}}
\newenvironment{listLileveliv}{\begin{enumerate}}{\end{enumerate}}
\newenvironment{listLiileveli}{\begin{itemize}}{\end{itemize}}
\newenvironment{listLiilevelii}{\begin{itemize}}{\end{itemize}}
\newenvironment{listLiileveliii}{\begin{itemize}}{\end{itemize}}
\newenvironment{listLiileveliv}{\begin{itemize}}{\end{itemize}} 

% Feinsatz 
% Remove with :g/^\\\(Enl\|Stretch\|Unstretch\|Hardstretch\)/d
\newcommand{\Stretch}[1][1]{\vspace{#1\baselineskip}}
\newcommand{\Unstretch}[1][0.5]{\vspace{-#1\baselineskip}}
\newcommand{\Hardstretch}[1][1]{\vspace*{#1\baselineskip}}
\newcommand{\Enl}[1][1]{\enlargethispage{#1\baselineskip}}
\newcommand{\Np}{\newpage}


% all sorts of shortcuts
\newcommand{\HStrut}[1]{\rule{0pt}{#1pt}}
\newcommand{\VStrut}[1]{\rule{#1pt}{0pt}}
\newcommand{\Sw}[1]{\begin{sideways}#1\end{sideways}}
\newcommand{\Ast}{*}
\definecolor{lg}{rgb}{.8,.8,.8}
\newcommand{\Dim}{\cellcolor{lg}}
\newcommand{\Tidx}[1]{\ensuremath{_{\mathnormal{#1}}}}
\newcommand{\Rollen}[1]{\ensuremath{\langle}#1\ensuremath{\rangle}}
\newcommand{\Qc}[1]{\texttt{#1}}
\newcommand{\acr}[1]{{#1}}
\newcommand{\ak}[1]{#1\marginpar{\textcolor{textblue}{\footnotesize #1}}}
\newcommand{\mar}[1]{\marginpar{\textcolor{textblue}{\footnotesize #1}}}
\newcommand{\zB}{z.\thinspace{}B.~}
\newcommand{\oA}{o.\thinspace{}ä.~}
\newcommand{\idR}{i.\,d.\,R.\ }
\newcommand{\BBel}[1]{\B[-5]{#1}}
\newcommand{\Bewegtes}[1]{\ensuremath{_{\textrm{#1}}}}
\newcommand{\ORi}{\Bewegtes{1}}
\newcommand{\ORii}{\Bewegtes{2}}
\newcommand{\ORiii}{\Bewegtes{3}}
\newcommand{\ORiv}{\Bewegtes{4}}
\newcommand{\ORv}{\Bewegtes{5}}
\newcommand{\Spur}[1]{t\Sub{#1}}
\newcommand{\Ti}{\Spur{1}}
\newcommand{\Tii}{\Spur{2}}
\newcommand{\Tiii}{\Spur{3}}
\newcommand{\Tiv}{\Spur{4}}
\newcommand{\Akz}{ˈ}
\newcommand{\Nakz}{ˌ}
\newcommand{\PhPr}[1]{\ensuremath{\stackrel{\textnormal{#1\ }}{\Longrightarrow}}}
\newcommand{\phopro}{\ensuremath{\Rightarrow}}

\newcommand{\KTArr}[1]{\ding{226}~\textit{#1}~\ding{226}}

\newcommand{\VfTest}{\KTArr{VfTest}~}
\newcommand{\PronTest}{\KTArr{PronTest}~}
\newcommand{\KoorTest}{\KTArr{KoorTest}~}
\newcommand{\onestar}{◆◇◇}
\newcommand{\twostar}{◆◆◇}
\newcommand{\tristar}{◆◆◆}
\newcommand{\RPr}{\ensuremath{\ll}}
\newcommand{\RUn}{\ensuremath{\sim}}
\newcommand{\REq}{\ensuremath{=}}
\newcommand{\Opsional}{★~}
\newcommand{\Nono}{---}
\newcommand{\Sub}[1]{\ensuremath{_{\text{#1}}}}
\newcommand{\Up}[1]{\ensuremath{^{\text{#1}}}}
\newcommand{\UpSub}[2]{\ensuremath{^{\text{#1}}_{\text{#2}}}}
\newcommand{\TuBegin}{\ding{217}}
\newcommand{\TuEnd}{}
\newcommand{\Folgt}{\ding{217}}

\newcommand*\circlearound[1]{\tikz[baseline=(char.base)]{\node[shape=circle,draw,inner sep=2pt] (char) {#1};}}


% boxes and stuff for definitions, axioms etc.
\definecolor{textblue}{rgb}{0,0,.5}
\definecolor{textred}{rgb}{.5,0,0}
\definecolor{textgreen}{rgb}{0,.5,0}
\definecolor{lightblue}{rgb}{.9,.9,1}
\definecolor{lightgreen}{rgb}{.9,1,.9}
\definecolor{lightred}{rgb}{1,.9,.9}
\definecolor{lightyellow}{rgb}{1,1,.8}
\definecolor{lightgray}{rgb}{.88,.88,.88}
\definecolor{lsLightgray}{gray}{0.88}
\definecolor{lsYellow}{cmyk}{0,0.25,1,0}

\newcommand{\whyte}[1]{\textcolor{white}{#1}}

% bookkeeping of own environments for definitions etc.
\newlistof[chapter]{dgdef}{dgd}{Verzeichnis der Definitionen}
\newlistof[chapter]{dgsatz}{dgs}{Verzeichnis der Sätze}
\newlistof[chapter]{dgvertief}{dgv}{Verzeichnis der Vertiefungen}

\newcounter{deskgram-ziel}
\newcounter{deskgram-ex}
\newcounter{deskgram-strsch}
\newcounter{deskgram-wfilt}
\newcounter{deskgram-pholproz}


% Fix spacing between numbers and caption in list of figures etc.
\makeatletter
  \renewcommand*\l@figure{\@dottedtocline{1}{1em}{3.2em}}
  \renewcommand*\l@table{\@dottedtocline{1}{1em}{3.2em}}
  \renewcommand*\l@dgdef{\@dottedtocline{1}{1em}{3.2em}}
  \renewcommand*\l@dgsatz{\@dottedtocline{1}{1em}{3.2em}}
\makeatother


% NEW Vertiefung (non-floating)

\newcommand{\tblsthickline}{{\color{gray}\rule{\textwidth}{1.5mm}}}

\newenvironment{Sandwich}[1]
  {
    \par\vspace{5mm}\noindent\tblsthickline
    {\par\vspace{3mm}\noindent\sffamily\large\bfseries#1\vspace{4mm}}
  }
  {\par\vspace{3mm}\noindent\tblsthickline\par\vspace{5mm}}


\newenvironment{Vertiefung}[1]
  { 
    \refstepcounter{dgvertief}
    \begin{Sandwich}{#1\hfill Vertiefung~\thechapter.\arabic{dgvertief}}
  }
  {\end{Sandwich}}


% if not using hyperref, redefine this as empty:
\newcommand{\Phantom}{\phantomsection}


\newcommand{\Definition}[2]{
  \refstepcounter{dgdef}
  \tblscolorbox[report]{lsLightGray}{#1\hfill Definition~\thechapter.\arabic{dgdef}}{#2}
}



\newcommand{\Satz}[2]{
  \refstepcounter{dgsatz}
  \tblscolorbox[glass]{lsLightGray}{#1\hfill Satz~\thechapter.\arabic{dgsatz}}{#2}
}



\newcommand{\Zusammenfassung}[1]{
  \tblsframebox[refresh]{lsYellow}{Zusammenfassung von Abschnitt~\thesection}{#1}
}



\newcommand{\WFiltTree}[7][0mm]{%
  \refstepcounter{deskgram-wfilt}
  \tblscolorbox[filter]{lsLightGray}{#2\hfill Wortklassenfilter~\arabic{deskgram-wfilt}}{
    \label{#3}
    \hspace{-5pt}\centering
    \begin{tikzpicture}[baseline]
    \node at (0,0) (Wort) [align=left] {#4};
    \node [right=of Wort.east, text width=3.5cm, align=left] (Filter) {#5};
    \node [above right=\baselineskip and 1cm of Filter.east] (Ja) {#6};
    \node [below right=\baselineskip and 1cm of Filter.east] (Nein) {#7}; 
    \path (Wort) edge [-{Latex[round]}] (Filter);
    \path (Filter.east) edge [-{Latex[round]}] node [above,sloped] {Ja} (Ja.west);
    \path (Filter.east) edge [-{Latex[round]}] node [below,sloped] {Nein} (Nein.west);
    \aeundefinethesenodes{Wort, Filter, Ja, Nein}
  \end{tikzpicture}\vspace{#1}%
  }
}

\newcommand{\strschemspace}{\hspace{1em}}
\newcommand{\Phrasenschema}[2]{
  \refstepcounter{deskgram-strsch}
  \tblscolorbox[tree]{lsLightGray}{#1\hfill Phrasenschema~\arabic{deskgram-strsch}}{
    \centering
    #2
  }
}



% OLD "further reading"
\newcommand{\WeitereLiteratur}[1]{
  \tblsframebox[book]{lsLightGray}{Weiterführende Literatur zu \thepart}{#1}
}


% Excercises.
\newcounter{Exer}[chapter]

\newcommand{\Uebung}[2][\twostar]{%
  \par\medskip\refstepcounter{Exer}\noindent\textbf{Übung~\arabic{Exer}}~#1\ %
  \ifthenelse{\equal{#2}{}}
  {}
  {(Lösung auf Seite \pageref{sol:#2})\ }
}

\newcommand{\Uebungen}{
  \clearpage
  \section*{Übungen zu Kapitel \thechapter}
  \markboth{Übungen zu Kapitel \thechapter}{Übungen zu Kapitel \thechapter}
  \setcounter{equation}{0}
}

\newcommand{\Loesungen}[2][\clearpage]{#1\section*{Zu Kapitel \ref{#2}}}
\newcommand{\Loesung}[1]{\subsection*{Übung \ref{exc:#1} (Seite \pageref{exc:#1})}}

\renewcommand{\Phantom}{}


%%%%%%%%%%%%%%%%%%%%%%%%%%%%%%%%%%%%%%
% Text used in more than one chapter %
%%%%%%%%%%%%%%%%%%%%%%%%%%%%%%%%%%%%%%

\newcommand{\DefWort}{Das (\textit{lexikalische}) \textit{Wort} ist eine Repräsentation von paradigmatisch zusammengehörenden Wortformen.
Umgangssprachlich kann man von der Zusammenfassung aller möglichen Formen eines Wortes sprechen.
Für das lexikalische Wort sind die Werte nur für diejenigen Merkmale spezifiziert, die in allen Wortformen des Paradigmas dieselben Werte haben.
Die restlichen Werte werden gemäß der Position im Paradigma bei den konkret vorkommenden Wortformen des Wortes gesetzt.}

\newcommand{\ThePhrasenExOne}{dass [[dem Jungen] [die Mutter] [ein Eis] [geschenkt hat]]}
\newcommand{\ThePhrasenExTwo}{dass [[ein Eis] [die Mutter] [dem Jungen] [geschenkt hat]]}


%%%%%%%%%%%%
%%% TikZ %%%
%%%%%%%%%%%%

% Undefining node names.

\makeatletter

\long\def\ifnodedefined#1#2#3{%
  \@ifundefined{pgf@sh@ns@#1}{#3}{#2}}

\newcommand\aeundefinenode[1]{%
  \expandafter\ifx\csname pgf@sh@ns@#1\endcsname\relax
  \else
    \typeout{Undefining node "#1"}%
    \global\expandafter\let\csname pgf@sh@ns@#1\endcsname\relax
  \fi
}

\newcommand\aeundefinethesenodes[1]{%
  \foreach \myn  in {#1}
    {%
      \ifnodedefined{\myn}{%
      \expandafter\aeundefinenode\expandafter{\myn}%
    }{}
    }%
}

\newcommand\aeundefinenumericnodes{%
  \foreach \myn in {1,2,...,50}
    {%
      \ifnodedefined{\myn}{%
      \expandafter\aeundefinenode\expandafter{\myn}%
    }{}
    }%
}
\makeatother

% Drawing sonority diagrams.

\newcommand{\plo}{0}
\newcommand{\fri}{0.5}
\newcommand{\nas}{1}
\newcommand{\liq}{1.5}
\newcommand{\vok}{2}

% Save text.
\newcommand{\lastsaved}{}
\newcommand{\textsave}[1]{\gdef\lastsaved{#1}#1}

\newcommand{\SonDiag}[2][0]{%
  \begin{tikzpicture}
    \textsave{.}
    \tikzset{
      normalseg/.style={fill=white},
      extrasyll/.style={circle, draw, fill=white},
      sylljoint/.style={diamond, draw, fill=white}
    }
    \node at (0,\plo) {P};
    \node at (0,\fri) {F};
    \node at (0,\nas) {N};
    \node at (0,\liq) {L};
    \node at (0,\vok) {V};

    % Draw the helper lines if required.
    \ifthenelse{\equal{#1}{0}}{}{%
      \foreach \y in {\plo, \fri, \nas, \liq,\vok} {%
	\draw [dotted, |-|] (0.25, \y) -- (#1.75, \y);
      }
    }

    \foreach [count=\x from 1, remember=\x as \lastx] \p / \y / \g in #2 {
      \ifthenelse{\equal{\y}{-1}}{\textsave{.}}{%

	% Draw the node, either plain, as Silbenbgelenk, or as extrasyllabic.
        \ifthenelse{\equal{\g}{1}}{%
	  \node (\x) [sylljoint] at (\x, \y) {\p};
	}{%
	  \ifthenelse{\equal{\g}{2}}{%
	    \node (\x) [extrasyll] at (\x, \y) {\p};
	  }{%
	    \node (\x) [normalseg] at (\x, \y) {\p};
	  }
	}

	% Draw the connection unless the previous node was not or was empty.
	\ifthenelse{\NOT\equal{\lastsaved}{.}}{%
	  \draw [->] (\lastx) to (\x);
	}{}
	\textsave{1}
      }
    }
    \aeundefinenumericnodes
  \end{tikzpicture}
}

\forestset{
  Ephr/.style={draw, circle, thick, inner sep=1pt},
  Eobl/.style={draw, rounded corners, inner sep=3pt},
  Eopt/.style={draw, rounded corners, densely dashed, inner sep=3pt},
  Erec/.style={draw, rounded corners, double, inner sep=3pt},
  Eoptrec/.style={draw, rounded corners, densely dashed, double, inner sep=3pt},
  Ehd/.style={rounded corners, fill=gray, inner sep=3pt,
    delay={content=\whyte{##1}}
  },
  Emult/.style={for children={no edge}, for tree={l sep=0pt}},
  phrasenschema/.style={for tree={l sep=2em, s sep=2em}},
  decide/.style={draw, chamfered rectangle, inner sep=2pt},
  finall/.style={rounded corners, fill=gray, text=white},
  intrme/.style={draw, rounded corners},
  yes/.style={edge label={node[near end, above, sloped, font=\scriptsize]{Ja}}},
  no/.style={edge label={node[near end, above, sloped, font=\scriptsize]{Nein}}},
  sake/.style={tier=preterminal},
  ake/.style={
    tier=preterminal
    },
}



%%%%%%%%%%
% forest %
%%%%%%%%%%

\forestset{
  narroof/.style={roof, inner xsep=-0.25em, rounded corners},
}


\thispagestyle{empty}

\noindent\Large\textbf{Roland Schäfer: \textit{Einführung in die grammatische\\
Beschreibung des Deutschen. Dritte, überarbeitete\\
und erweiterte Auflage}}\\

\noindent\Large\textbf{Errata}\\[0.25\baselineskip]

\normalsize

Aktuelle korrigierte Fassung: \url{https://github.com/rsling/egbd}

\renewcommand{\arraystretch}{1.5}
\noindent\begin{longtable}{p{0.35\textwidth}p{0.65\textwidth}}
  S.~51, Z.~11      & werden → wird \\
  S.~64, Z.~18      & indefinites oder → definites oder \\
  S.~80, Z.~18      & sprechen daher → sprechen und daher \\
  S.~116, Z.~3      & Ein → Eine \\
  S.~121, Satz~5.3  & betont sind → betont sind, \\
  S.~125, Abs.~2, Z.~5 & Es bei normaler → Es wäre bei normaler \\
  S.~138, Abb.~5.8  &   \parbox{0.6\textwidth}{
                           \vspace{0.5\baselineskip}
                           \scalebox{0.75}{%
                           \SonDiag[8]{{ʃ/\fri/2, t/\plo/0, ʁ/\liq/0, ɔ/\vok/0, l/\liq/0, ç/\fri/0, s/\fri/2, t/\plo/2}}}
                           \vspace{0.5\baselineskip}
                         }\\
  S.~144, (37e)     &{} [klɛm+t] → [klɛmt] \\
  S.~147, Z.~−3 über D.~5.14 & zeigen. → zeigen dies. \\
  S.~172, Z.~2      & grundlegende → grundlegenden \\
  S.~188, Z.~3      & Positionen → Position \\
  S.~258, Abb.~9.2, rechts  &  \textit{\char`~en (S4b) → \char`~e (S4b)} \\
  S.~268, (49b)     & ist es mit → ist mit \\
  S.~273, Z.~11     & Dativ Plural der Substantive auf \textit{-n}  → Dativ Plural der Substantive auf \textit{-en} \\
  S.~324, Abb.~11.1 &  \parbox{0.6\textwidth}{
                       \scalebox{0.75}{%
                       \begin{forest}
                       [Silbe, calign=last
                         [Anfangsrand, ake, calign=first
                           [f][ʁ]
                         ]
                         [Reim, calign=first
                           [Kern,ake
                             [ɛ]
                           ]
                           [Endrand, ake, calign=last
                             [m][t]
                           ]
                         ]
                       ]
                     \end{forest}}} \\
  S.~324, Abb.~11.3 &  \parbox{0.6\textwidth}{
                       \scalebox{0.75}{%
                         \begin{forest}
                           [NP, calign=child, calign child=2
                             [AP, tier=preterminal
                               [\textit{rote}, narroof]
                             ]
                             [N, tier=preterminal
                               [\textit{Zahnbürsten}]
                             ]
                             [NP, tier=preterminal
                               [\textit{des Königs}, narroof]
                             ]
                             [RS, tier=preterminal
                               [\textit{die benutzt waren}, narroof]
                             ]
                           ]
                         \end{forest}}} \\
  S.~345, Übung~11.1, Nr.~4 & vergab ein [ Gäste-Stürmer → vergab [ein Gäste-Stürmer\\
  S.~349, Abb.~12.2 & \parbox{0.6\textwidth}{
                       \scalebox{0.75}{%
                         \begin{forest}
                            [\textbf{N}, calign=child, calign child=2
                              [\textbf{N}, tier=preterminal
                                [\it Kuchen]
                              ]
                              [Konj, tier=preterminal
                                [\it und]
                              ]
                              [\textbf{N}, tier=preterminal
                                [\it Sahne]
                              ]
                            ]
                         \end{forest}}} \\
  S.~351, unter Definition~12.1 &  Artikelwörter, links  → Links\\
  S.~351, unter Definition~12.1 &  Genetivs → Genitive\\
  S.~385, Z.~1 & um eingebetteten Konstituenten → um eingebettete Konstituenten \\
  S.~520, Übung~4.2 & \parbox{0.6\textwidth}{
                           \vspace{0.5\baselineskip}
                           \scalebox{0.75}{%
                             \SonDiag[6]{{ʃ/\fri/2, t/\plo/0, ʁ/\liq/0, ʊ/\vok/0, m/\nas/0, p͡f/\plo/0}}
                           }
                           \vspace{0.5\baselineskip}
                         }\\
  S.~520, Übung~4.4 & \parbox{0.6\textwidth}{
                           \vspace{0.5\baselineskip}
                           \scalebox{0.6}{%
                             \SonDiag[11]{{k/\plo/0, v/\fri/0, a͡ə/\vok/0, k/\plo/0, 0/-1/0, ʃ/\fri/2, p/\plo/0, a͡ɛ/\vok/0, 0/-1/0, z/\fri/0, ə/\vok/0}}
                           }
                           \vspace{0.5\baselineskip}
                         }\\
  S.~522, Übung~4.10  & \parbox{0.6\textwidth}{
                           \vspace{0.5\baselineskip}
                           \scalebox{0.75}{%
                             \SonDiag[7]{{f/\fri/0, ɛ/\vok/0, n/\nas/0, 0/-1/0, s/\fri/2, t/\plo/0, ɐ/\vok/0}}
                           }
                           \vspace{0.5\baselineskip}
                         }\\
  S.~561, Berko (1974) & \textbf{+} Wolfgang Eichler \& Adolf Hofer (Hrsg.) \\
  S.~562, Clark (1978) & Awareeness → Awareness \\
  S.~561–568           & Bei den englischsprachigen Referenzen ist die Groß- und Kleinschreibung nicht ganz einheitlich. Die Details werden hier nicht aufgelistet. S.~aktuelle Version auf GitHub.\\
\end{longtable}

\end{document}
