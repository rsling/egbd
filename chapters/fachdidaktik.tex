\chapter{Grammatik im Lehramtsstudium}
\label{sec:grammatikimlehramtsstudium}


\section{Grammatik in der Schule}
\label{sec:grammatikinderschule}

Die Frage nach der Rolle von Grammatik im Lehramtsstudium kann nicht losgelöst von der Rolle der Grammatik im Schulunterricht diskutiert werden.
In diesem Abschnitt werden daher 


%%%%%%%%%%%%%%%%%%%%%%%%%%%%%%%%%%%%%%%%%%%%%%%%%%%%%%%%%%%%%%%%%%%%%%%
\paragraph*{\citet{Feilke2012} -- Bildungssprache}

Bildungssprache setzt Verstehen und abstraktes Bedeutungswissen voraus (4)

Sprache als zentraler Gegenstand und Medium von Bildung (4, zitiert Roth 2010:574)

Bildungssprache werde zwar in vieler Hinsicht vorausgesetzt, aber nicht gelehrt. (4)

[Bildungssprache] sind die besonderen sprachlichen Formate und Prozeduren einer auf Texthandlungen wie Beschreiben, Vergleichen, Erklären, Analysieren, Erörtern etc.\ bezogenen Sprachkompetenz, wie man sie im schulischen und akademischen Bereich findet. (5)

[Die sprachlichen Mittel findet man auch] auch in Texten mit Alltagsthemen, die sachlich komplexe Verhältnisse darstellen. (5)

Vielmehr ist es die „Hauptaufgabe der Bildungssprache, zwischen Wissenschaft bzw. speziellem Sphärenwissen und Alltag zu vermitteln“ (Ortner 2009, S. 2232) (6)

Bildungssprache sei primär auf die Schriftsprache (bzw.\ schriftliche Situationen) bezogen, damit einher gehe eine Dekontextualisierung und damit erhebliche Abstraktion. (6)

Auf solche komplexen Verstehens- bzw. Formulierungsaufgaben bereitet ein rein formenorientierter Grammatikunterricht kaum vor. (8)

Explizieren (8): Sachverhalte für Lesende nachvollziehbar explizieren und fokussieren; komplexe Adverbiale, Attribute und Sätze, explizite Konnexion (konditional, final, adversativ usw.)

Verdichten (8): Sachverhalte ohne finites Verb ausdrücken; Nominalisierungen, Komposita, Partizipialattribute, präpositionale Adverbiale, FVG

Verallgemeinern (9): dekontextualisiert und generalisierte Darstellung; generische dritte Person, generischer Artikel, generisches Präsens

Diskutieren (9): Sachverhalte als Gegenstände mit variierendem Wahrheitsgehalt; Modalverben, Konjunktiv, Konzessiva

[Die Bildungssprache] ist deshalb gleichermaßen ein Bildungskapital, wie sie eine Hürde für das Verstehen sein kann. Auch bezogen auf die Produktion gibt es besondere Ansprüche der Bildungssprache: Erwartet wird, dass die Schüler das für das Verstehen Notwendige grammatisch, lexikalisch und textlich explizit machen. Diese Explizitformenerwartung (Maas 2010, Feilke 2012) ist nicht nur kommunikativ begründet. Gerade im Kontext der Schule hat das Explizieren nicht nur mit Verständigung zu tun, sondern verweist auf die kognitive Funktionalisierung der Sprache für Zwecke des Lernens: einen Begriff definieren, ein Konzept strukturieren, einen Text zusammenfassen usw. (11)

Die Situierung der Sprachanforderungen ist ein wichtiges Merkmal bildungssprachlicher Didaktik. Es sollten Kontexte konstruiert werden, in denen der Formengebrauch ebenso wie deren Reflexion pragmatisch motiviert ist. (12)

%%%%%%%%%%%%%%%%%%%%%%%%%%%%%%%%%%%%%%%%%%%%%%%%%%%%%%%%%%%%%%%%%%%%%%%
\paragraph*{\citet{Schroeterbrauss2013}}

Bildungssprachliche Kompetenzen im Fachunterricht stärken und global entwickeln.
Deutschunterricht kann dann mehr zu den Grammatik-Kernkompetenzen zurückkehren.
Benachteiligung nicht-erstsprachlicher Lerner in Deutschland laut PISA.

%%%%%%%%%%%%%%%%%%%%%%%%%%%%%%%%%%%%%%%%%%%%%%%%%%%%%%%%%%%%%%%%%%%%%%%
\paragraph*{\citet{Eisenberg2013c}}

Grammatik nicht als Selbstzweck (7)

Lehrenden wird die Sprache der Lernenden anvertraut.
Sie müssen sprachliche Defizite erkennen und erkennen, welche dialektale Prägung vorliegt, Interferenzen mit anderen Sprachen. (7) Und (RS): Sie müssen sprachliche Leistungen bewerten.

Lehrende müssen sich also auf die Sprache der Lernenden einlassen und sie metasprachliche verstehen.
Dazu müssen sie etwas über Sprache wissen, und dieses Wissen ist primär Grammatikwissen.
Im Studium wird also Grammatik für Lehrende vermittelt. (7)

Transferproblem: Grammatik -> Sprache können (8)
Sprachwissenschaft in Hochschulen oft wenig an Schulrelevanz interessiert (8)

Im Vorwort von \citet[2]{KoepckeZiegler2013}: Inhalte des Grammatikunterrichts unverändert, empirisch nicht geklärt, was hilft und was nicht (9)

Beherrschung der Orthographie setzt voraus, dass die Anwendenden grammatisch versiert sind (9--10); und \citet{Duerscheid2013} sagt korrekt Sprachbetrachtung ist Schriftbetrachtung (13)

Sprachreflexion relevant für die Beherrschung des Standards als einer von vielen Varietäten (10), ebenso Textsortenspezifik (10), aber das funktioniert nur, wenn man die verschiedenen Varietäten (aller Art) als System begreift, das auf zugrundeliegende Möglichkeiten eines grammatischen Systems aufbaut (11)

Modellunabhängige Formulierung von grammatischen Regularitäten (11)

Orientierung am Gebrauch nicht so einschlägig, wenn es um Normbeherrschung geht (12); deswegen (RS) auch Unterschied zwischen meiner Forschung und meiner Lehre bzw. dem Buch

Sekundarstufe II ist grammatikfreie Zone (12), Tendenz von Schulgrammatiken ist die Reduktion von Theorie und Methode

BA-Bücher handeln vor allem von Sprachwissenschaft statt von Sprache (13)

%%%%%%%%%%%%%%%%%%%%%%%%%%%%%%%%%%%%%%%%%%%%%%%%%%%%%%%%%%%%%%%%%%%%%%%
\paragraph*{\citet{Eisenberg2004}}

Sprachliches Verhalten zu beurteilen heißt, es verstehen zu müssen. (4)

Literalisierung von Lernenden führt zu einer Restrukturierung ihrer unbewussten Grammatik (4) Siehe auch fast wörtlich so \citet[78]{Portmanntselikas2011}.

KMK formuliert keine konkreten Grammatikkenntnisse für Schüler, dafür aber hohes Niveau in sprachlichem Wissen für Lehrer (6)

Unter anderem wird von Schülern registersicheres Sprechen und Schreiben sowie "`Standardsprache"' verlangt (6)

Mit Bezug auf \citet{Braun1979}, eine Untersuchung von Zweifelsfällen in der Bewertung durch Studierende und Lehrpersonen ist festzustellen, dass diese meist mehr Fehler finden, als sie sollten.
Dabei besteht zwischen der untersuchten Gruppe nicht immer Einigung, welche \textit{eine} Lösung die richtige sein soll.
Das entspricht der Beobachtung des Autors (RS), dass oft statt einer differenzierten grammatischen\slash graphematischen Analyse und einem Abgleich mit dem Standard, wo er denn definiert ist, persönliche Richtigkeitsurteile vehement vertreten werden, (angehende) Lehrpersonen also ihren eigenen Stil als Bewertungsgrundlage hernehmen.

Neben ausufernden Formenpluralismus und Regulierungswut (bzw. die Einheitslösung) stellt Eisenberg die Einordnung sprachlicher Formen ins System und die genaue Analyse der Bedeutungs- und Gebrauchsunterschiede von Zweifelsfällen. (8-9) Genau das tut die Analyse von Zweifelsfällen als Alternationen. (RS)

Beispiel für interne Funktion: Schwa-Tilgung in der ersten Singular (9--11)

Grammatik für das Lehramtsstudium: KEINE Fixierung auf Zweifelsfälle (11), denn (RS) die Zweifelsfälle bleiben ohne das System eine nicht verstehbare Liste von Einzelfällen. Außerdem (PE): Methodik statt Fakten! (11)

Kein Spielen mit der Norm beim Orthographieunterricht! Das kann eine Hauptquelle von Rechtschreibschwächen. (12) Es gibt reichlich Fehlertypologien, die Lehrpersonen kennen sollten, aber (RS) ihr Verständnis erfordert die Kenntnis des gesamten Systems.

Eine methodisch-didaktisch erfolgreiche Aufarbeitung setzt explizites und (konzeptuell) vollständiges Wissen bei Lehrpersonen voraus. (14, stark RS-formuliert)

Schriftspracherwerb stellt eine Rekonstruktion der Grammatik dar und darf nicht behindert werden. (12, 14)
Die Schrift führt zur Ausbildung eines metasprachlichen Bewusstseins. (14)
Damit einher geht die Fähigkeit zu dekontextualisierter Sprache. (15)
Kinder lernen in der Schule eine neue Sprache, nicht bloß das Schreiben einer bereits erworbenen Sprache. (15)
Erlernen von neuen Stilen (eher: Registern, RS), die mit spezifischen grammatischen Mitteln verknüpft sind (z.B. der explizierende Stil mit kausalen und konditionalen Konstruktionen, komplexen Tempusbezügen usw.) (15)

Ein (ungelöstes, RS) Problem beim Schulunterricht ist die Frage, wie das explizite (Grammatik-)Wissen in tatsächliches automatisches Sprachverhalten umgesetzt werden kann.
Aber das befreit die Lehrpersonen nicht von der Pflicht, das explizite Wissen zu haben.
Es geht nicht darum, im Studium Wissen zu vermitteln, das später in der Schule so weitergegeben wird, sondern um die Vermittlung von Wissen, das für einen erfolgreichen Sprachunterricht notwendig ist. (19)
>>
Nur das hilft auch bei der Einschätzung des sprachlichen Hintergrunds von lernenden Kindern. (22)
Erst auf dem Hintergrund umfassenden Wissens kann nach der methodisch-didaktischen Aufbereitung gefragt werden. (22)


Der Linguistik werde zu viel und der Sprache zu wenig Raum eingeräumt. 
\textit{Unser Gegenstand ist die Sprache, nicht die Sprachwissenschaft.} (22)

\textit{Das Verhältnis zur Sprache, das als '`metasprachliche Kompetenz'', durch '`Reflexion über Sprache'' und `'Transfer von explizitem zu implizitem Wissen'' von den meisten Lehrplänen gefordert wird, können die Schüler nicht entwickeln, wenn es die Lehrer nicht haben.} (23)


\paragraph{\citet{Portmanntselikas2011}}

Der späte Spracherwerb dreht sich um literale Kompetenzen (Schrift, Hochsprache, Fachsprachen, literale Register, Unterschied zwischen Bedeutung und pragmatischer Funktion, Textkompetenzen \zB in Form von Kohärenz) (71)

Auseinandersetzung mit Orthographie erstreckt sich über die Schulzeit hinaus. (72)

Orale Korrektheit ist im Bezug auf Literalität zu werten. (72)

Grammatikunterricht steuert diesen Erwerb und entdeckt/korrigiert Probleme und muss daher fundamentale Regularitäten vermitteln, da sonst der systematische Charakter von Sprache nicht zu Geltung kommt. (72)

Problem: Wissen über Sprache kann nicht in automatische und spontan funktionierende Mechanismen der Sprachproduktion umgesetzt werden. (73)

\citet[75--79]{Portmanntselikas2011} beschreibt ein Sprachproduktions- und Sprachverarbeitungsmodell, das primär unbewusste spontane Mechanismen annimmt.
Es gibt jedoch Rückkopplungsschleifen, die die bewusste Kontrolle des produzierten sprachlichen Materials erlauben (76), und die in vielen bildungssprachlichen Kontexten standardmäßig zum Einsatz kommen (die meisten schriftsprachlichen Kontexte [auch 77], Normdruck, explizite Sprachreflexion).

Im Laufe der Erwerbskarriere verschieben sich die Schwerpunkte von orthographischer und grammatischer Korrektheit zu kontextueller und funktionaler und stilistischer Angemessenheit, "`Grammatisches spielt aber fast überall mit eine Rolle"'. (77--78)

Im Laufe der Restrukturierung der Sprachkompetenz durch Schriftspracherwerb setzt der Zweifel an der Angemessenheit der eigenen produzierten Sprache ein. (78--79)
Der Zweck davon ist die Produktion von informativerer, für die Hörenden bzw.\ Lesenden optimalerer Sprache. (79)

Grammatisches Wissen\ldots

Zunächst ist für die Beherrschung von Sprache immer das Erkennen und Unterscheiden von Formen relevant, auch wenn es meist implizit geschieht. (79)
Inwiefern dabei Explizierungen helfen, ist nicht abschließend geklärt. (80)

(Frei paraphrasiert, RS:) Für einen Sprachunterricht, der eine systematische und nicht bloß inselhaft-episodische Wirkung hat, ist muss auf begrifflich fassbare Regularitäten zurückgreifen können, und die relevanten Probleme müssen gezielt in den Aufmerksamkeitsfokus gebracht werden. (80)

Grammatisches Wissen, das nur exemplarisch und damit inselhaft bzw.\ nicht systematisch eingeführt wird, reicht nicht aus, weil die Reichweite der entsprechenden Regularität, die mit einem systematische Zugang einhergehen würde, nicht notwendigerweise erkannt wird. (81)
Trotzdem erfolgt der Erwerb immer anhand von konkreten Einzelbeispielen, und viele bildungssprachlich relevante Phänomene sind im Input nicht eben häufig.
Dem Unterricht kommt also die wichtige Funktion zu, das Erlernen dieser Regularitäten durch Auswahl und Zusammenstellung der Beispiele und ihre explizierende Einordnung, das ans ich unter erschwerten Bedingungen stattfindet, zu erleichtern. (83)

Der Sprachunterricht schafft eine Kultur der \textit{habituellen Aufmerksamkeit} (84) und versetzt Lernende dadurch in die Lage, ihre eigene Sprache zu reflektieren, (RS) wo dies nötig wird, um bildungssprachlich angemessene Sprache zu produzieren.
Grammatische Begriffe sind in der Sprachwirklichkeit der Lernenden durchaus entbehrlich, aber das explizite Wissen um die Sprache erschließt erst die Sichtweise auf Sprache, die das kontrollierte Beherrschen von Standard- und Bildungssprache erst möglich macht. (85, frei interpretiert)

Es ist nicht einzusehen, warum wir eine analytische Einsicht in die Chemie oder Physik schulisch Lernenden fordern sollten, aber keine analytische Einsicht in Sprache.
Die analytisch-systematische Sicht auf die Welt prägt die Kultur, die unsere schulischen Inhalte bestimmt, und sie darf vor Sprache nicht halt machen. (85, sehr frei)










\section{Form und Funktion in der Grammatik}
\label{sec:formundfunktionindergrammatik}

Zunächst ist festzustellen, dass frühere Auflagen dieses Buches aufgrund bestimmter Formulierungen und Analysen teilweise missverstanden wurden, und dass dieses Missverständnis direkt die Tauglichkeit des Buches für das Lehramtsstudium berührt.
Ein anonymes Gutachten der ersten Auflage attestierte dem Buch (recht kritisch) einen radikal formalistischen Ansatz, der grammatische Mittel vollständig losgelöst von ihrer Funktion betrachtet, und der damit das Wesentliche der Sprache, also ihre bedeutungsvermittelnden und kommunikativen Funktionen außer acht lässt.
So radikal klangen zugegebenermaßen auch einige Formulierungen in den einleitenden Kapiteln des Buches, die bis zur dritten Auflage erheblich abgetönt, aber bewusst nicht vollständig entfernt wurden.
Für die schulische Lehre wäre solch ein radikaler Ansatz jedoch völlig verfehlt und geradezu anachronistisch, und auch in der universitären Lehre wäre ein rein formorientierter Ansatz hochgradig fragwürdig.%
\footnote{Das Adjektiv \textit{anachronistisch} ist hier auf Basis der Geschichte des schulischen Unterrichts in deutscher Grammatik zu verstehen, die \zB in den Abschnitten~3.1 und~3.2 aus \citet{Bredel2013} zusammengefasst wird.}
Das Buch versteht sich aber einerseits nicht als exklusive Lektüre für das gesamte Studium.
Es sollte also von weiterer Lektüre flankiert werden, die dann das gesamte relevante Spektrum der Linguistik des Deutschen abdeckt.
<++>

Außerdem bewies dieser Text trotz einer stellenweise formlastigen Rhetorik von der ersten Auflage an, dass es die Beziehung von Form und Funktion durchaus berücksichtigt, zum Beispiel bei der Beschreibung der Intonation (Kapitel~\ref{sec:phonologie}), der Flexionskategorien (Kapitel~\ref{sec:nominalflexion} und ~\ref{sec:verbalflexion}) oder bei der Behandlung von semantischen Rollen und Passivphänomenen (Kapitel~\ref{sec:relationenundpraedikate}).

Es muss hier jedoch vor allem zwischen \textit{systeminterner} und \textit{systemexterner} (\zB kommunikativer) Funktion unterschieden werden.\index{Form und Funktion}
Das System der vier nominalen Kasus ist zum Beispiel nur äußerst schlecht mit \textit{systemexternen} Funktionen zu verbinden, die direkt irgendetwas mit Semantik, Pragmatik, Textaufbau und Argumentationstechniken etc.\ zu tun haben.
Seine \textit{systeminterne} Funktion im grammatischen System ist allerdings von erheblicher Wichtigkeit, insbesondere angesichts der Möglichkeiten des Deutschen, die Bestandteile von Sätzen vergleichsweise frei im Satz zu positionieren.
Der Kasus eines Nomens kodiert nämlich vor allem die Relation dieses Nomens zu einem Verb (und im Fall des Genitivs primär zu einem anderen Nomen), und zwar in starker Abhängigkeit von teilweise auch semantisch motivierbaren Typen von Verben.%
\footnote{Ein Feuerwerk der systeminternen funktionalen Betrachtung ergibt sich in diesem Zusammenhang durch das Einbeziehen der Kasussynkretismen, also des Zusammenfalls von Formen.
Dieser fällt je nach Genus, Flexionsklasse und Numerus ganz unterschiedlich aus.
Zudem müssen Substantive, Adjektive und Artikel insgesamt betrachtet werden, da sie auf unterschiedliche Weise an der Kasusmarkierung beteiligt sind und einander diese sogar gegebenenfalls abnehmen.
Kapitel~\ref{sec:nominalflexion} beschäftigt sich ausführlich mit dem Kasussystem.}
Während also Kasus nicht direkt semantisch interpretierbar ist, ist er dennoch von großer Bedeutung für die Konstruktion der Bedeutung von Sätzen.
Darauf wird in dieser Einführung bei zahlreichen Gelegenheiten vertiefend eingegangen.
Wir geben Definition~\ref{def:internexternefunktion}, die sich auf Konzepte rückbezieht, die in Kapitel~\ref{sec:grammatik} eingeführt wurden.
Kapitel~\ref{sec:grundbegriffedergrammatik} führte bereits vertiefend in systeminterne Funktionen ein, zum Beispiel mit Begriffen wie \textit{Merkmal} und \textit{Wert} oder \textit{Relationen} (\textit{Strukturbildung}, \textit{Rektion} und \textit{Kongruenz}).

\Definition{Systeminterne und systemexterne Funktion}{\label{def:internexternefunktion}%
Sprachliche Formen (Formen von Wörtern, Sätzen usw.) haben durch die überwiegende Regelmäßigkeit ihrer Bildungen und Kombinationsmöglichkeiten eine Systematik (die Grammatik).
Systeminterne Funktionen grammatischer Mittel sind solche, die innerhalb dieses Systems gelten, \zB die Funktion von Kasus bei der Markierung von Beziehungen zwischen Nomina und Verben.
Systemexterne Funktionen grammatischer Mittel sind ihre Bedeutungen, ihre systematischen Beiträge zur Konstruktion von Kommunikationssituationen und Texten, ihre Art, bestimmte Register und Stile zu markieren usw.
\index{Funktion!systemintern und systemintern}
}

Es ist nun nicht davon auszugehen, dass sich interne und externe Funktionen stets sauber voneinander trennen lassen, und eine umfassende Betrachtung ist sicherlich das Ziel der Grammatikforschung und des Studiums.
Die Motivation, diese Einführung dennoch stark (wenn auch nicht dogmatisch) an der Form sprachlicher Äußerungen und ihren systeminternen Funktionen auszurichten, begründet sich aus meiner persönlichen Lehrerfahrung im Fach Deutsche Philologie.
Während diese Erfahrung gewiss subjektiven und anekdotischen Charakter hat, wurde sie durch \citet{SchaeferSayatz2017a} auf methodisch stringente Weise bestätigt, worauf in Abschnitt~\ref{sec:grammatikkentnissevonstudierenden} ausführlicher eingegangen wird.
Programmatisch formuliert setzt die \textit{systematische} Betrachtung von Form und Funktion sprachlicher Einheiten -- um die es nach meiner Auffassung sowohl in der Schule bzw.\ Fachdidaktik als auch in der universitären Ausbildung bzw.\ Fachwissenschaft geht -- eine genaue Kenntnis der Systematik der Formen und ihrer systeminternen Funktionen voraus.%
\footnote{Abweichende Versuche der Didaktik früherer Jahrzehnte bestätigen wahrscheinlich dieses Programm indirekt eher als dass sie es widerlegen, siehe Abschnitt~\ref{sec:deutschunterrichtunddiewellenderdidaktik} und die dort zitierte Literatur.}
In der universitären Lehre aber davon auszugehen \textit{oder zu verlangen}, dass Studierende vor dem ersten Semester bereits eine ausreichende Kenntnis des formalen grammatischen Systems erworben haben, ist allerdings verfehlt und potentiell fatal für ihre Ausbildung und ihre spätere berufliche Praxis.
Auch wenn es vielleicht nicht immer eine Mehrheit der Studierenden betrifft, so wird doch in der frühen Phase des Studiums immer wieder gerne Kasus direkt mit Bedeutungen identifiziert und mit den unzureichenden \textit{Grammatikerfragen} ("`\textit{wer oder was}"' usw.) analysiert (s.\ Abschnitt~\ref{sec:kasus}), grammatisches Genus wird mit Bedeutungsklassen identifiziert, Subjekte werden als der "`Satzgegenstand"' \textit{definiert}, aber mit dem Nominativ \textit{identifiziert}, Wortarten werden semantisch motiviert usw.%
\footnote{Gleichermaßen werden nahezu rein formale Phänomene nicht zuverlässig beschrieben.
Dass zum Beispiel die \textit{Klatschmethode} (s.\ Abschnitt~\ref{sec:silben}) bei der Beschreibung der Silbentrennung nicht besonders weit trägt, ist Studierenden meist bewusst.
Die Fähigkeit, die Silbifizierung im Deutschen und ihre graphematischen Folgen anderweitig und präzise zu beschreiben, kann und darf hingegen trotzdem nicht vorausgesetzt werden.}
Während alle diese Phänomene (einschließlich der Wortklassen) hochinteressante und relevante Beziehungen zur Bedeutung sowie zahlreiche andere externe Funktionen haben, haben sie doch zunächst eine formale Seite und Systematik, die für jede weitere Betrachtung bekannt sein muss.
Die insgesamt stark formorientierte Ausrichtung des Buches entspringt also der Überzeugung, dass vor allen funktionalen Betrachtungen, die selbstverständlich Teil des Studiums sein sollten, mit einer gründlichen Analyse der Form und der systeminternen Funktionen begonnen werden sollte, die auch mit eingebrannten bedeutungsfixierten Didaktisierungen aus der Schulzeit (wie den semantischen Definitionen von Wortklassen) aufräumt.

Komplexere Phänomene wie -- um ein arbiträres Beispiel herzunehmen -- Relativsätze werden zudem von Studierenden zu Studienbeginn formal oft gar nicht verstanden, und eine Diskussion über ihre Funktion erübrigt sich damit.
Sehen wir uns die Folgen einer unzureichenden formalen Analyse etwas genauer an.
Es ist gewiss ein vordergründig rein formaler Sachverhalt, dass ein Relativsatz ein im Prinzip vollständiger Satz ist, der aber so konstruiert wird, dass eins der in ihm vorkommenden "`Satzglieder"' (besser gesagt eine der Ergänzungen und Angaben, s.\ Abschnitte~\ref{sec:valenz} und~\ref{sec:konstituentenundsatzglieder}) durch ein spezielles Pronomen realisiert wird, welches sich dann auf ein Bezugsnomen außerhalb des Relativsatzes bezieht.
Das Bezugsnomen steuert die semantische Interpretation (Bedeutung) des Relativpronomens bei.
Wir erhalten (\ref{ex:formundfunktionindergrammatik002}) parallel zu (\ref{ex:formundfunktionindergrammatik001}) als einen einfachen Fall eines Relativsatzes.

\begin{exe}
  \ex
  \begin{xlist}
    \ex{\label{ex:formundfunktionindergrammatik001} Eine\slash die Kommilitonin hat gelacht.}
    \ex{\label{ex:formundfunktionindergrammatik002} eine\slash die Kommilitonin, die gelacht hat}
  \end{xlist}
\end{exe}

Es fällt rein formal auf, dass zumindest die Verbformen im Relativsatz andere Positionen einnehmen als im unabhängigen Satz.
Dass (und \textit{auf welche Weise}) dies völlig regelmäßig geschieht, muss allerdings erst einmal verstanden werden, um die Beziehung zwischen den beiden Konstruktionen überhaupt wahrnehmen zu können.
Wird das durch einen Relativsatz modifizierte Substantiv wie in (\ref{ex:formundfunktionindergrammatik003}) in einer übergeordneten Satzperiode verwendet, wird die systeminterne Funktion schnell klar.
Der Relativsatz ermöglicht es uns stark vereinfacht gesagt, zwei im Grunde vollständige Sätze zu einem zu kombinieren (Hypotaxe).
Im Gegensatz zu einer einfachen aufzählenden Verbindung (Parataxe) wie in (\ref{ex:formundfunktionindergrammatik004}) nutzt der Relativsatz die Identität eines nominalen Bezugs (also der Bedeutung zweier Nomina) und vermeidet dadurch die zweimalige unabhängige Aufnahme der semantischen Referenz.

\begin{exe}
\ex
\begin{xlist}
  \ex{\label{ex:formundfunktionindergrammatik003} Die Kommilitonin, die gelacht hat, heißt Camilla.}
  \ex{\label{ex:formundfunktionindergrammatik004} Eine Kommilitonin hat gelacht, und die\slash sie heißt Camilla.}
\end{xlist}
\end{exe}

Die systeminterne Funktion hat also eine semantische Funktion in ihrem Gefolge, die in ihrer Systematik nur als Folge der formalen zu verstehen ist.
Die Erkenntnis, dass darüberhinaus die ebenfalls externe \textit{textuelle} oder \textit{rhetorische Funktion} eines Relativsatzes in den Bereichen des \textit{Verdichtens} und \textit{Präzisierens} anzusiedeln ist, kann nur in Zusammenhang mit den ersten beiden Schritten vollzogen werden.
Ohne diese Schritte kann überhaupt nicht sicher identifiziert werden, was ein Relativsatz denn überhaupt sein soll, woran wir ihn erkennen und wie wir ihn formulieren.
Selbst wenn wir davon ausgehen, dass (\ref{ex:formundfunktionindergrammatik003}) und (\ref{ex:formundfunktionindergrammatik004}) bedeutungsgleich (aber eventuell rhetorisch verschieden) sind, kommen wir nicht umhin, festzustellen, dass ihre Verwendung hochgradig stil-, register- und medienspezifisch ist, womit weitere externe Funktionen angesprochen werden.
In fast allen bildungs- und schriftsprachlichen Kontexten ist nämlich davon auszugehen, dass (\ref{ex:formundfunktionindergrammatik004}) unangenehm auffällt.
Verdichtungen und Präzisierungen, die komplexere formale Mittel wie Relativsätze einsetzen, kennzeichnen nämlich genau solche Kontexte, und wir irritieren unser sprachliches Gegenüber im Ernstfall erheblich, wenn wir in der Bildungssprache zu ausufernder Parataxe greifen.

Ganz unabhängig davon sehen wir an (\ref{ex:formundfunktionindergrammatik004}), dass die systeminternen Interaktionen weiter reichen, als uns vielleicht lieb ist.
Die parataktische Version zwingt uns nämlich, zweimal einen Artikel bzw.\ ein Pronomen zu verwenden.
Hier wurde zuerst der indefinite \textit{eine} und dann das Definitpronomen \textit{die} oder alternativ das Personalpronomen \textit{sie} verwendet.
Die andere mögliche Variante wäre die in (\ref{ex:formundfunktionindergrammatik005}), wobei diese sich allerdings bereits im Grenzbereich zwischen stilistischem Mangel und Ungrammatikalität bewegt.

\begin{exe}
  \ex[?]{\label{ex:formundfunktionindergrammatik005} Die Kommilitonin hat gelacht, und die\slash sie heißt Camilla.}
\end{exe}

Gehen wir noch weiter und sehen uns informationsstrukturelle Funktionen an, stellen sich auch diese als mit der Form eng verknüpft heraus.
Weder (\ref{ex:formundfunktionindergrammatik004}) noch (\ref{ex:formundfunktionindergrammatik005}) sind nämlich wirklich robust bedeutungsgleich mit (\ref{ex:formundfunktionindergrammatik003}).
Ohne tief in die Details zu gehen, zeigt (\ref{ex:formundfunktionindergrammatik007}) durch eine besondere Betonung (also ein formales Mittel), dass wir dank des Relativsatzes Ausdrucksmöglichkeiten haben, die wir mit der parataktischen Version schlichtweg nicht haben.
Die in Majuskeln geschriebenen Silben sind mit hervorhebender Betonung zu lesen.

\begin{exe}
  \ex{\label{ex:formundfunktionindergrammatik007} DIE Kommilitonin, die geLACHT hat, heißt Camilla.}
\end{exe}

In geschriebenen Texten markieren wir Intonation normalerweise nicht.
Dort werden die entsprechenden Sätze ohne besondere Markierungen aber trotzdem ähnlich verwendet, oder es kommen andere Markierung wie zum Beispiel Adverbiale oder besondere Wortstellungen zum Zug.
Als Beispiel wird in (\ref{ex:formundfunktionindergrammatik008}) ein Aufzählungskontext gegeben.
Ohne Relativsätze wären wir hier im Ausdruck erheblich eingeschränkt.

\begin{exe}
  \ex{\label{ex:formundfunktionindergrammatik008} Die etwas ruhigere Kommilitonin heißt Kiki. Die Kommilitonin (hingegen), die gelacht hat, heißt Camilla.}
\end{exe}

Gleichermaßen gibt es für Sätze wie (\ref{ex:formundfunktionindergrammatik009}), die allgemeine Aussagen formulieren, kaum eine semantisch adäquate parataktische Variante, die eine vertretbare Länge hat und stilistisch unauffällig ist.

\begin{exe}
  \ex{\label{ex:formundfunktionindergrammatik009} Die Mitspielerin, die zuletzt lacht, gewinnt.}
\end{exe}

Wir feiern mit dem Relativsatz also ohne weiteres ein grammatisches Fest, bei dem Wissen aus unterschiedlichsten Teilen der Grammatik abgerufen wird.
Um die semantischen, textuellen und rhetorischen, stilistischen und registerbezogenen Funktionen von Relativsätzen zu benennen, müssen wir in der Lage sein, Relativsätze zu erkennen, ihre Form zu analysieren, und dieses Wissen in einen größeren Zusammenhang zu stellen.

Wir sind weiter oben von der Frage ausgegangen, welche Kenntnisse Studierende in die Lehramtsausbildung mitbringen, und wir haben argumentiert, dass formale Analysen eine Bedingung für eine systematische funktionale Betrachtung von Sprache sind.
Angesichts der Komplexität des grammatischen Systems und der externen Funktionen grammatischer Mittel (wie am Beispiel des Relativsatzes vorgeführt) können wir davon ausgehen, dass es zielführender ist, im Studium einen systematischen Neuanfang zu wagen, der sich nicht auf vermeintlich vorhandenes Schulwissen stützt.
Die fundamentale Rechtfertigung dieses Ansatzes liegt zudem in der oft übersehenen Tatsache, dass der Grammatikunterricht in der Schule grundlegend andere Ziele verfolgt als die universitäre Ausbildung von zukünftigen Lehrpersonen.
Darum wird es in diesem Kapitel weiter unten noch ausführlicher gehen.

Nicht kontrovers ist allerdings die Minimalforderung, dass Lehrkräfte, die von Universitäten und Hochschulen mit einem erfolgreichen Studienabschluss in das Schulfach Deutsch entlassen werden, diejenigen Kenntnisse mitbringen müssen, die sie Kindern und Jugendlichen vermitteln sollen.
Das Studium muss diese Kenntnisse über den weitaus höheren Anspruch des spezifischen Studienwissens hinaus sicherstellen.
In diesem Zusammenhang ist es nun bemerkenswert, dass Dozierende an Universitäten oft keinerlei Daten darüber haben, was sie bei Studierenden voraussetzen können und was nicht.
Hier ist dringend nachzuhelfen, vor allem weil die Gefahr besteht, dass ein in seinem oft nicht gut definiertes \textit{Schulwissen} von Lehrpersonen an Universitäten stillschweigend vorausgesetzt und auf dieses "`aufgebaut"' wird.
In diesem Zuge treten wir auch aus dem Nebel des Anekdotischen heraus und sehen empirisch klar in Abschnitt~\ref{sec:kenntnisseerwartungenundmotivationvonstudierenden}.


\section{Kenntnisse, Erwartungen und Motivationen von Studierenden}
\label{sec:kenntnisseerwartungenundmotivationvonstudierenden}

\subsection{Grammatikkenntnisse von Studierenden}
\label{sec:grammatikkentnissevonstudierenden}

In \citet{SchaeferSayatz2017a} berichten Ulrike Sayatz und ich ein Experiment (im wissenschaftlichen Sinn des Worts), mit dem wir den letzten Punkt des vorangehenden Abschnitts untersucht haben.
Konkret haben wir evaluiert, welche schulgrammatischen Kenntnisse Studierende am Anfang des Studiums haben.
Es ging uns also um die berüchtigten \textit{schulischen Vorkenntnisse}.

Für unsere Studie gab es verschiedene Vorläufer.
Besonders medienwirksam wurden 2007 die Ergebnisse von einer Art Studieneingangstest berichtet, der in Bayern durchgeführt wurde und laut der journalistischen Aufbereitung \citep{Spiegelonline2007,Szonline2007} ergab, dass Studierende der Germanistik nicht ausreichend durch die Schule auf das Studium vorbereitet würden.
Nach einer Umrechnung der Ergebnisse in Noten wäre über die Hälfte der Teilnehmenden durchgefallen.
Die betroffenen Studierenden dürften sich allerdings durch die Art der Formulierungen, mit denen in der Presse operiert wurde, lediglich gedemütigt und keinesfalls motiviert fühlen.
Spiegel Online spricht von einem "`Grammatik-Fiasko"', bei dem Studierende "`mit Karacho durchgefallen"' seien.
Noch weiter treibt es die Süddeutsche Zeitung Online, die im genannten Artikel ausführt, die "`Germanistik-Studenten"' hätten sich "`blamiert"', und viele von ihnen seien "`Grammatik-Nieten"'.
Die Unterschiede in der konkreten Wortwahl zeigen, dass es sich hier um eine zwar gleichermaßen reißerische, aber dennoch voneinander unabhängige journalistische Aufbereitung handelt, nicht etwas um wörtliche Zitate der Durchführenden des Tests.
Trotzdem schreibt die Presse weiter, die "`Professoren [seien] erschüttert"' (Spiegel Online), und es müsse folglich "`in der Schule wieder mehr Grammatik gepaukt"' werden (Süddeutsche Zeitung Online). 
Einen größeren Schaden an der Sache können Medienberichte kaum auslösen.
In der Presse werden vor allem die Studierenden geradezu als die Schuldigen aufgebaut und entsprechend als blamable Nieten usw.\ tituliert.
Für die intendierte Sache, also eine Stärkung des Grammatikunterrichts an Schulen, leistet dies in jedem Fall keinen Beitrag.
Höchstens lehnt sich das angepeilte Zeitungspublikum zufrieden in der Gewissheit zurück, dass eben die Schule und die Jugend heutzutage nichts mehr taugen.%
\footnote{Ich wiederhole, dass ich mich hier auf die journalistische Aufarbeitung beziehe.
Diese Formulierungen stammen gewiss nicht von den Durchführenden der Studie.}

Unabhängig von der journalistischen Verarbeitung gibt es aber ein weiteres Problem mit solchen Tests.
Das Ziel der Durchführenden in Bayern war es wohl, den Grammatikunterricht an Schulen zu stärken, indem entsprechende Wissenslücken aufgezeigt wurden.
Das Argument scheint dabei gewesen zu sein, dass Studierende nach ihrer Schulzeit nicht hinreichend auf das Germanistikstudium vorbereitet seien.
Das ist allerdings, wie weiter unten in diesem Kapitel argumentiert wird, überhaupt nicht das Ziel des schulischen Unterrichts in der Grammatik des Deutschen, ebensowenig wie es das Ziel des Geschichtsunterrichts ist, auf ein Studium der Geschichtswissenschaft vorzubereiten.
Noch viel weniger ist das Ziel des Deutschunterrichts das "`Pauken von Grammatik"'.
Sowohl die Fachdidaktik des Deutschen als auch die didaktisch informierte germanistische Sprachwissenschaft wird selbstverständlich einen Nürnberger Trichter für deklaratives schulisches Grammatikwissen als Absurdität zurückweisen.

Neben diesem Versuch eines Eingangstests gibt es eine aktuelle aktive Diskussion um Funktionen und Nutzen von Studieneingangstests (siehe vor allem \citealt{Schindler2016}, aber \zB auch \citealt{Bremerichvos2016} und \citealt{FuhrhopTeuber2016}).
\citet[16]{Schindler2016} benennt neben der Funktion im Sinne einer Zulassungsschranke für das Studium vor allem zwei Funktionen solcher Tests, nämlich erstens die Diagnostik für das studierende Individuum, anhand derer gezielt Defizite erkannt und rechtzeitig ausgeglichen werden können.
Zweitens führt die Autorin an, dass Eingangstests Bewusstsein und Interesse für die Inhalte des Fachs bei Studierenden wecken können.
In \citet[226]{SchaeferSayatz2017a} argumentieren wir, dass eine weitere Funktion der Tests die Optimierung der Lehre im Studium im Sinne einer Ausrichtung auf die Bedürfnisse der Studierenden sein sollte.
Hier werden jetzt die wesentlichen Ergebnisse des Experiments berichtet, vor um bei Studierenden das Bewusstsein für die Inhalte des Fachs und die Verbindung zwischen Schulgrammatik und linguistischem Fachwissens zu wecken (Schindlers zweite Funktion), und um Lehrende anzuregen, ähnliche Evaluationen vorzunehmen und ihre Lehrinhalte und Lehrmethoden entsprechend an die Ergebnisse anzupassen (die von \citealt{SchaeferSayatz2017a} benannte Funktion).

Wir haben 220 Studierende der Freien Universität Berlin, die überwiegend in Berlin und Brandenburg zur Schule gegangen sind, einen freiwilligen anonymen Test vorgelegt, der aus gezielt ausgewählten Aufgaben aus Lehrwerken bestand, die in diesen Bundesländern für die Jahrgangsstufen sechs bis zehn (in einem Fall für die Grundschule) zugelassen sind.
Es handelte sich also um Aufgaben, die die Studierenden so oder ähnlich wahrscheinlich in ihrer Schulzeit gelöst haben.
Insgesamt wurden zehn Aufgaben ausgewählt und ausgewertet, die verschiedene Schwierigkeitsgrade und verschiedene Themen der Grammatik abdeckten, die auf der Seite der systemexternen Funktion ebenfalls verschiedenen Bereichen zuzuordnen sind.
Die Teilnehmenden, die aus allen Semestern des BA-Studiums kamen, studierten zu 37,7\% Grundschullehramt (83 Personen), zu 37,3\% Deutsch für das Lehramt (82 Personen), zu 23,2\% Deutsch als Fachwissenschaft (51 Personen) und zu 1,8\% andere Fächer (4 Personen).
Zunächst wurde überprüft, ob wirklich durchweg ungenügende "`Leistung"' erbracht wurde.
Abbildung~\ref{fig:grammatikkentnissevonstudierenden001} zeigt, dass dies nicht der Fall ist.

\begin{figure}[htpb]
  \centering
  \resizebox{0.9\textwidth}{!}{\includegraphics{figures/notenspiegel}}
  \caption{Verteilung der auf akademische Noten umgerechneten Leistungen im Experiment aus \citet{SchaeferSayatz2017a}}
  \label{fig:grammatikkentnissevonstudierenden001}
\end{figure}

Nur 15\% der Teilnehmenden wären durchgefallen, wenn es sich um eine Prüfung gehandelt hätte.
Die restlichen Leistungen verteilen sich nach der Umrechnung auf akademische Noten nahezu gleichmäßig zwischen 1,3 und 4,0.
Solch ein Bild ist vollständig erwartbar, wenn davon ausgegangen wird, dass nicht alle Teilnehmenden dasselbe gelernt haben und die Lernzeitpunkte teilweise sieben Jahre zurückliegen.
Es ist davon auszugehen, dass für jedes andere Schulfach ähnliche Ergebnisse erzielt würden, denn immerhin hatten die Teilnehmenden keine Gelegenheit, sich gezielt auf die gestellten Fragen vorzubereiten.

Allerdings haben, wie oben erwähnt, Studierende aller Bachelor-Semester teilgenommen, und es sollte deshalb eigentlich nicht wundernehmen, wenn das Ergebnis besser als das des bayrischen Eingangstests wäre, bei dem schließlich nur Studierende des ersten Semesters befragt wurden, die noch keine spezifische Ausbildung in germanistischer Linguistik erhalten haben.
Abbildung~\ref{fig:grammatikkentnissevonstudierenden002} schlüsselt die Ergebnisse in Prozent nach Studienjahr auf.

\begin{figure}[htpb]
  \centering
  \resizebox{0.9\textwidth}{!}{\includegraphics{figures/semester}}
  \caption{Verteilung der erreichten Prozent im Experiment aus \citet{SchaeferSayatz2017a}, gegliedert nach Studienjahr in einem sogenannten Beanplot; jede kurze waagerechte Linie entspricht einem individuellen Ergebnis, die Hüllen entsprechen einer Dichteschätzung der empirischen Verteilung pro Gruppe, die dicken längeren waagerechten Linien markieren den Mittelwert pro Gruppe}
  \label{fig:grammatikkentnissevonstudierenden002}
\end{figure}

Wie man leicht sieht, werden die Ergebnisse im verlauf des Studiums im Mittel nicht besser.
Es gibt eine leichte Verschiebung der Verteilung (des Bauchs der jeweiligen Beanplots) nach oben, insbesondere im dritten Studienjahr.
Allerdings wäre schon aufgrund des Ausscheidens leistungs- und motivationsschwächerer Studierender (Fachwechsel oder Studienabbruch) im Laufe der Semester eine stärkere Verbesserung der Leistung zu erwarten.
Wie wir in \citet[242--243]{SchaeferSayatz2017a} argumentieren, kann der Grund der beobachteten Stagnation nicht lokal in der Freien Universität gesucht werden, da sich im Vergleich der Kurrikula mehrerer Universitäten aus verschiedenen Bundesländern zeigt, dass die Freie Universität ein sehr typisches Profil in ihren germanistischen Studiengängen anbietet.

Wir gehen davon aus, dass das Problem vielmehr eine meist nicht optimale Kopplung der universitären Inhalte an die Berufsziele der Studierenden ist.
Wie oben deutlich wurde, sind ca.\ 75\% der Studierenden an der Freien Universität angehende Lehrpersonen, und die Lehre müsste sich deutlich an diesem Berufsziel orientieren.
Von einer anderen Seite nähern sich auch \citet{TopalovicDuenschede2014} demselben Problem.
In einer Bundesweiten Umfrage mit 1.019 Lehrpersonen aus allen Schulformen fanden Sie im Jahr 2013 heraus, dass 48\% der Lehrpersonen sich durch ihre Ausbildung nicht hinreichend auf den Grammatikunterricht vorbereitet fühlen \citep[76--77]{TopalovicDuenschede2014}.


\subsection{Studentische Sichtweisen auf Studium und Schulunterricht}
\label{sec:studentischesichtweisenaufstudiumundschulunterricht}

\begin{figure}[htpb]
  \centering
  \resizebox{0.9\textwidth}{!}{\includegraphics{figures/funktionengrammatikunterricht}}
  \caption{}
  \label{fig:studentischesichtweisenaufstudiumundschulunterricht001}
\end{figure}





\section{Grammatik im Lehramtsstudium}
\label{sec:grammatikimlehramtsstudium}








% \section{Ziele dieses Kapitels}
% 
% Dieses Kapitel ist insofern ungewöhnlich, als es in der dritten Auflage des Buches komplett neu hinzugekommen ist, und als es in ihm zumindest vordergründig nicht um Sprache (und Personen, die Sprache benutzen) geht, sondern um Menschen, die mit Sprache reflektierend umgehen, nämlich Lehr- und Lernpersonen in Schulen und Universitäten.
% Nachdem die wesentlichen Grundlagen der wissenschaftliche Grammatikforschung in den vorangegangenen Kapiteln dargelegt wurden, folgt also scheinbar eine Digression.
% Während das Kapitel ohne jegliche Folgen für das weitere Verständnis übersprungen werden kann, ist es gerade für Lehrpersonen im Fach Deutsch an Schulen wichtig, über den Stellenwert einer gründlichen grammatischen Beschreibung im Deutschunterricht nachzudenken.
% Der mangelnde \textit{Anwendungsbezug} wird schließlich immer wieder in Evaluationen von Lehrveranstaltungen zur Linguistik insbesondere von angehenden Lehrpersonen bemängelt.
% Wir diskutieren hier, worin der Anwendungsbezug der Grammatik an sich für die schulische Lehre besteht, und warum es nicht zielführend ist, bei jedem linguistischen Detail, das im Studium besprochen wird, die Rundumschlagsfrage zu stellen: "`Wofür brauche ich das denn später im Lehramt?"'
% 
% Es geht damit mittelbar auch darum, den Zweck des gesamten Buches und den Nutzen einer linguistischen bzw.\ grammatischen Ausbildung für einen Großteil des Publikums dieses Buches (nämlich angehende Lehrpersonen) zu motivieren.
% Ich argumentiere dafür, dass das, was hier auf vielen hundert Seiten dargelegt wird, einen sehr genau definierten und wichtigen Platz in der Ausbildung von Lehrpersonen für das Fach Deutsch hat.
% Dies setzt natürlich auch voraus, dass die Ziele des Schulfachs Deutsch, zumindest was die Grammatik angeht, angegeben werden.
% 
% Das Kapitel ist wie folgt aufgebaut.
% Zunächst wird in Abschnitt~\ref{sec:formundfunktionindergrammatik} genauer über die Beziehung von Form und Funktion zu sprechen, weil diese ein Hauptthema in der Diskussion um Grammatik in Schule und Lehramtsstudium ist.
% Ich gehe in diesem Abschnitt dabei sowohl von einer bestimmten Sorte der Kritik an diesem Buch als auch von spezifischen Problemen, die in der Lehramtsausbildung zu beobachten sind, aus.
% In Abschnitt~\ref{sec:kenntnisseerwartungenundmotivationvonstudierenden} werden dann empirische Daten präsentiert, die illustrieren, welchen Wissensstand Studierende in Schulgrammatik haben, und welche Funktionen des schulischen Grammatikunterrichts für Studierende wichtig sind.
% In Abschnitt~\ref{sec:grammatikinderschule} werden kompakt Konzepte des schulischen Grammatikunterrichts diskutiert, die zwischen Fachdidaktik und Fachwissenschaft diskutiert werden.
% In Abschnitt~\ref{sec:grammatikimlehramtsstudium} wird schließlich ein Konzept der Grammatikausbildung im Lehramtsstudium skizziert.
% Stilistisch bewegen wir uns hier übrigens etwas ungewöhnlich zwischen einer Diskussion, die die germanistische Linguistik und die germanistische Fachdidaktik schon länger um die Sache führen, und einer Vermittlung von Inhalten und Anregungen für Studierende des Fachs.
