\chapter{Grammatik im Lehramtsstudium}
\label{sec:grammatikimlehramtsstudium}


\section{Grammatik in der Schule}
\label{sec:grammatikinderschule}

<+Allgemeine Einleitung+>

Die Frage nach der Rolle von Grammatik im Lehramtsstudium kann nicht losgelöst von der Rolle der Grammatik im Schulunterricht diskutiert werden.
In diesem Abschnitt wird daher bündig ein Konzept von Grammatik in der Schule diskutiert.
In Abschnitt~\ref{sec:grammatikimlehramtsstudium} werden die Konsequenzen für das Lehramtsstudium gezogen.
Abschnitt~\ref{sec:formundfunktionindergrammatik} verortet dieses Buch auf Basis des vorher Gesagten im Kontext des Lehramtsstudiums.
Dieses Kapitel kann ohne Probleme für die weitere Lektüre übersprungen werden.

\subsection{Bildungssprache und ihr Erwerb}
\label{sec:bildungsspracheundihrerwerb}

Wenn Kinder eingeschult werden, können sie bereits auf einem hohen Niveau sprechen, vollständige Sätze formulieren und diese erfolgreich in alltäglichen Kommunikationssituationen verwenden.%
\footnote{Wir beziehen uns hier der Einfachheit halber nur auf Kinder mit deutscher Erstsprache und ohne Lern- oder Kongitionsschwächen.
Insbesondere der steigende Anteil von Kindern mit mehrsprachigem und auch nicht-erstsprachlichem Hintergrund sollte aber in diesem Kontext insgesamt berücksichtigt werden.
Die zitierte und weiterführende Literatur enthält dazu viel Wichtiges.}
Es kann also nicht die Aufgabe des Deutschunterrichts sein, Kindern elementare orale Kommunikationsfähigkeiten für Alltagssituationen zu vermitteln.
Hingegen ist es eine offensichtliche Aufgabe des Deutschunterrichts, die Schrift und die Schreibungen des Deutschen zu lehren.
Ein oft anzutreffender Irrglaube ist dabei, dass ungefähr mit der Primarstufe der Schrift- und Schreibungserwerb abgeschlossen sei und es in diesem Bereich nur noch um das "`Ausmerzen"' verbleibender Schwächen gehe.
Es ist vielmehr davon auszugehen, dass der Orthographieerwerb sogar nach der Schulzeit (natürlich in Form eines sich sättigenden -- also stetig langsamer werdenden -- Prozesses) fortgeführt wird \citep[72]{Portmanntselikas2011}.
Der Erwerb der Schriftsprache muss aber vielmehr in einem größeren Rahmen betrachtet werden, nämlich dem des Erwerbs der \textit{Bildungssprache}.
Dieser Abschnitt widmet sich daher dem Bildungsspracherwerb.

Die Sprache, die vorliterate Kinder (die wir hier vereinfachend mit noch nicht eingeschulten Kindern gleichsetzen) sprechen, ist an bestimmte Situationen und Funktionen gebunden (Essen, Spielen, Aufräumen, Zeigen von verschiedenen Emotionen zur Stimulation einer Reaktion beim Gegenüber usw.) und wird weitgehend ohne systematische Instruktionen in diesen Situationen erlernt.
Mit der Einschulung und damit der Literalisierung verlangen wir von Kindern zunehmend auch die Fähigkeit, völlig andere Modi der Kommunikation zu bedienen.
Es wird erwartet, dass Kinder zunehmend in der Lage sind, sowohl schriftlich als auch mündlich komplexe Sachverhalte darzustellen, Begriffe intensional (also nicht durch bloße Aufzählung von Beispielen) zu definieren, für oder gegen eine Sache zu argumentieren, in verschiedenen Situationen teilweise stark unterschiedliche angemessene sprachliche Mittel zu verwenden, die Standardsprache zu beherrschen und vieles dergleichen mehr.
Diese Fähigkeiten charakterisieren die \textit{Bildungssprache} (\zB \citealt{Feilke2012}).
Merkmale der Bildungssprachliche finden sich nun vor allem in Kontexten, denen vorliterate Kinder nicht ausgesetzt sind, \zB in schulischen Lernsituationen (wohlgemerkt in allen Fächern, nicht nur im Deutschunterricht) und im akademischen Bereich.
Darüberhinaus werden auch in Alltagssituationen von Erwachsenen ggf.\ ähnlich fortgeschrittene sprachliche Leistungen erwartet, zum Beispiel bei der Auseinandersetzung mit juristischen oder finanziellen Fragen oder der Lösung technischer Probleme \citep[5]{Feilke2012}.
Damit ist die Beherrschung der Bildungssprache "`gleichermaßen ein Bildungskapital, wie sie eine Hürde für das Verstehen sein kann"' \citep[11]{Feilke2012}.
Werden bildungssprachliche Kompetenzen nämlich nur unvollständig oder mangelhaft erworben, ist eine Teilhabe in vielen Bereichen der Gesellschaft nur eingeschränkt möglich.
Zugleich ist die bildungssprachliche Ausbildung auch für die schulische Karriere jenseits des Deutschunterrichts von besonderer Wichtigkeit, da die bildungssprachlichen Anforderungen in allen Fächern mit der Komplexität der Inhalte steigen.
Dementsprechend verlangen die Richtlinien der Kultusministerkonferenz auch gerne global, dass Lernende nach der Schulzeit in der Lage sein mögen, situationsangemessen und standardkonform zu sprechen und schreiben \citep[6]{Eisenberg2004}.

Dabei ist es nun richtig, dass sich die Schwerpunkte im Laufe der Erwerbskarriere von der Beherrschung basaler Orthographie mehr und mehr hin zu Fragen des Stils und der situationsangemessenen Schreibweise und Sprechweise verschieben \citep[77]{Portmanntselikas2011}, aber beide sind von Anfang bis Ende auf besondere Weise miteinander verknüpft.
Ein Großteil der Fachdidaktik des Deutschen und der didaktisch interessierten Linguistik des Deutschen versteht Standardsprache und Bildungssprache als untrennbar gekoppelt an die historische Entwicklung der Schrifttradition des Deutschen \textbf{(Ulrike: gute Quelle hierzu?)}.
Auf der kognitiven bzw.\ individuellen Seite werden bildungs- und standardsprachliche Kompetenzen bedingt und geformt durch den Erwerb der Schriftsprache.
In diese Richtung argumentieren zum Beispiel \citet[4,12,14,15]{Eisenberg2004}, \citet[71,78]{Portmanntselikas2011}, \citet[6]{Feilke2012}.
Für eine ausführlichere Betrachtung der Position von \citet{Bredel2013} siehe Abschnitt~\ref{sec:sprachbetrachtungundkonzeptederdeutschdidaktik}.

Der Schrift kommt eine besondere Rolle zu, weil sie bestimmte Charakteristika der Bildungssprache forciert.
Oben wurde betont, dass vorliterate Kinder Sprache vor allem situationsgebunden verwenden, und die erwähnten später erworbenen oder zu erwerbenden sprachlichen Kompetenzen wie Erläutern, Definieren, Argumentieren sind alle verbunden mit einer Loslösung der Sprache von der gegebenen Sprechsituation.
Es geht nicht mehr darum, in der aktuellen Situation mit unmittelbarer Wirkung sprachlich zu handeln, sondern zu größeren Zusammenhängen zu abstrahieren und allgemeine sowie komplexere Sachverhalte aufzugreifen und zueinander in Beziehung zu setzen.
Gleichzeitig geht mit dieser Distanzierung eine Distanzierung von der eigenen sprechenden Person einher, weil typischerweise Verstehens- und Denkprozesse der angesprochenen Personen in der Wahl der eigenen sprachlichen Mittel berücksichtigt werden.
Wir wollen verstanden werden und passen unsere Sprache daher den Zuhörenden und Lesenden an.
Dies alles wird mit dem Begriff der \textit{Dekontextualisierung} der Sprache erfasst.\index{Dekontextualisierung}%
\footnote{Eine einfache Form der Dekontextualisierung ist die berichtende Erzählung, zum Beispiel von Erlebtem.
Dabei werden Geschehnisse aus einem vergangenen -- also nicht dem gegenwärtigen -- Kontext wiedergegeben.
Freilich sind vorliterate Kinder dazu in der Lage.
Die Art, in der sie über erlebte Ereignisse berichten, unterscheidet sich aber erheblich von einem bildungssprachlich strukturierten Bericht, was die Gewichtung und Serialisierung der Teile der Erzählung angeht.
Wir alle kennen wahrscheinlich Erzählungen vorliterater Kinder, die aus einer Reihung von "`und dann \ldots"'-Konstruktionen bestehen, und denen zu folgen oft schwer ist, weil Ereignisse nicht nach ihrer Relevanz gewichtet wiedergegeben werden, und die damit insgesamt unter einem Mangel von Kohärenz leidet.
\citet{Bredel2013} enthält eine relevante Einführung in diesen Phänomenbereich jenseits solcher persönlicher episodischer Erfahrungen.}

Wichtig ist nun, dass die Schrift als Medium die Dekontextualisierung forciert.
Ein Schriftstück entsteht prototypischerweise ohne direkten Äußerungskontext und soll ohne einen solchen wirken.
Selbst dialogische Online-Kommunikation (vor allem Textnachrichten) muss ohne räumliche Nähe auskommen, und die schreibende Person muss für eine gelungene Kommunikation dabei in der Lage sein, zu verstehen und in ihren sprachlichen Äußerungen zu berücksichtigen, dass die angeschriebene Person nicht dasselbe sieht und hört wie sie selber.
Schriftsprache führt also automatisch zu einer Dekontextualisierung und fördert die Abstraktion des Worts.
Neben dieser engen Verbindung von wesentlichen Merkmalen der Bildungssprache und Schrift begünstigt Schrift die Herausbildung und die Reflexion über die Konformität zu einer wie auch immer gearteten Norm oder einem Standard und damit notwendigerweise eine Fokussierung auf formale Eigenschaften von Sprache \citep[41]{Bredel2013}.%
\footnote{Zur expliziten Reflexion über Sprache siehe auch Abschnitt~\ref{sec:sprachbetrachtungundkonzeptederdeutschdidaktik}.}
Einerseits nimmt es nicht wunder, dass die Herausbildung von Standards für ein bestimmtes Deutsch als allgemeine Verkehrssprache durch die weite Verbreitung von Schriftgut durch den Buchdruck und damit einen starken Zuwachs an literalisierten Personen Vorschub erhielt \textbf{(Ulrike: Quelle?)}.
Durch die Fixierung von Sprache in Form eines Schriftstücks und damit die Möglichkeit der Revision dieser Sprache sowie einer Reflexion über sie, die nicht durch die eigene Erinnerungsfähigkeit begrenzt wird, werden Überlegungen über angemessene und \textit{richtige} Sprache gefördert.
Auch in der individuellen Entwicklung wird daher die Frage nach der Richtigkeit von Sprache und der Norm vom geschriebenen Wort ausgehend auf orale Kommunikation erweitert (\zB \citealt[72]{Portmanntselikas2011}).
Immerhin sind die Gelegenheiten, zu denen es wirklich auf normnahe Sprache ankommt, in der oralen Kommunikation eher selten (\zB politische Ansprachen, Bewerbungsvorträge für Germanistikprofessuren), in der schriftlichen Kommunikation aber nahezu der Normalfall.

Die spezifischen \textit{Funktionen} der Schrift- und Bildungssprache gehen nun mit einem erweiterten und komplexeren Inventar von \textit{formalen Mitteln} bzw.\ mit einer anspruchsvolleren Verwendung der existierenden formalen Mittel einher.\index{Form und Funktion}
\citet[8--9]{Feilke2012} gibt eine beispielhafte Liste von anspruchsvollen formalen Mitteln, die für Bildungssprache charakteristisch sind, und die er bestimmten funktionalen Bereichen zuordnet.
Zum Bereich des \textit{Explizierens}, also dem Verständlichmachen von komplexen Sachverhalten zählt er komplexe Adverbiale, Attribute und explizite Satzverbindung (\zB kausal oder adversativ).
Für den Bereich des \textit{Diskutierens}, also der Abwägung von Sachverhalten bezüglich ihres Wahrheitsgehalts, listet er beispielhaft den Gebrauch von Modalverben, Konjunktionen und konzessiven Konstruktionen.%
\footnote{Die beiden anderen von \citet[8--9]{Feilke2012} diskutierten Bereiche sind die des \textit{Verdichtens} und \textit{Verallgemeinerns}.}
Diese Mittel werden in der situationsgebundenen Alltagssprache erheblich seltener gebraucht (siehe auch \citealt[82--83]{Portmanntselikas2011}) und mit Beginn der Literalisierung erst -- zusammen mit den zugehörigen anderen kognitiven Fähigkeiten, die auf der inhaltlichen Seite erforderlich sind -- langsam ausgebildet.
Es ist also ein Irrtum, zu glauben, dass Kinder in der Schule einfach nur eine ihnen bereits bekannte Sprache zu schreiben lernen.
Vielmehr wird ihre gesamte Grammatik umgebaut, erweitert, um neue Register bereichert und in ihrem Stil verändert \citep[4,12--15]{Eisenberg2004}.
Zudem findet eine Ausrichtung auf die Norm und ein Bewusstmachen von Sprache statt, das vorher nicht existiert (siehe Abschnitt~\ref{sec:sprachbetrachtungundkonzeptederdeutschdidaktik}).

Damit ist also die Aufgabe des Deutschunterrichts \textit{für die gesamte Schulzeit} und nicht etwa nur für die Primarstufe definiert.
Die Aufgabe könnte nicht größer und schwieriger und verantwortungsvoller sein.
\citet[7]{Eisenberg2013c} spricht davon, dass den Lehrenden die Sprache der Lernenden anvertraut werde.
Der Umbau der Sprache von Schulkindern betrifft, wie oben argumentiert wurde, die sprachlichen Formen und Funktionen.
Da die Grammatik die Disziplin ist, die sich mit den sprachlichen Formen beschäftigt, ist die Frage nach dem Zweck von grammatischem Wissen für angehende Lehrpersonen im Fach Deutsch eigentlich bereits beantwortet.
Es wird aber gerne die Frage gestellt, welches Grammatikwissen denn das richtige sei, und ob die Grammatik nicht der Form an sich einen zu hohen Stellenwert beimesse.
In Abschnitt~\ref{sec:sprachbetrachtungundkonzeptederdeutschdidaktik} werden daher kurz didaktische Konzepte für den Deutschunterricht und der Systembegriff der Grammatik diskutiert.
In Abschnitt~\ref{sec:anforderungenanlehrpersonen} werden dann kompakt die Anforderungen formuliert, mit denen sich Lehrpersonen nahezu zwangsläufig konfrontiert sehen.


\subsection{Sprachbetrachtung und Konzepte der Deutschdidaktik}
\label{sec:sprachbetrachtungundkonzeptederdeutschdidaktik}
% inkl. System

Eins der Ziele dieses Kapitels ist es, für die Existenzberechtigung der Grammatik in der Schule und dem Germanistikstudium zu argumentieren.
Es mag auf Verwunderung stoßen, dass sich die Grammatik überhaupt zu rechtfertigen versucht.
Dies hat damit zu tun, wie sich das Verhältnis der Linguistik des Deutschen, der Fachdidaktik Deutsch und dem Lehrpersonal des Schulfachs Deutsch entwickelt hat.
Eine umfassende Betrachtung der Geschichte der Deutschdidaktik gehört nicht hierher, aber die wesentlichen Linien, die zur heutigen Situation geführt haben, sind recht einfach.%
\footnote{Ich folge wie im Rest dieses Abschnitts der ausführlichen Einführung aus \citet{Bredel2013} (hier besonders Abschnitt~3.1), verwende aber meine eigenen Formulierungen und lasse gezielt Dinge aus, die im gegebenen Kontext nicht relevant erscheinen.
Bei Bredel finden sich auch zahlreiche einschlägigen Referenzen.}

Der {traditionelle Grammatikunterricht} der ersten Hälfte des neunzehnten Jahrhunderts ist ein rein an der Form orientierter deklarativer Benennungsunterricht.
Aus der Lateingrammatik entlehnte Kategorien werden auf das Deutsche übertragen, und die Leistung der Lernenden besteht darin, Wortarten, Nebensatztypen, grammatische Funktionen usw.\ in Sätzen zu identifizieren und zu benennen.
Dass dieser Ansatz beim Erlernen der Bildungssprache nicht sonderlich hilfreich sein dürfte, erscheint heute offensichtlich.%
\footnote{Umso erstaunlicher ist es, dass immer noch traditionell Grammatik unterrichtet wird, worauf im weiteren noch eingegangen wird.}
Mit den späten sechziger Jahren des neunzehnten Jahrhunderts kommt allerdings im Rahmen eines generellen Zweifels an traditionellen Lehrmethoden und im Sinne einer Sprachkritik Bewegung in die Angelegenheit, und es setzte eine konstante Diskussion in mehreren Wellen (bzw.\ \textit{Wenden}) ein.

Nach dem gescheiterten Versuch der Linguistik, in den siebziger Jahren die sogennante \textit{generative Transformationsgrammatik} (ein abstrakt-algebraisches und nach zehn Jahren bereits überholtes theoretisches Syntaxmodell) für den Schulunterricht nutzbar zu machen, kam es zu einer anti-formalen und anti-systematischen \textit{kommunikativen Wende}.
Im \textit{situativen Sprachunterricht} der späten siebziger Jahre \citep[229--232]{Bredel2013} wurden grammatische Eigenschaften von sprachlichen Äußerungen nur als Teil der situativ-kommunikativen Funktion von Sprache betrachtet.
Die Vermittlung sollte entsprechend rein situativ, nicht abstrakt-systematisch erfolgen.
Das Hauptproblem dieses Ansatzes ist die Unmöglichkeit, durch rein situative Vermittlung (allein schon durch die Begrenztheit der zeitlichen Ressourcen in der Schule) ein echtes systematisches Wissen aufzubauen (so auch im Wesentlichen Bredels Kritik).

Der \textit{funktionale Grammatikunterricht} der achtziger Jahre und darüberhinaus \citep[233--239]{Bredel2013} strebt durchaus eine systematische Betrachtung an.
Es wird auf Erkenntnisprozesse bei Lernenden gesetzt, die erkennen sollen Funktionen von sprachlichen Äußerungen erkennen, lernen, sie kritisch zu hinterfragen, produktiv mit ihnen umzugehen usw.
Der Fokus liegt hier allerdings ausschließlich auf systemexternen Funktionen von Sprache (siehe Abschnitt~\ref{sec:formundfunktionindergrammatik}), und die interne Systematik der Grammatik bleibt außen vor.
Während also durchaus systematisch und teilweise induktiv vorgegangen wird, wird nicht im eigentlichen Sinn Grammatik (also eine Formbetrachtung) angestrebt.

Für das hier vorgelegte Werk ist die \textit{Grammatik-Werkstatt} der neunziger (und späterer) Jahre (\citealt{EisenbergMenzel1995}; \citealt[Abschnitt~3.2.4]{Bredel2013}) mit Abstand der einflussreichste Vorschlag einer Methode für den Deutschunterricht.\index{Grammatik-Werkstatt}
Sie setzt auf systematische Grammatik, aber entgegen dem traditionellen Grammatikunterricht basiert sie auf induktiven Erkenntnisprozessen auf Basis gezielt ausgewählten sprachlichen Materials.
Durch ein gesteuertes Arbeiten (nahezu ein Spiel) mit Formen und Formenreihen erfassen die Lernenden Muster und Regularitäten.
Sind diese Muster und Regularitäten erkannt, können sie zu Funktionen in Beziehung gesetzt werden.
Das Verfahren ist induktiv, setzt also auf Erkenntnisse auf Basis der Betrachtung von sprachlichem Material, nicht etwa auf die Vermittlung eines gegebenen oder gar theoretischen Wissens ex cathedra.
Durch diese Methode sollen die Ergebnisse besser im Gedächtnis verankert werden.
Zudem soll die Fähigkeit, auch über den Unterrichtsstoff hinaus grammatisch zu arbeiten, erworben werden.
Vor allem ist zu bemerken, dass die in der Grammatik-Werkstatt vertretene Methode der grundlegenden Methode der traditionellen Linguistik entspricht \citep[239]{Bredel2013}.
Natürlich setzt das Spiel mit Formen und Formenreihen eine bestehende Intuition über Sprache bei den Lernenden voraus und ist damit vor allem an die Erstsprache gebunden.
Die vorhandene Intuition soll aber weiter geschult und systematisiert werden \citep[241]{Bredel2013}.
Auf diesen Ansatz kommen wir später noch zurück.

Bei alledem ist aber zunächst ernüchternd festzustellen, dass neue Methoden den Schulunterricht gar nicht oder nur bruchstückhaft erreichen \citep[243]{Bredel2013}.%
\footnote{Erst kürzlich wurde mir gegenüber in persönliche Kommunikation aus linguistischen Kreisen vehement argumentiert, dass insbesondere die Linguistik viel zur Weiterentwicklung der realen schulischen Deutschdidaktik beigetragen hätte.
Insbesondere die Lehramtsinitiative der Deutschen Gesellschaft für Sprachwissenschaft wurde in diesem Zusammenhang erwähnt.
Es muss betont werden, dass die Lehramtsinitiative eine der wenigen langjährigen Initiativen sind, die den Dialog zwischen der germanistischen Linguistik und Lehrpersonen aufrechtzuerhalten versucht.
Allerdings wird in der Literatur zur Didaktik des Deutschen wieder betont, dass die Schulwirklichkeit im Prinzip auf einem Stand des traditionellen Grammatikunterrichts stehengeblieben ist (\zB \citealt[211]{Steets2003} und dort zitiert \citealt[143]{SteinigHuneke2002}; \citealt[243,257]{Bredel2013}; \citealt[2]{KoepckeZiegler2013}).}
Offiziell geben die Richtlinien des Bundes und der Länder -- auch in den neueren "`kompetenzorientierten"' Formulierungen -- wenig Konkretes vor \citep[250--255]{Bredel2013}.
Statt aufwändig entwickelter didaktischer Ansätze in der Grammatikdidaktik hat vor allem hat die \textit{Terminiliste} der Klutusministerkonferenz Berühmtheit erlangt \citep[244--249]{Bredel2013}.%
\footnote{In dieser Liste werden zahlreiche (und in ihrer Auswahl umstrittene) grammatische Begriffe, die Lernenden nach ihrer schulischen Laufbahn bekannt sein sollen, schlicht und einfach aufgelistet.}
Diese liefert allerdings keine Definitionen und keine Methoden der Vermittlung, ist also eher ein Mittel, das in der Praxis den traditionellen Grammatikunterricht fördert.
Zudem kennen selbst Lehrpersonen diese Liste nur zu einem geringen Prozentsatz \citep{Haecker2009}.

Die letzten fünfzig Jahre der Ideen für den schulischen Grammatikunterricht waren also von einer Auseinandersetzung verschiedenen Fragen geprägt.
Soll der Grammatikunterricht formal oder funktional ausgerichtet sein?
Soll er induktiv vorgehen (also Erkenntnisprozesse auf Basis von sprachlichen Daten fördern) oder ein feststehendes Wissen vermitteln?
Mit alldem verbunden ist auch die Frage, ob es überhaupt einen expliziten Grammatikunterricht geben muss.
Es wird nun argumentiert, dass die Fähigkeit zur \textit{Sprachbetrachtung} wesentlich für bildungssprachliche Kommunikation und Teil unserer Sprachkultur ist.\index{Sprachbetrachtung}
Allen Konzepten des schulischen Grammatikunterrichts, die wir oben in diesem Abschnitt besprochen haben, ist gemein, dass sie die Lernenden in die Lage versetzen sollen, über Sprache mehr oder weniger explizit nachzudenken, um dadurch gezielt Entscheidungen über ihren eigenen Sprachgebrauch fällen zu können.
Dies steht wiederum im Dienst der Beherrschung der Schrift- und Bildungssprache (s.\ Abschnitt~\ref{sec:bildungsspracheundihrerwerb}).
\citet[14]{Bredel2013} stellt richtig fest, dass zum bloßen Beherrschen einer Sprache keine Reflexion über diese Sprache selber erforderlich ist.
Sprachbetrachtung ist eine alltägliche Tätigkeit, \zB wenn wir nach den richtigen Wörtern suchen, wenn wir Fragen der Orthographie und Interpunktion nachgehen, wenn wir die Ausdrucksweise von einem Gegenüber im Diskurs bewerten \citep[22]{Bredel2013}.
Das Gleiche gilt, wenn wir \zB Formulierungen in Texten optimieren oder Schreibvarianten ausprobieren \citet[23]{Bredel2013}.
In Anlehnung an \citet{Clark1978} spricht \citet[35]{Bredel2013} auch von Sprachbetrachtungsaktivitäten wie dem sprachlichen Einstellen auf zuhörende Personen, dem Kommentieren und Korrigieren von Äußerungen anderer, der expliziten Definition sprachlicher Begriffe, der Erklärung von Bedeutungen und Grammatikalität.
Diese Dinge sind in der Tat sprachlicher Alltag und eine Folge dessen, dass wir unsere Kultur stark am geschriebenen Wort -- und der vom geschriebenen Wort ausgehenden Bildungssprache -- ausgerichtet haben.%
\footnote{Die Sprachwissenschaft von Sprachen wie dem Deutschen ist im übrigen selber schriftorientiert.
Anders als manche in der Linguistik, die das \textit{Primat der gesprochenen Sprache} beschwören (dazu mehr in Abschnitt~\ref{sec:graphematikalsteildergrammatik}), attestiert \citet[40]{Bredel2013} dies korrekterweise.}

Zur Bildungssprache gehört also die Fähigkeit zu expliziten Sprachbetrachtung.
Inwiefern durch sprachbetrachtenden Schulunterricht allerdings direkt das sprachliche Handeln von Lernenden beeinflusst werden kann, ist nicht geklärt (\citealt[73,75--79,80]{Portmanntselikas2011}; \citealt[94]{Bredel2013}; \citealt[8]{Eisenberg2013c}; \citealt[2]{KoepckeZiegler2013}).
Wir wissen also nicht, ob es zum Beispiel Kindern in der Primarstufe hilft, ihre eigenen sprachlichen Fähigkeiten zu verbessern, wenn wir sie dem Grammatikunterricht aussetzen.
Die Grammatik in der Schule aufzugeben, wird aber kaum jemand wagen, denn insbesondere der Erwerb der Orthographie und Interpunktion erfordert das Explizieren von grammatischen Sachverhalten \citep[9--10]{Eisenberg2013c}.
Natürlich könnte man sich darauf verlassen, dass nur Kinder, die freiwillig viel Lesen, die Orthographie und Interpunktion einwandfrei auch ohne großartige Instruktion lernen, aber niemand würde auch nur im Ansatz wagen, Kindern deswegen einfach nur die Buchstaben zu lehren und sie dann ihrem (hoffentlich lesenden) Schicksal zu überlassen.
Ganz unabhängig von der nur unvollständig erforschten Wirkung des frühen Grammatikunterrichts auf den Bildungsspracherwerb ist Grammatikunterricht insgesamt eben auch eine langfristige Ausbildung zur Sprachbetrachtung, und diese ist Bestandteil unseres sprachlichen Alltags.

Im nächsten Abschnitt wird bezüglich der vorgeschlagenen Konzepte für den Grammatikunterricht Stellung bezogen.


\subsection{System und Funktion im Grammatikunterricht}
\label{sec:systemundfunktionimgrammatikunterricht}

Wir haben gezeigt, dass im Sinn der Förderung des Schrift- und Bildungsspracherwerbs schulischer Deutschunterricht zur Sprachbetrachtung anleiten soll.
Die oben beschriebenen didaktischen Modelle des Grammatikunterrichts (traditionell, situativ, funktional, Grammatik-Werkstatt) unterscheiden sich vor allem auf drei Achsen (siehe auch \citealt[8--9]{Menzel2017}):
erstens \textit{deduktiv vs.\ induktiv}, zweitens \textit{systematisch vs.\ nicht-systematisch}, drittens \textit{funktional vs.\ nicht-funktional}.%
\footnote{Hier wird bewusst nicht \textit{funktional} gegen \textit{formal} gestellt.}

Der traditionelle Unterricht ist deduktiv, das heißt er vermittelt ein einmal festgelegtes Wissen, und sprachliches Material wird zu diesem Wissen in Beziehung gesetzt, also in den Begriffen und Kategorien der traditionellen Grammatik analysiert.
Alle anderen Ansätze sind ganz oder teilweise induktiv, sie setzen also auf Erkenntnisprozesse bei den Lernenden, die auf gegebenem Material aufbauen.
Es scheint also in der neueren Didaktik ein Konsens darüber zu bestehen, dass Lernenden in der Schule nicht einfach ein deklaratives Wissen vorgesetzt werden soll.
Der traditionelle Grammatikunterricht kann daher als disqualifiziert gelten.

Bezüglich der Unterscheidung von systematischem und nicht-systematischem Unterricht steht die Grammatik-Werkstatt von den verbleibenden Ansätzen alleine auf der Seite des grammatischen Systems.%
\footnote{Das in \citet{Bredel2013} vorgeschlagene Konzept ist nicht identisch zur Grammatik-Werkstatt, setzt aber wie diese auf einen systematischen Zugang und ist ihr generell vergleichsweise ähnlich.}
Der situative Sprachunterricht setzt auf in konkreten Situationen erworbenes Inselwissen, der funktionale geht von der Funktion aus uns vermittelt eher textuelles Wissen als systematische Grammatik.
An dieser Stelle müssen einige Punkte berücksichtigt werden.
Erstens wurde in Kapitel~\ref{sec:grammatik} bereits argumentiert, dass das System der Grammatik ein Eigenleben (jenseits der kommunikativen Funktion sprachlicher Zeichen) hat.
Zweiten zeigt bereits Kapitel~\ref{sec:grundbegriffedergrammatik}, wie komplex dieses System eigentlich ist, und wie trotzdem auf Basis von sprachlichem Material Begriffe wie Rektion und Valenz induktiv erarbeitet werden können.
Drittens wissen wir auch aus Kapitel~\ref{sec:grammatik}, dass die Norm (also zum Beispiel das Standarddeutsche) nur schwer greifbar ist, dass nicht jeder sprachliche Zweifelsfall überhaupt normativ geregelt ist, und dass Sprache (selbst im Standard oder in der Nähe des Standards) starker Variation zwischen Regionen, persönlichen Stilen, situativen Registern usw.\ unterliegt.
Viertens geht Schulunterricht aber nicht ohne Norm, denn Eltern würden zurecht widersprechen, wenn Lehrpersonen ihren Kindern "`rein deskriptive Sprachbetrachtung"' beibrächten (weil das die Linguistik im Lehramstsstudium auch so gemacht hat).
Fünftens muss festgestellt werden, dass die Zeit, die dem Deutschunterricht -- und insbesondere der Sprachbetrachtung und der Grammatik -- in der Schule eingeräumt wird, angesichts der Aufgabe äußerst knapp ist.
Nimmt man diese Punkte zusammen, kommt man um einen Unterricht nicht herum, der das grammatische System als Ganzes im Blick hat.
In der Kürze der Zeit und angesichts der Komplexität des Systems wird ein kleinteiliger rein kommunikativ-funktional ausgerichteter Unterricht die Lernenden nicht in den Stand versetzen, souverän normkonforme bildungssprachliche grammatische Entscheidungen zu treffen.
Sprachbetrachtung hilft, den Standard als eine von vielen Varietäten versteht zu verstehen und von diesen zu unterscheiden.
Gleichzeitig ist sie der Schlüssel zum Verständnis der Variation \textit{innerhalb} des Standards in Form von Stil- und Registerspezifik.
Das funktioniert aber nur, wenn die verschiedenen Varietäten (aller Art) als System begriffen werden, das auf die zugrundeliegenden Möglichkeiten des grammatischen Systems aufbaut (\citealt[10--11]{Eisenberg2004}; siehe auch \citealt[10]{Menzel2017}).
Gegen ausufernden Formenpluralismus (rein deskriptive Grammatik) und Regulierungswut (also den Glauben an die Einheitslösung, die doch irgendwo in einer Normgrammatik stehen muss) stellt \citet[8--9]{Eisenberg2004} die Ausbildung zur Fähigkeit, sprachliche Formen in das System einzuordnen und die Bedeutungs- und Gebrauchsunterschiede im Zweifelsfall selber analysieren zu können.%
\footnote{Auch \citet[80--83]{Portmanntselikas2011} argumentiert, dass ein Sprachunterricht, der eine systematische und nicht bloß inselhaft-episodische Wirkung haben soll, auf begrifflich fassbare Regularitäten zurückgegriffen werden muss.
Grammatisches Wissen, das nur exemplarisch und damit inselhaft bzw.\ nicht systematisch eingeführt wird, reicht nicht aus, weil die Reichweite der entsprechenden Regularität nicht notwendigerweise erkannt wird.}

Mit der Forderung nach einem deduktiven, systematischen und funktionalen Grammatikunterricht im Rahmen der Ausbildung in Schrift- und Bildungssprache beenden wir diese kurzen Überlegungen zum schulischen Spracherwerb.
Bevor Schlussfolgerungen für das Lehramtsstudium in Abschnitt~\ref{sec:grammatikimlehramtsstudium} gezogen werden, gibt Abschnitt~\ref{sec:anforderungenanlehrpersonen} einen kurzen Überblick über die Bandbreite dessen, was Lehrpersonen in der Schule leisten müssen.


\subsection{Anforderungen an Lehrpersonen}
\label{sec:anforderungenanlehrpersonen}

Neben den Aufgaben, die Lehrpersonen aller Jahrgangsstufen zu bewältigen haben, gibt es zunächst die besonderen Anforderungen an Lehrpersonen an Grundschulen.
Hier wird die Grundlage für die Fähigkeit zur expliziten Sprachbetrachtung gelegt und die elementare Schreibfähigkeit vermittelt.


\section{Grammatik im Lehramtsstudium}
\label{sec:grammatikimlehramtsstudium}

\subsection{Aufgaben der Linguistik im Lehramtsstudium}
\label{sec:aufgabenderlinguistikimlehramtsstudium}

\subsection{Grammatikkenntnisse von Studierenden}
\label{sec:grammatikkentnissevonstudierenden}

In \citet{SchaeferSayatz2017a} berichten Ulrike Sayatz und ich ein Experiment (im wissenschaftlichen Sinn des Worts), mit dem wir den letzten Punkt des vorangehenden Abschnitts untersucht haben.
Konkret haben wir evaluiert, welche schulgrammatischen Kenntnisse Studierende am Anfang des Studiums haben.
Es ging uns also um die berüchtigten \textit{schulischen Vorkenntnisse}.

Für unsere Studie gab es verschiedene Vorläufer.
Besonders medienwirksam wurden 2007 die Ergebnisse von einer Art Studieneingangstest berichtet, der in Bayern durchgeführt wurde und laut der journalistischen Aufbereitung \citep{Spiegelonline2007,Szonline2007} ergab, dass Studierende der Germanistik nicht ausreichend durch die Schule auf das Studium vorbereitet würden.
Nach einer Umrechnung der Ergebnisse in Noten wäre über die Hälfte der Teilnehmenden durchgefallen.
Die betroffenen Studierenden dürften sich allerdings durch die Art der Formulierungen, mit denen in der Presse operiert wurde, lediglich gedemütigt und keinesfalls motiviert fühlen.
Spiegel Online spricht von einem "`Grammatik-Fiasko"', bei dem Studierende "`mit Karacho durchgefallen"' seien.
Noch weiter treibt es die Süddeutsche Zeitung Online, die im genannten Artikel ausführt, die "`Germanistik-Studenten"' hätten sich "`blamiert"', und viele von ihnen seien "`Grammatik-Nieten"'.
Die Unterschiede in der konkreten Wortwahl zeigen, dass es sich hier um eine zwar gleichermaßen reißerische, aber dennoch voneinander unabhängige journalistische Aufbereitung handelt, nicht etwas um wörtliche Zitate der Durchführenden des Tests.
Trotzdem schreibt die Presse weiter, die "`Professoren [seien] erschüttert"' (Spiegel Online), und es müsse folglich "`in der Schule wieder mehr Grammatik gepaukt"' werden (Süddeutsche Zeitung Online). 
Einen größeren Schaden an der Sache können Medienberichte kaum auslösen.
In der Presse werden vor allem die Studierenden geradezu als die Schuldigen aufgebaut und entsprechend als blamable Nieten usw.\ tituliert.
Für die intendierte Sache, also eine Stärkung des Grammatikunterrichts an Schulen, leistet dies in jedem Fall keinen Beitrag.
Höchstens lehnt sich das angepeilte Zeitungspublikum zufrieden in der Gewissheit zurück, dass eben die Schule und die Jugend heutzutage nichts mehr taugen.%
\footnote{Ich wiederhole, dass ich mich hier auf die journalistische Aufarbeitung beziehe.
Diese Formulierungen stammen gewiss nicht von den Durchführenden der Studie.}

Unabhängig von der journalistischen Verarbeitung gibt es aber ein weiteres Problem mit solchen Tests.
Das Ziel der Durchführenden in Bayern war es wohl, den Grammatikunterricht an Schulen zu stärken, indem entsprechende Wissenslücken aufgezeigt wurden.
Das Argument scheint dabei gewesen zu sein, dass Studierende nach ihrer Schulzeit nicht hinreichend auf das Germanistikstudium vorbereitet seien.
Das ist allerdings, wie weiter unten in diesem Kapitel argumentiert wird, überhaupt nicht das Ziel des schulischen Unterrichts in der Grammatik des Deutschen, ebensowenig wie es das Ziel des Geschichtsunterrichts ist, auf ein Studium der Geschichtswissenschaft vorzubereiten.
Noch viel weniger ist das Ziel des Deutschunterrichts das "`Pauken von Grammatik"'.
Sowohl die Fachdidaktik des Deutschen als auch die didaktisch informierte germanistische Sprachwissenschaft wird selbstverständlich einen Nürnberger Trichter für deklaratives schulisches Grammatikwissen als Absurdität zurückweisen.

Neben diesem Versuch eines Eingangstests gibt es eine aktuelle aktive Diskussion um Funktionen und Nutzen von Studieneingangstests (siehe vor allem \citealt{Schindler2016}, aber \zB auch \citealt{Bremerichvos2016} und \citealt{FuhrhopTeuber2016}).
\citet[16]{Schindler2016} benennt neben der Funktion im Sinne einer Zulassungsschranke für das Studium vor allem zwei Funktionen solcher Tests, nämlich erstens die Diagnostik für das studierende Individuum, anhand derer gezielt Defizite erkannt und rechtzeitig ausgeglichen werden können.
Zweitens führt die Autorin an, dass Eingangstests Bewusstsein und Interesse für die Inhalte des Fachs bei Studierenden wecken können.
In \citet[226]{SchaeferSayatz2017a} argumentieren wir, dass eine weitere Funktion der Tests die Optimierung der Lehre im Studium im Sinne einer Ausrichtung auf die Bedürfnisse der Studierenden sein sollte.
Hier werden jetzt die wesentlichen Ergebnisse des Experiments berichtet, vor um bei Studierenden das Bewusstsein für die Inhalte des Fachs und die Verbindung zwischen Schulgrammatik und linguistischem Fachwissens zu wecken (Schindlers zweite Funktion), und um Lehrende anzuregen, ähnliche Evaluationen vorzunehmen und ihre Lehrinhalte und Lehrmethoden entsprechend an die Ergebnisse anzupassen (die von \citealt{SchaeferSayatz2017a} benannte Funktion).

Wir haben 220 Studierende der Freien Universität Berlin, die überwiegend in Berlin und Brandenburg zur Schule gegangen sind, einen freiwilligen anonymen Test vorgelegt, der aus gezielt ausgewählten Aufgaben aus Lehrwerken bestand, die in diesen Bundesländern für die Jahrgangsstufen sechs bis zehn (in einem Fall für die Grundschule) zugelassen sind.
Es handelte sich also um Aufgaben, die die Studierenden so oder ähnlich wahrscheinlich in ihrer Schulzeit gelöst haben.
Insgesamt wurden zehn Aufgaben ausgewählt und ausgewertet, die verschiedene Schwierigkeitsgrade und verschiedene Themen der Grammatik abdeckten, die auf der Seite der systemexternen Funktion ebenfalls verschiedenen Bereichen zuzuordnen sind.
Die Teilnehmenden, die aus allen Semestern des BA-Studiums kamen, studierten zu 37,7\% Grundschullehramt (83 Personen), zu 37,3\% Deutsch für das Lehramt (82 Personen), zu 23,2\% Deutsch als Fachwissenschaft (51 Personen) und zu 1,8\% andere Fächer (4 Personen).
Zunächst wurde überprüft, ob wirklich durchweg ungenügende "`Leistung"' erbracht wurde.
Abbildung~\ref{fig:grammatikkentnissevonstudierenden001} zeigt, dass dies nicht der Fall ist.

\begin{figure}[htpb]
  \centering
  \resizebox{0.9\textwidth}{!}{\includegraphics{figures/notenspiegel}}
  \caption{Verteilung der auf akademische Noten umgerechneten Leistungen im Experiment aus \citet{SchaeferSayatz2017a}}
  \label{fig:grammatikkentnissevonstudierenden001}
\end{figure}

Nur 15\% der Teilnehmenden wären durchgefallen, wenn es sich um eine Prüfung gehandelt hätte.
Die restlichen Leistungen verteilen sich nach der Umrechnung auf akademische Noten nahezu gleichmäßig zwischen 1,3 und 4,0.
Solch ein Bild ist vollständig erwartbar, wenn davon ausgegangen wird, dass nicht alle Teilnehmenden dasselbe gelernt haben und die Lernzeitpunkte teilweise sieben Jahre zurückliegen.
Es ist davon auszugehen, dass für jedes andere Schulfach ähnliche Ergebnisse erzielt würden, denn immerhin hatten die Teilnehmenden keine Gelegenheit, sich gezielt auf die gestellten Fragen vorzubereiten.

Allerdings haben, wie oben erwähnt, Studierende aller Bachelor-Semester teilgenommen, und es sollte deshalb eigentlich nicht wundernehmen, wenn das Ergebnis besser als das des bayrischen Eingangstests wäre, bei dem schließlich nur Studierende des ersten Semesters befragt wurden, die noch keine spezifische Ausbildung in germanistischer Linguistik erhalten haben.
Abbildung~\ref{fig:grammatikkentnissevonstudierenden002} schlüsselt die Ergebnisse in Prozent nach Studienjahr auf.

\begin{figure}[htpb]
  \centering
  \resizebox{0.9\textwidth}{!}{\includegraphics{figures/semester}}
  \caption{Verteilung der erreichten Prozent im Experiment aus \citet{SchaeferSayatz2017a}, gegliedert nach Studienjahr in einem sogenannten Beanplot; jede kurze waagerechte Linie entspricht einem individuellen Ergebnis, die Hüllen entsprechen einer Dichteschätzung der empirischen Verteilung pro Gruppe, die dicken längeren waagerechten Linien markieren den Mittelwert pro Gruppe}
  \label{fig:grammatikkentnissevonstudierenden002}
\end{figure}

Wie man leicht sieht, werden die Ergebnisse im verlauf des Studiums im Mittel nicht besser.
Es gibt eine leichte Verschiebung der Verteilung (des Bauchs der jeweiligen Beanplots) nach oben, insbesondere im dritten Studienjahr.
Allerdings wäre schon aufgrund des Ausscheidens leistungs- und motivationsschwächerer Studierender (Fachwechsel oder Studienabbruch) im Laufe der Semester eine stärkere Verbesserung der Leistung zu erwarten.
Wie wir in \citet[242--243]{SchaeferSayatz2017a} argumentieren, kann der Grund der beobachteten Stagnation nicht lokal in der Freien Universität gesucht werden, da sich im Vergleich der Kurrikula mehrerer Universitäten aus verschiedenen Bundesländern zeigt, dass die Freie Universität ein sehr typisches Profil in ihren germanistischen Studiengängen anbietet.

Wir gehen davon aus, dass das Problem vielmehr eine meist nicht optimale Kopplung der universitären Inhalte an die Berufsziele der Studierenden ist.
Wie oben deutlich wurde, sind ca.\ 75\% der Studierenden an der Freien Universität angehende Lehrpersonen, und die Lehre müsste sich deutlich an diesem Berufsziel orientieren.
Von einer anderen Seite nähern sich auch \citet{TopalovicDuenschede2014} demselben Problem.
In einer Bundesweiten Umfrage mit 1.019 Lehrpersonen aus allen Schulformen fanden Sie im Jahr 2013 heraus, dass 48\% der Lehrpersonen sich durch ihre Ausbildung nicht hinreichend auf den Grammatikunterricht vorbereitet fühlen \citep[76--77]{TopalovicDuenschede2014}.


\subsection{Studentische Sichtweisen auf Studium und Schulunterricht}
\label{sec:studentischesichtweisenaufstudiumundschulunterricht}

\begin{figure}[htpb]
  \centering
  \resizebox{0.9\textwidth}{!}{\includegraphics{figures/funktionengrammatikunterricht}}
  \caption{}
  \label{fig:studentischesichtweisenaufstudiumundschulunterricht001}
\end{figure}


\section{Form und Funktion in der Grammatik}
\label{sec:formundfunktionindergrammatik}

Zunächst ist festzustellen, dass frühere Auflagen dieses Buches aufgrund bestimmter Formulierungen und Analysen teilweise missverstanden wurden, und dass dieses Missverständnis direkt die Tauglichkeit des Buches für das Lehramtsstudium berührt.
Ein anonymes Gutachten der ersten Auflage attestierte dem Buch (recht kritisch) einen radikal formalistischen Ansatz, der grammatische Mittel vollständig losgelöst von ihrer Funktion betrachtet, und der damit das Wesentliche der Sprache, also ihre bedeutungsvermittelnden und kommunikativen Funktionen außer acht lässt.
So radikal klangen zugegebenermaßen auch einige Formulierungen in den einleitenden Kapiteln des Buches, die bis zur dritten Auflage erheblich abgetönt, aber bewusst nicht vollständig entfernt wurden.
Für die schulische Lehre wäre solch ein radikaler Ansatz jedoch völlig verfehlt und geradezu anachronistisch, und auch in der universitären Lehre wäre ein rein formorientierter Ansatz hochgradig fragwürdig.%
\footnote{Das Adjektiv \textit{anachronistisch} ist hier auf Basis der Geschichte des schulischen Unterrichts in deutscher Grammatik zu verstehen, die \zB in den Abschnitten~3.1 und~3.2 aus \citet{Bredel2013} zusammengefasst wird.}
Das Buch versteht sich aber einerseits nicht als exklusive Lektüre für das gesamte Studium.
Es sollte also von weiterer Lektüre flankiert werden, die dann das gesamte relevante Spektrum der Linguistik des Deutschen abdeckt.


Außerdem bewies dieser Text trotz einer stellenweise formlastigen Rhetorik von der ersten Auflage an, dass es die Beziehung von Form und Funktion durchaus berücksichtigt, zum Beispiel bei der Beschreibung der Intonation (Kapitel~\ref{sec:phonologie}), der Flexionskategorien (Kapitel~\ref{sec:nominalflexion} und ~\ref{sec:verbalflexion}) oder bei der Behandlung von semantischen Rollen und Passivphänomenen (Kapitel~\ref{sec:relationenundpraedikate}).

Es muss hier jedoch vor allem zwischen \textit{systeminterner} und \textit{systemexterner} (\zB kommunikativer) Funktion unterschieden werden.\index{Form und Funktion}
Das System der vier nominalen Kasus ist zum Beispiel nur äußerst schlecht mit \textit{systemexternen} Funktionen zu verbinden, die direkt irgendetwas mit Semantik, Pragmatik, Textaufbau und Argumentationstechniken etc.\ zu tun haben.
Seine \textit{systeminterne} Funktion im grammatischen System ist allerdings von erheblicher Wichtigkeit, insbesondere angesichts der Möglichkeiten des Deutschen, die Bestandteile von Sätzen vergleichsweise frei im Satz zu positionieren.
Der Kasus eines Nomens kodiert nämlich vor allem die Relation dieses Nomens zu einem Verb (und im Fall des Genitivs primär zu einem anderen Nomen), und zwar in starker Abhängigkeit von teilweise auch semantisch motivierbaren Typen von Verben.%
\footnote{Ein Feuerwerk der systeminternen funktionalen Betrachtung ergibt sich in diesem Zusammenhang durch das Einbeziehen der Kasussynkretismen, also des Zusammenfalls von Formen.
Dieser fällt je nach Genus, Flexionsklasse und Numerus ganz unterschiedlich aus.
Zudem müssen Substantive, Adjektive und Artikel insgesamt betrachtet werden, da sie auf unterschiedliche Weise an der Kasusmarkierung beteiligt sind und einander diese sogar gegebenenfalls abnehmen.
Kapitel~\ref{sec:nominalflexion} beschäftigt sich ausführlich mit dem Kasussystem.}
Während also Kasus nicht direkt semantisch interpretierbar ist, ist er dennoch von großer Bedeutung für die Konstruktion der Bedeutung von Sätzen.
Darauf wird in dieser Einführung bei zahlreichen Gelegenheiten vertiefend eingegangen.
Wir geben Definition~\ref{def:internexternefunktion}, die sich auf Konzepte rückbezieht, die in Kapitel~\ref{sec:grammatik} eingeführt wurden.
Kapitel~\ref{sec:grundbegriffedergrammatik} führte bereits vertiefend in systeminterne Funktionen ein, zum Beispiel mit Begriffen wie \textit{Merkmal} und \textit{Wert} oder \textit{Relationen} (\textit{Strukturbildung}, \textit{Rektion} und \textit{Kongruenz}).

\Definition{Systeminterne und systemexterne Funktion}{\label{def:internexternefunktion}%
Sprachliche Formen (Formen von Wörtern, Sätzen usw.) haben durch die überwiegende Regelmäßigkeit ihrer Bildungen und Kombinationsmöglichkeiten eine Systematik (die Grammatik).
Systeminterne Funktionen grammatischer Mittel sind solche, die innerhalb dieses Systems gelten, \zB die Funktion von Kasus bei der Markierung von Beziehungen zwischen Nomina und Verben.
Systemexterne Funktionen grammatischer Mittel sind ihre Bedeutungen, ihre systematischen Beiträge zur Konstruktion von Kommunikationssituationen und Texten, ihre Art, bestimmte Register und Stile zu markieren usw.
\index{Funktion!systemintern und systemintern}
}

Es ist nun nicht davon auszugehen, dass sich interne und externe Funktionen stets sauber voneinander trennen lassen, und eine umfassende Betrachtung ist sicherlich das Ziel der Grammatikforschung und des Studiums.
Die Motivation, diese Einführung dennoch stark (wenn auch nicht dogmatisch) an der Form sprachlicher Äußerungen und ihren systeminternen Funktionen auszurichten, begründet sich aus meiner persönlichen Lehrerfahrung im Fach Deutsche Philologie.
Während diese Erfahrung gewiss subjektiven und anekdotischen Charakter hat, wurde sie durch \citet{SchaeferSayatz2017a} auf methodisch stringente Weise bestätigt, worauf in Abschnitt~\ref{sec:grammatikkentnissevonstudierenden} ausführlicher eingegangen wird.
Programmatisch formuliert setzt die \textit{systematische} Betrachtung von Form und Funktion sprachlicher Einheiten -- um die es nach meiner Auffassung sowohl in der Schule bzw.\ Fachdidaktik als auch in der universitären Ausbildung bzw.\ Fachwissenschaft geht -- eine genaue Kenntnis der Systematik der Formen und ihrer systeminternen Funktionen voraus.%
\footnote{Abweichende Versuche der Didaktik früherer Jahrzehnte bestätigen wahrscheinlich dieses Programm indirekt eher als dass sie es widerlegen, siehe Abschnitt~\ref{sec:deutschunterrichtunddiewellenderdidaktik} und die dort zitierte Literatur.}
In der universitären Lehre aber davon auszugehen \textit{oder zu verlangen}, dass Studierende vor dem ersten Semester bereits eine ausreichende Kenntnis des formalen grammatischen Systems erworben haben, ist allerdings verfehlt und potentiell fatal für ihre Ausbildung und ihre spätere berufliche Praxis.
Auch wenn es vielleicht nicht immer eine Mehrheit der Studierenden betrifft, so wird doch in der frühen Phase des Studiums immer wieder gerne Kasus direkt mit Bedeutungen identifiziert und mit den unzureichenden \textit{Grammatikerfragen} ("`\textit{wer oder was}"' usw.) analysiert (s.\ Abschnitt~\ref{sec:kasus}), grammatisches Genus wird mit Bedeutungsklassen identifiziert, Subjekte werden als der "`Satzgegenstand"' \textit{definiert}, aber mit dem Nominativ \textit{identifiziert}, Wortarten werden semantisch motiviert usw.%
\footnote{Gleichermaßen werden nahezu rein formale Phänomene nicht zuverlässig beschrieben.
Dass zum Beispiel die \textit{Klatschmethode} (s.\ Abschnitt~\ref{sec:silben}) bei der Beschreibung der Silbentrennung nicht besonders weit trägt, ist Studierenden meist bewusst.
Die Fähigkeit, die Silbifizierung im Deutschen und ihre graphematischen Folgen anderweitig und präzise zu beschreiben, kann und darf hingegen trotzdem nicht vorausgesetzt werden.}
Während alle diese Phänomene (einschließlich der Wortklassen) hochinteressante und relevante Beziehungen zur Bedeutung sowie zahlreiche andere externe Funktionen haben, haben sie doch zunächst eine formale Seite und Systematik, die für jede weitere Betrachtung bekannt sein muss.
Die insgesamt stark formorientierte Ausrichtung des Buches entspringt also der Überzeugung, dass vor allen funktionalen Betrachtungen, die selbstverständlich Teil des Studiums sein sollten, mit einer gründlichen Analyse der Form und der systeminternen Funktionen begonnen werden sollte, die auch mit eingebrannten bedeutungsfixierten Didaktisierungen aus der Schulzeit (wie den semantischen Definitionen von Wortklassen) aufräumt.

Komplexere Phänomene wie -- um ein arbiträres Beispiel herzunehmen -- Relativsätze werden zudem von Studierenden zu Studienbeginn formal oft gar nicht verstanden, und eine Diskussion über ihre Funktion erübrigt sich damit.
Sehen wir uns die Folgen einer unzureichenden formalen Analyse etwas genauer an.
Es ist gewiss ein vordergründig rein formaler Sachverhalt, dass ein Relativsatz ein im Prinzip vollständiger Satz ist, der aber so konstruiert wird, dass eins der in ihm vorkommenden "`Satzglieder"' (besser gesagt eine der Ergänzungen und Angaben, s.\ Abschnitte~\ref{sec:valenz} und~\ref{sec:konstituentenundsatzglieder}) durch ein spezielles Pronomen realisiert wird, welches sich dann auf ein Bezugsnomen außerhalb des Relativsatzes bezieht.
Das Bezugsnomen steuert die semantische Interpretation (Bedeutung) des Relativpronomens bei.
Wir erhalten (\ref{ex:formundfunktionindergrammatik002}) parallel zu (\ref{ex:formundfunktionindergrammatik001}) als einen einfachen Fall eines Relativsatzes.

\begin{exe}
  \ex
  \begin{xlist}
    \ex{\label{ex:formundfunktionindergrammatik001} Eine\slash die Kommilitonin hat gelacht.}
    \ex{\label{ex:formundfunktionindergrammatik002} eine\slash die Kommilitonin, die gelacht hat}
  \end{xlist}
\end{exe}

Es fällt rein formal auf, dass zumindest die Verbformen im Relativsatz andere Positionen einnehmen als im unabhängigen Satz.
Dass (und \textit{auf welche Weise}) dies völlig regelmäßig geschieht, muss allerdings erst einmal verstanden werden, um die Beziehung zwischen den beiden Konstruktionen überhaupt wahrnehmen zu können.
Wird das durch einen Relativsatz modifizierte Substantiv wie in (\ref{ex:formundfunktionindergrammatik003}) in einer übergeordneten Satzperiode verwendet, wird die systeminterne Funktion schnell klar.
Der Relativsatz ermöglicht es uns stark vereinfacht gesagt, zwei im Grunde vollständige Sätze zu einem zu kombinieren (Hypotaxe).
Im Gegensatz zu einer einfachen aufzählenden Verbindung (Parataxe) wie in (\ref{ex:formundfunktionindergrammatik004}) nutzt der Relativsatz die Identität eines nominalen Bezugs (also der Bedeutung zweier Nomina) und vermeidet dadurch die zweimalige unabhängige Aufnahme der semantischen Referenz.

\begin{exe}
\ex
\begin{xlist}
  \ex{\label{ex:formundfunktionindergrammatik003} Die Kommilitonin, die gelacht hat, heißt Camilla.}
  \ex{\label{ex:formundfunktionindergrammatik004} Eine Kommilitonin hat gelacht, und die\slash sie heißt Camilla.}
\end{xlist}
\end{exe}

Die systeminterne Funktion hat also eine semantische Funktion in ihrem Gefolge, die in ihrer Systematik nur als Folge der formalen zu verstehen ist.
Die Erkenntnis, dass darüberhinaus die ebenfalls externe \textit{textuelle} oder \textit{rhetorische Funktion} eines Relativsatzes in den Bereichen des \textit{Verdichtens} und \textit{Präzisierens} anzusiedeln ist, kann nur in Zusammenhang mit den ersten beiden Schritten vollzogen werden.
Ohne diese Schritte kann überhaupt nicht sicher identifiziert werden, was ein Relativsatz denn überhaupt sein soll, woran wir ihn erkennen und wie wir ihn formulieren.
Selbst wenn wir davon ausgehen, dass (\ref{ex:formundfunktionindergrammatik003}) und (\ref{ex:formundfunktionindergrammatik004}) bedeutungsgleich (aber eventuell rhetorisch verschieden) sind, kommen wir nicht umhin, festzustellen, dass ihre Verwendung hochgradig stil-, register- und medienspezifisch ist, womit weitere externe Funktionen angesprochen werden.
In fast allen bildungs- und schriftsprachlichen Kontexten ist nämlich davon auszugehen, dass (\ref{ex:formundfunktionindergrammatik004}) unangenehm auffällt.
Verdichtungen und Präzisierungen, die komplexere formale Mittel wie Relativsätze einsetzen, kennzeichnen nämlich genau solche Kontexte, und wir irritieren unser sprachliches Gegenüber im Ernstfall erheblich, wenn wir in der Bildungssprache zu ausufernder Parataxe greifen.

Ganz unabhängig davon sehen wir an (\ref{ex:formundfunktionindergrammatik004}), dass die systeminternen Interaktionen weiter reichen, als uns vielleicht lieb ist.
Die parataktische Version zwingt uns nämlich, zweimal einen Artikel bzw.\ ein Pronomen zu verwenden.
Hier wurde zuerst der indefinite \textit{eine} und dann das Definitpronomen \textit{die} oder alternativ das Personalpronomen \textit{sie} verwendet.
Die andere mögliche Variante wäre die in (\ref{ex:formundfunktionindergrammatik005}), wobei diese sich allerdings bereits im Grenzbereich zwischen stilistischem Mangel und Ungrammatikalität bewegt.

\begin{exe}
  \ex[?]{\label{ex:formundfunktionindergrammatik005} Die Kommilitonin hat gelacht, und die\slash sie heißt Camilla.}
\end{exe}

Gehen wir noch weiter und sehen uns informationsstrukturelle Funktionen an, stellen sich auch diese als mit der Form eng verknüpft heraus.
Weder (\ref{ex:formundfunktionindergrammatik004}) noch (\ref{ex:formundfunktionindergrammatik005}) sind nämlich wirklich robust bedeutungsgleich mit (\ref{ex:formundfunktionindergrammatik003}).
Ohne tief in die Details zu gehen, zeigt (\ref{ex:formundfunktionindergrammatik007}) durch eine besondere Betonung (also ein formales Mittel), dass wir dank des Relativsatzes Ausdrucksmöglichkeiten haben, die wir mit der parataktischen Version schlichtweg nicht haben.
Die in Majuskeln geschriebenen Silben sind mit hervorhebender Betonung zu lesen.

\begin{exe}
  \ex{\label{ex:formundfunktionindergrammatik007} DIE Kommilitonin, die geLACHT hat, heißt Camilla.}
\end{exe}

In geschriebenen Texten markieren wir Intonation normalerweise nicht.
Dort werden die entsprechenden Sätze ohne besondere Markierungen aber trotzdem ähnlich verwendet, oder es kommen andere Markierung wie zum Beispiel Adverbiale oder besondere Wortstellungen zum Zug.
Als Beispiel wird in (\ref{ex:formundfunktionindergrammatik008}) ein Aufzählungskontext gegeben.
Ohne Relativsätze wären wir hier im Ausdruck erheblich eingeschränkt.

\begin{exe}
  \ex{\label{ex:formundfunktionindergrammatik008} Die etwas ruhigere Kommilitonin heißt Kiki. Die Kommilitonin (hingegen), die gelacht hat, heißt Camilla.}
\end{exe}

Gleichermaßen gibt es für Sätze wie (\ref{ex:formundfunktionindergrammatik009}), die allgemeine Aussagen formulieren, kaum eine semantisch adäquate parataktische Variante, die eine vertretbare Länge hat und stilistisch unauffällig ist.

\begin{exe}
  \ex{\label{ex:formundfunktionindergrammatik009} Die Mitspielerin, die zuletzt lacht, gewinnt.}
\end{exe}

Wir feiern mit dem Relativsatz also ohne weiteres ein grammatisches Fest, bei dem Wissen aus unterschiedlichsten Teilen der Grammatik abgerufen wird.
Um die semantischen, textuellen und rhetorischen, stilistischen und registerbezogenen Funktionen von Relativsätzen zu benennen, müssen wir in der Lage sein, Relativsätze zu erkennen, ihre Form zu analysieren, und dieses Wissen in einen größeren Zusammenhang zu stellen.

Wir sind weiter oben von der Frage ausgegangen, welche Kenntnisse Studierende in die Lehramtsausbildung mitbringen, und wir haben argumentiert, dass formale Analysen eine Bedingung für eine systematische funktionale Betrachtung von Sprache sind.
Angesichts der Komplexität des grammatischen Systems und der externen Funktionen grammatischer Mittel (wie am Beispiel des Relativsatzes vorgeführt) können wir davon ausgehen, dass es zielführender ist, im Studium einen systematischen Neuanfang zu wagen, der sich nicht auf vermeintlich vorhandenes Schulwissen stützt.
Die fundamentale Rechtfertigung dieses Ansatzes liegt zudem in der oft übersehenen Tatsache, dass der Grammatikunterricht in der Schule grundlegend andere Ziele verfolgt als die universitäre Ausbildung von zukünftigen Lehrpersonen.
Darum wird es in diesem Kapitel weiter unten noch ausführlicher gehen.

Nicht kontrovers ist allerdings die Minimalforderung, dass Lehrkräfte, die von Universitäten und Hochschulen mit einem erfolgreichen Studienabschluss in das Schulfach Deutsch entlassen werden, diejenigen Kenntnisse mitbringen müssen, die sie Kindern und Jugendlichen vermitteln sollen.
Das Studium muss diese Kenntnisse über den weitaus höheren Anspruch des spezifischen Studienwissens hinaus sicherstellen.
In diesem Zusammenhang ist es nun bemerkenswert, dass Dozierende an Universitäten oft keinerlei Daten darüber haben, was sie bei Studierenden voraussetzen können und was nicht.
Hier ist dringend nachzuhelfen, vor allem weil die Gefahr besteht, dass ein in seinem oft nicht gut definiertes \textit{Schulwissen} von Lehrpersonen an Universitäten stillschweigend vorausgesetzt und auf dieses "`aufgebaut"' wird.
In diesem Zuge treten wir auch aus dem Nebel des Anekdotischen heraus und sehen empirisch klar in Abschnitt~\ref{sec:kenntnisseerwartungenundmotivationvonstudierenden}.


