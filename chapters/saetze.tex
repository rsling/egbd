\chapter{Sätze}
\label{sec:saetze}

\section{Hauptsatz und Matrixsatz}
\label{sec:hauptsatzundmatrixsatz}

\subsection{Formale Grundbegriffe für Satzstrukturen}
\label{sec:formalegrundbegriffefuersatzstrukturen}

Außer dem durch Subjunktion eingeleiteten Nebensatz -- also der Subjunktionsphrase (SP) mit eingebetteter VP -- gibt es weitere wichtige Satztypen im Deutschen, die sich jeweils durch eine besondere Konstituentenstellung auszeichnen.
Das sind die des \textit{unabhängigen Aussagesatzes} wie in (\ref{ex:hauptsatzundmatrixsatz002}) und die des \textit{Fragesatzes mit Fragepronomen} (\textit{w}"=Fragesatz) wie in (\ref{ex:hauptsatzundmatrixsatz003}), die des \textit{Entscheidungsfragesatzes} (\textit{Ja\slash Nein-Frage}) wie in (\ref{ex:hauptsatzundmatrixsatz004}) und die des \textit{Relativsatzes} wie in (\ref{ex:hauptsatzundmatrixsatz005}) sowie des sehr ähnlichen \textit{eingebetteten \textit{w}"=Fragesatzes} wie in (\ref{ex:hauptsatzundmatrixsatz006}).

\begin{exe}
  \ex\label{ex:hauptsatzundmatrixsatz001}
  \begin{xlist}
    \ex{\label{ex:hauptsatzundmatrixsatz002} Wahrscheinlich hat der Arzt das Bild gekauft.}
    \ex{\label{ex:hauptsatzundmatrixsatz003} Was hat der Arzt gekauft?}
    \ex{\label{ex:hauptsatzundmatrixsatz004} Hat der Arzt das Bild gekauft?}
    \ex{\label{ex:hauptsatzundmatrixsatz005} (Das ist das Bild,) das der Arzt gekauft hat.}
    \ex{\label{ex:hauptsatzundmatrixsatz006} (Ischariot fragt sich,) was der Arzt gekauft hat.}
  \end{xlist}
\end{exe}

Während schon definiert wurde, was wir unter einem Nebensatz verstehen (Definition~\ref{def:nebensatz} auf Seite~\pageref{def:nebensatz} und Abschnitt~\ref{sec:verbphraseundverbkomplex}), wird jetzt definiert, was wir unter einem \textit{unabhängigen Satz} -- landläufig \textit{Hauptsatz} -- verstehen, vgl.\ Definition~\ref{def:satz}.%
\footnote{Zum Begriff der \textit{Abhängigkeit} siehe genauer den Begriff der \textit{Dependenz} in Abschnitt~\ref{sec:phrasenkoepfeundmerkmale}.}

\Definition{Unabhängiger Satz}{\label{def:satz}%
Ein \textit{unabhängiger Satz} (den man auch einen \textit{Hauptsatz} oder abgekürzt einfach einen \textit{Satz} nennt) ist eine Struktur, in der alle Konstituenten unmittelbar oder mittelbar von einem finiten Verb abhängig sind, das selber von keiner anderen syntaktischen Einheit abhängig ist, und das damit auf keinen Fall regiert sein kann.
Alle Valenzanforderungen des Verbs und aller von ihm unmittelbar oder mittelbar abhängigen Konstituenten werden innerhalb dieser Struktur erfüllt.
\index{Satz}
}

Folglich ist (\ref{ex:hauptsatzundmatrixsatz008}) ein unabhängiger Satz, weil \textit{ist} ein finites Verb ist, das nicht regiert wird, bzw.\ noch allgemeiner ein Verb, das von keiner anderen Einheit syntaktisch abhängt.
Außerdem sind offensichtlich alle Valenzanforderungen des Verbs und der von ihm abhängigen Einheiten erfüllt.
Hingegen kann (\ref{ex:hauptsatzundmatrixsatz009}) kein unabhängiger Satz sein, weil zwar genau ein finites Verb vorkommt, dieses aber von \textit{dass} regiert wird und damit syntaktisch von \textit{dass} abhängt.
Konstruktionen wie (\ref{ex:hauptsatzundmatrixsatz010}) und (\ref{ex:hauptsatzundmatrixsatz011}) können zwar als Äußerungen verwendet werden, aber in dem hier vertretenen Verständnis sind sie keine Sätze.
Aus Sicht der Grammatik ist diese Auf"|fassung durchaus zielführend, weil beide Äußerungen syntaktisch ganz anders aufgebaut sind als (\ref{ex:hauptsatzundmatrixsatz008}).

\begin{exe}
  \ex\label{ex:hauptsatzundmatrixsatz007}
  \begin{xlist}
    \ex{\label{ex:hauptsatzundmatrixsatz008} Die Post ist da.}
    \ex{\label{ex:hauptsatzundmatrixsatz009} dass die Post da ist}
    \ex{\label{ex:hauptsatzundmatrixsatz010} Hurra!}
    \ex{\label{ex:hauptsatzundmatrixsatz011} Nieder mit dem König!}
  \end{xlist}
\end{exe}

\textit{Hauptsatz} und \textit{Nebensatz} sind zunächst \textit{kategoriale} Begriffe.
Sie ordnen syntaktische Strukturen Kategorien zu, die über bestimmte Merkmale (zum Beispiel bestimmte Konstituentenstellungen) definiert sind.
Wichtig ist aber außerdem der Begriff der \textit{Matrix} bzw.\ spezieller der des \textit{Matrixsatzes}.
Er wird verwendet, um das Verhältnis von einem Hauptsatz und einem in ihn eingebetteten Nebensatz (oder allgemeiner einer einbettenden und einer eingebetteten Konstituente) zu beschreiben.
Der Begriff \textit{Matrix}(\textit{satz}) ist also ein \textit{relationaler} Begriff, der eine Beziehung zwischen zwei Einheiten beschreibt (in diesem Fall eine Teil-Ganzes-Beziehung), vgl.\ Definition~\ref{def:matrixsatz}.

\Definition{Matrix}{\label{def:matrixsatz}%
Die \textit{Matrix} einer Konstituente ist die Konstituente, in die sie unmittelbar eingebettet ist.
\index{Matrix}
}

In (\ref{ex:hauptsatzundmatrixsatz012}) ist zur Illustration die Matrix jeweils in [~]\Sub{M} und die eingebettete Konstituente in [~]\Sub{E} eingeklammert.
Die Interpunktion wird im Sinne der Übersichtlichkeit weggelassen.

\begin{exe}
  \ex\label{ex:hauptsatzundmatrixsatz012}
  \begin{xlist}
    \ex{\label{ex:hauptsatzundmatrixsatz013} [Der Arzt weiß [dass Ischariot Maler ist]\Sub{E} ]\Sub{M}}
    \ex{\label{ex:hauptsatzundmatrixsatz014} Der Arzt kennt [den Mann [der die Bilder gemalt hat]\Sub{E} ]\Sub{M}}
  \end{xlist}
\end{exe}

Man würde von \textit{Nebensätzen und ihren Matrixsätzen} sprechen.
Bei Relativsätzen wie in (\ref{ex:hauptsatzundmatrixsatz014}) würde man von der \textit{Matrix-NP} (hier \textit{den Mann}) sprechen, weil der Relativsatz eine unmittelbare Konstituente dieser NP ist, aber nur eine mittelbare Konstituente des ganzen Satzes.
Vgl.\ auch Abschnitt~\ref{sec:relativsaetze} zu Relativsätzen.
Genauso kann man in (\ref{ex:hauptsatzundmatrixsatz015}) von der \textit{Matrix-NP} für die eingebettete PP sprechen.

\begin{exe}
  \ex{\label{ex:hauptsatzundmatrixsatz015} [das Pferd [mit der schönsten Mähne]\Sub{E} ]\Sub{M}}
\end{exe}

In diesem Kapitel geht es aber nicht generell um eingebettete Konstituenten und ihre jeweilige Matrix, sondern vor allem um Nebensätze und unabhängige Sätze.
In Abschnitt~\ref{sec:funktionenvonsatzartigenkonstituenten} und Abschnitt~\ref{sec:funktionaleunterschiedezwischennebensatztypen} wird kurz auf die semantischen und kommunikativen Funktionen dieser Einheiten eingegangen.


\subsection{Funktionen von satzartigen Konstituenten}
\label{sec:funktionenvonsatzartigenkonstituenten}

\index{Form und Funktion}

Die hier verwendete Definition von "`unabhängigen Sätzen"' bezieht sich ausschließlich auf die syntaktische Unabhängigkeit der Sätze, sie ist also rein formal.
Unabhängige und eingebettete Sätze haben außerdem bestimmte kommunikative Funktionen und stilistische oder registerbezogene Eigenschaften.
Um diese Funktionen und Eigenschaften geht es in diesem und dem nächsten Abschnitt.
Es wird zuerst argumentiert, dass ein Konzept von kommunikativer Unabhängigkeit im Sinn der Sprechaktkonstitution nicht zuverlässig auf syntaktische Unabhängigkeitskonzepte abgebildet werden kann.

Der Unterschied zwischen unabhängigen und eingebetteten Sätzen wird erst dann relevant, wenn das, was ausgedrückt oder erreicht werden soll, komplexer ist, als es ein einzelner einfacher Satz ermöglicht.
Man spricht dann von \textit{parataktischen} und \textit{hypotaktischen} Konstruktionen, vgl.\ Definition~\ref{def:funktionenvonsatzartigenkonstituenten001}.

\index{Parataxe und Hypotaxe}

\Definition{Parataxe und Hypotaxe}{\label{def:funktionenvonsatzartigenkonstituenten001}%
\textit{Parataxe} (Nebenordnung) ist die Verbindung mehrerer unabhängiger Sätze.
\textit{Hypotaxe} (Unterordnung) ist die Verbindung von unabhängigen und abhängigen Sätzen.}

\index{Angabensatz}

Auch mit parataktischen Verbindungen kann man kohärent komplexere Sachverhalte ausdrücken, zum Beispiel durch Verwendung von Adverben, die auf semantischer oder pragmatischer Ebene syntaktisch unabhängige Sätze verbinden.
In (\ref{ex:funktionenvonsatzartigenkonstituenten002}) wird ein Beispiel gegeben.

\begin{exe}
  \ex\label{ex:funktionenvonsatzartigenkonstituenten002} 
  \begin{xlist}
    \ex\label{ex:funktionenvonsatzartigenkonstituenten003} Es regnet. Juliette geht trotzdem zum Training.
    \ex\label{ex:funktionenvonsatzartigenkonstituenten004} Obwohl es regnet, geht Juliette zum Training.
  \end{xlist}
\end{exe}

Die parataktische Variante in (\ref{ex:funktionenvonsatzartigenkonstituenten003}) und die hypotaktische Variante in (\ref{ex:funktionenvonsatzartigenkonstituenten004}) bringen ziemlich genau dasselbe zum Ausdruck.
Die parataktische verwendet zur Kennzeichnung der argumentativen Relation zwischen den beiden Teil-Sachverhalten (also dem Sachverhalt, dass es regnet, und dem Sachverhalt, dass Juliette zum Training geht) das Adverb \textit{trotzdem}, und in der hypotaktischen Variante kommt die spezielle Subjunktion \textit{obwohl} zum Einsatz.%
\footnote{Eine Version von (\ref{ex:funktionenvonsatzartigenkonstituenten003}) mit \textit{und} (\textit{Es regnet, und Juliette geht trotzdem zum Training.}) wäre ebenfalls als parataktisch einzuordnen.}
Man kann also auf keinen Fall sagen, dass die semantischen, kommunikativen, argumentativen (usw.) Funktionen von parataktischen und hypotaktischen Verbindungen scharf getrennt sind.
Der Unterschied ist zunächst einmal ganz eindeutig ein formaler.

Bestimmte hypotaktische Verbindungen sind allerdings nicht gut oder nicht ähnlich kompakt durch parataktische Verbindungen paraphrasierbar.
Während (\ref{ex:funktionenvonsatzartigenkonstituenten005}) ein weiteres Beispiel für eine gute parataktische Paraphrasierbarkeit ist, sieht es in (\ref{ex:funktionenvonsatzartigenkonstituenten008}) schlecht aus.

\begin{exe}
  \ex\label{ex:funktionenvonsatzartigenkonstituenten005}
  \begin{xlist}
    \ex\label{ex:funktionenvonsatzartigenkonstituenten006} Es regnet. Deswegen fährt Adrianna noch nicht nachhause.
    \ex\label{ex:funktionenvonsatzartigenkonstituenten007} Weil es regnet, fährt Adrianna noch nicht nachhause.
  \end{xlist}
  \ex\label{ex:funktionenvonsatzartigenkonstituenten008}
  \begin{xlist}
    \ex\label{ex:funktionenvonsatzartigenkonstituenten009} Kristine bleibt im Garten, damit sie nach der Hitze mehr vom Regen abbekommt.
    \ex\label{ex:funktionenvonsatzartigenkonstituenten010} Kristine bleibt im Garten. Das Ziel ist, dass sie nach der Hitze mehr vom Regen abbekommt.
    \ex\label{ex:funktionenvonsatzartigenkonstituenten011} Kristine bleibt im Garten. Das Ziel ist das Abbekommen von mehr Regen nach der Hitze.
  \end{xlist}
\end{exe}

Die Variante in (\ref{ex:funktionenvonsatzartigenkonstituenten010}) kommt zwar ohne die ursprüngliche Hypotaxe aus, aber im zweiten Satz wird dann eine neue Hypotaxe (der Ergänzungssatz mit \textit{dass}) benötigt.
Das wird in (\ref{ex:funktionenvonsatzartigenkonstituenten011}) zwar vermieden, aber nur um den Preis einer stilistisch extrem holzigen Nominalisierung (\textit{das Abbekommen von mehr Regen}).
Welche Mittel die Grammatik für welche semantisch-kommunikativen Verknüpfungen zur Verfügung stellt, ist also erst einmal der Willkür des grammatischen Systems überlassen.
Die Beispiele in (\ref{ex:funktionenvonsatzartigenkonstituenten002}) sind allerdings ein Hinweis darauf, dass Parataxe und Hypotaxe abhängig von Stil und Register unterschiedlich wirken.
In vielen Sprech- und Schreibsituationen wären (\ref{ex:funktionenvonsatzartigenkonstituenten003}), (\ref{ex:funktionenvonsatzartigenkonstituenten006}) und erst recht (\ref{ex:funktionenvonsatzartigenkonstituenten010}) und (\ref{ex:funktionenvonsatzartigenkonstituenten011}) auf"|fällig.
Dies sind vor allem sogenannte \textit{bildungssprachliche} Kontexte, in denen die Kompaktheit und größere Flexibilität hypotaktischer Mittel in der Regel erwünscht ist (siehe Abschnitte~\ref{sec:bildungsspracheundihrerwerb} und \ref{sec:formundfunktionindergrammatik}).

\index{Satz!unabhängig}

Statt der rein formalen Definitionen von Nebensätzen und unabhängigen Sätzen wie in den Definitionen~\ref{def:nebensatz} und~\ref{def:satz} könnte man nun versuchen, den Unterschied über eine pragmatische Basisfunktion zu definieren.
Ein unabhängiger Satz könnte einer sein, der selbständig einen \textit{Sprechakt} konstituieren kann.
Ein Sprechakt ist eine sprachliche Handlung, die eine bestimmte Wirkung hat, zum Beispiel das Geben einer Anweisung oder eines Versprechens, die Bestätigung einer bestimmten Information oder den Ausdruck von Emotionen.
Während das sicherlich eine \textit{prototypische} (also keineswegs eine \textit{notwendige}) Funktion syntaktisch unabhängiger Sätze ist, taugt es nicht als hartes Kriterium zur Unterscheidung der Funktionen von Nebensätzen und unabhängigen Sätzen.%
\footnote{\citet{PantherKoepcke2008} liefern eine Diskussion von Satz-Definitionen, die den notwendigerweise unscharfen Charakter berücksichtigt und von mehr oder weniger \textit{prototypisch} unabhängigen Sätzen spricht.
Dazu auch \citet{SchaeferSayatz2016}.}

Die Beispiele in (\ref{ex:funktionenvonsatzartigenkonstituenten002}) enthalten beide gleichermaßen satz- bzw.\ nebensatzartige Einheiten, die ohne einen passenden Kontext nicht als Sprechakt taugen.
Mit dem entsprechenden Kontext könnten sie aber beide einen Sprechakt konstituieren, weil auch Fragmente wie (\ref{ex:funktionenvonsatzartigenkonstituenten014}) in der gesprochenen (oder standardfernen geschriebenen) Kommunikation durchaus als eigenständige Äußerung funktionieren, wie der Dialog in (\ref{ex:funktionenvonsatzartigenkonstituenten019}) zeigt.

\begin{exe}
  \ex\label{ex:funktionenvonsatzartigenkonstituenten012}
  \begin{xlist}
    \ex[ ]{\label{ex:funktionenvonsatzartigenkonstituenten013} Juliette geht trotzdem zum Training.}
    \ex[*]{\label{ex:funktionenvonsatzartigenkonstituenten014} Obwohl es regnet.}
  \end{xlist}
  \ex\label{ex:funktionenvonsatzartigenkonstituenten019} Kristine: Ich habe gehört, dass Juliette sogar heute zum Training geht.\\
  Adrianna: Obwohl es regnet!
\end{exe}

Trotzdem ist (\ref{ex:funktionenvonsatzartigenkonstituenten013}) ein syntaktisch vollständiger Satz und (\ref{ex:funktionenvonsatzartigenkonstituenten014}) eben nicht.
Es stimmt also auch nicht, dass für einen Sprechakt immer mindestens ein vollständiger unabhängiger Satz (im formalen Sinn) vorhanden sein muss.
Ein unabhängiger Satz ist also aus Sicht der Grammatik vor allem eine formal zu definierende Einheit, und die kommunikativen Grundfunktionen (also die Eignung oder Nicht-Eignung, selbständig einen Sprechakt zu konstituieren) verteilen sich nicht eins-zu-eins auf unabhängige Sätze und Nebensätze.
Präferenzen für Hypotaxe in bestimmten bildungssprachlichen Stilen und Registern sind allerdings offensichtlich.


\subsection{Funktionale Unterschiede zwischen Nebensatztypen}
\label{sec:funktionaleunterschiedezwischennebensatztypen}

In diesem Abschnitt gehen wir kurz auf die \textit{systeminternen} Funktionsunterschiede zwischen den drei wichtigen Nebensatzarten, die in Abschnitt~\ref{sec:nebensaetze} besprochen werden, ein.
Unter den Nebensätzen gibt es zunächst eine Zweiteilung in Ergänzungssätze und Angabensätze auf der einen und Relativsätze auf der anderen Seite.
Wie im Rest dieses Kapitels mehrfach gezeigt wird, sind Ergänzungs- und Angabensätze syntaktisch gleich aufgebaut (eine SP mit eingebetteter intakter VP).
Relativsätze werden grundlegend anders konstruiert (als VP, aus der ein Relativelement extrahiert wurde, siehe Abschnitt~\ref{sec:relativsaetze}).
Außerdem ist ihre Matrix kein Satz, sondern eine NP, zu der sie ein Attribut bilden.

\index{Nebensatz!Funktion}
\index{Ergänzungssatz}

Alle drei Typen von Nebensätzen sind \textit{konzeptuell unabhängig}, Relativsätze allerdings mit leichten Einschränkungen.
Unter \textit{konzeptuell unabhängig} wird hier die Eigenschaft verstanden, alle Konstituenten zu enthalten, aus denen ein vollständiger unabhängiger Satz konstruiert werden könnte.
Für die Ergänzungs- und Angabensätze trifft das zu, denn man muss nur die Subjunktion weglassen und hat alle nötigen Konstituenten, um einen unabhängigen Satz zu bauen (siehe Abschnitt~\ref{sec:konstituentenstellunginunabhaengigensaetzen}).
Beispiel (\ref{ex:funktionaleunterschiedezwischennebensatztypen030}) illustriert diesen Sachverhalt.

\begin{exe}
  \ex\label{ex:funktionaleunterschiedezwischennebensatztypen030}
  \begin{xlist}
    \ex\label{ex:funktionaleunterschiedezwischennebensatztypen031} Adrianna weiß, [dass es bald regnen wird].
    \ex\label{ex:funktionaleunterschiedezwischennebensatztypen032} es bald regnen wird
    \ex\label{ex:funktionaleunterschiedezwischennebensatztypen033} Es wird bald regnen.
  \end{xlist}
\end{exe}

Den eingeklammerten Ergänzungssatz aus (\ref{ex:funktionaleunterschiedezwischennebensatztypen031}) können wir herausschneiden und die Subjunktion entfernen.
Wir erhalten Beispiel (\ref{ex:funktionaleunterschiedezwischennebensatztypen032}), das durch eine Umstellung zu dem vollständigen unabhängigen Satz in (\ref{ex:funktionaleunterschiedezwischennebensatztypen033}) wird.
Das Gleiche illustriert (\ref{ex:funktionaleunterschiedezwischennebensatztypen034}) für einen Angabensatz, wobei aufgrund der einfachen Struktur des Nebensatzes nicht einmal eine Umstellung erforderlich ist.
Auf der semantischen Seite geht die konzeptuelle Unabhängigkeit damit einher, dass diese Nebensätze genau wie unabhängige Sätze einen Sachverhalt wiedergeben.%
\footnote{Imperative und Fragesätze tun dies nicht.
Eine vollständige Beschreibung der Semantik von Imperativ- und Fragesätzen würde hier zu weit führen.
Imperative sind zudem auch morphologisch auf"|fällig (siehe Abschnitt~\ref{sec:formendesimperativs}).}

\index{Angabensatz}

\begin{exe}
  \ex\label{ex:funktionaleunterschiedezwischennebensatztypen034}
  \begin{xlist}
    \ex\label{ex:funktionaleunterschiedezwischennebensatztypen035} Adrianna und Kristine spielen Tennis, [während es regnet].
    \ex\label{ex:funktionaleunterschiedezwischennebensatztypen036} Es regnet.
  \end{xlist}
\end{exe}

\index{Relativsatz}

Bei Relativsätzen haben wir es mit einer leicht eingeschränkten konzeptuellen Unabhängigkeit zu tun, da dem Relativsatz das Substantiv fehlt, auf das er sich syntaktisch bezieht (siehe auch schon Abschnitt~\ref{sec:formundfunktionindergrammatik}).
Wenn wir das Material aus dem Relativsatz in (\ref{ex:funktionaleunterschiedezwischennebensatztypen038}) umstellen, kommt (\ref{ex:funktionaleunterschiedezwischennebensatztypen039}) heraus.
Es sind zwar Konstituenten in allen nötigen Kategorien vorhanden, aber das Ergebnis klingt zumindest semantisch unrund oder sogar ungrammatisch.

\begin{exe}
  \ex\label{ex:funktionaleunterschiedezwischennebensatztypen037}
  \begin{xlist}
    \ex[ ]{\label{ex:funktionaleunterschiedezwischennebensatztypen038} Kristine trifft später die Freundin, [deren Katze sie verwahren soll].}
    \ex[?]{\label{ex:funktionaleunterschiedezwischennebensatztypen039} Sie soll deren Katze verwahren.}
  \end{xlist}
\end{exe}

Die eingebetteten (also abhängigen Nebensätze) sind also alle konzeptuell unabhängig, der Relativsatz allerdings nur mit gewissen Einschränkungen.
Die Matrix ist es nicht in jedem Fall, und wir gehen jetzt der Frage nach der konzeptuellen Unabhängigkeit auch für die jeweiligen Matrix-Konstituenten nach.
Beim Angabensatz ist die Matrix ein Satz und auch ohne den Angabensatz vollständig, wie (\ref{ex:funktionaleunterschiedezwischennebensatztypen040}) durch die Weglassprobe zeigt.

\begin{exe}
  \ex\label{ex:funktionaleunterschiedezwischennebensatztypen040}
    \begin{xlist}
      \ex[ ]{\label{ex:funktionaleunterschiedezwischennebensatztypen041} Adrianna und Kristine spielen Tennis, während es regnet.} 
      \ex[ ]{\label{ex:funktionaleunterschiedezwischennebensatztypen042} Adrianna und Kristine spielen Tennis.}
    \end{xlist}
\end{exe}

Im Fall von Ergänzungssätzen funktioniert das sehr oft nicht, wie man in (\ref{ex:funktionaleunterschiedezwischennebensatztypen043}) sieht.
Dadurch, dass die Ergänzungssätze oft nicht weglassbare Ergänzungen sind, bleiben bei der Weglassprobe unvollständige Sätze wie in (\ref{ex:funktionaleunterschiedezwischennebensatztypen045}) zurück.%
\footnote{Sogenannte \textit{Korrelate} (siehe Abschnitt~\ref{sec:ergaenzungssaetze}) verändern das Bild etwas.}
Die Matrix ist also nicht konzeptuell unabhängig.

\begin{exe}
    \ex\label{ex:funktionaleunterschiedezwischennebensatztypen043}
    \begin{xlist}
      \ex[ ]{\label{ex:funktionaleunterschiedezwischennebensatztypen044} Juliette bemerkte, dass ihre Freundinnen keine Lust auf Tennis hatten.}
      \ex[*]{\label{ex:funktionaleunterschiedezwischennebensatztypen045} Juliette bemerkte.}
    \end{xlist}
\end{exe}

Beim Relativsatz ist die Matrix eine NP.
NPs sind sowieso nicht konzeptuell unabhängig, und das Weglassen des Relativsatzes ändert nichts daran, wie in (\ref{ex:funktionaleunterschiedezwischennebensatztypen046}) demonstriert wird.


\begin{exe}
  \ex\label{ex:funktionaleunterschiedezwischennebensatztypen046}
    \begin{xlist}
      \ex[*]{\label{ex:funktionaleunterschiedezwischennebensatztypen047} Juliettes Freundinnen, die in den Regen geraten sind.}
      \ex[*]{\label{ex:funktionaleunterschiedezwischennebensatztypen048} Juliettes Freundinnen.}
    \end{xlist}
\end{exe}

Diese Beobachtungen über die Form passen zur Semantik der jeweiligen Konstituenten.
Ein unabhängiger Satz drückt prototypischerweise einen Sachverhalt im weiteren Sinn (einschließlich Ereignissen, Zuständen usw.) aus, ein Substantiv verweist prototypischerweise auf (ontologische, nicht grammatische) Objekte.
Eine NP mit Relativsatz verweist auf Objekte, und über diese Objekte wird durch den Relativsatz ein zusätzlicher Sachverhalt ausgedrückt.
In (\ref{ex:funktionaleunterschiedezwischennebensatztypen047}) sind die Objekte Juliettes Freundinnen, und der zusätzliche Sachverhalt ist, dass sie in den Regen geraten sind.
Da die NP als Ganzes aber auf die Objekte und nicht den zusätzlichen Sachverhalt verweist, ist sie auch nicht konzeptuell unabhängig.

Im Fall des Ergänzungssatzes haben wir es mit \textit{zwei} Sachverhalten zu tun, die von der Matrix und dem Nebensatz ausgedrückt werden.
Die Matrix ist aber ohne den Nebensatz deshalb nicht konzeptuell unabhängig, weil der Sachverhalt, den der Nebensatz ausdrückt, Teil des Matrix-Sachverhalts ist.
In sehr typischen Fällen von Objektsätzen ist das Verb im Matrixsatz zum Beispiel ein Äußerungsverb wie \textit{sagen}, \textit{ausdrücken}, \textit{schreiben}.
Siehe Beispiel (\ref{ex:funktionenvonsatzartigenkonstituenten049}).

\begin{exe}
  \ex\label{ex:funktionenvonsatzartigenkonstituenten049} [Kristine sagt, [dass es bald regnen wird]\Sub{E}]\Sub{M}.
\end{exe}

Hier haben wir es zuerst mit dem Sachverhalt zu tun, dass es bald regnen wird.
Dieser muss nicht faktisch sein, aber er ist untrennbar Teil des zweiten Sachverhalts, dass Kristine \textit{sagt}, dass er faktisch ist. 
Wegen dieser Einschluss-Relation zwischen den Bedeutungen von Matrix und Nebensatz ist beim Ergänzungssatz die Matrix nicht konzeptuell unabhängig, der eingebettete Nebensatz aber schon.

Bei Angabensatzkonstruktionen sind ebenfalls zwei Sachverhalte beteiligt.
In Beispiel (\ref{ex:funktionenvonsatzartigenkonstituenten050}) sind es der Sachverhalt, dass Kristine sich freut, und der, dass es regnet.

\begin{exe}
  \ex\label{ex:funktionenvonsatzartigenkonstituenten050} Kristine freut sich, weil es regnet.
\end{exe}

Anders als bei der Ergänzungssatzkonstruktion ist hier keiner der beiden Sachverhalte Teil des anderen.
Vielmehr wird durch die Wahl der Subjunktion \textit{weil} zwischen beiden eine besondere argumentative Relation hergestellt, hier eine kausale Relation.
Dass es regnet, ist der Grund dafür, dass Kristine sich freut.
Die größere semantische Unabhängigkeit passt zu der Tatsache, dass hier Matrix und Nebensatz konzeptuell unabhängig sind.

Die unterschiedlichen systeminternen Funktionen und die semantischen Funktionen passen also bei den drei Nebensatztypen genau zueinander.
Tabelle~\ref{tab:nebensatzeigenschaften} fasst alle relevanten Eigenschaften der drei Typen und ihrer Matrix-Einheiten zusammen.

\begin{table}
  \centering
  \resizebox{\textwidth}{!}{%
  \begin{tabular}{llll}
    \lsptoprule
                                   & \textbf{Relativsatz}  & \textbf{Ergänzungssatz} & \textbf{Angabensatz} \\
    \midrule
    syntaktische Kategorie         & RS                    & SP                      & SP                     \\
    konzeptuell unabhängig         & ja (eingeschränkt)    & ja                      & ja                     \\
    Semantik                       & Sachverhalt           & Sachverhalt             & Sachverhalt            \\
    \midrule
    synt. Relation zur Matrix      & Attribut              & Subjekt\slash Objekt    & Angabe                 \\
    synt. Kategorie der Matrix     & NP                    & VP                      & VP                     \\
    Matrix konzeptuell unabhängig  & nein                  & nein                    & ja                     \\
    Semantik der Matrix            & Objekte (ontologisch) & Sachverhalt             & Sachverhalt            \\
    \lspbottomrule
  \end{tabular}
  }
  \caption{Eigenschaften der drei Typen von Nebensätzen und ihrer jeweiligen Matrix; zur syntaktischen Kategorie RS siehe Abschnitt~\ref{sec:relativsaetze}}
  \label{tab:nebensatzeigenschaften}
\end{table}

Damit beenden wir die kurzen Betrachtungen zur Funktion von Sätzen und Nebensätzen und wenden uns ganz ihren Formen zu.
In Abschnitt~\ref{sec:konstituentenstellungundfeldermodell} wird dafür argumentiert, dass man die verschiedenen Konstituentenstellungen wie in (\ref{ex:hauptsatzundmatrixsatz001}) mittels \textit{Bewegung} von Konstituenten aus der in Abschnitt~\ref{sec:verbphraseundverbkomplex} beschriebenen VP ableiten kann.
Außerdem wird das sogenannte \textit{Feldermodell} eingeführt, das diese Konstituentenstellungen deskriptiv klassifiziert.
In Abschnitt~\ref{sec:schematafuersaetze} werden dann Phrasenschemata für Sätze angegeben, die alle wichtigen Varianten der Konstituentenstellung beschreiben.
Schließlich wird in Abschnitt~\ref{sec:nebensaetze} auf Besonderheiten verschiedener Typen von Nebensätzen eingegangen.


\Zusammenfassung{%
In einem unabhängigen Satz können satzartige Strukturen (Nebensätze) eingebettet sein.
Die einbettende Struktur ist die \textit{Matrix} des eingebetteten Satzes.
Die Matrix ist für Ergänzungs- und Angabensätze der einbettende Satz, für Relativsätze die einbettende NP.
Hauptsätze bzw.\ unabhängige Sätze lassen sich nicht über eine einfache pragmatische Grundfunktion (wie die Fähigkeit, selbständig einen Sprechakt zu konstituieren) definieren oder von Nebensätzen vollständig abgrenzen.
Der definitorisch hinreichende Unterschied zwischen Hauptsätzen und Nebensätzen ist ein syntaktischer.
Die Bedeutung bzw.\ Funktion der drei Typen von Nebensätzen (Ergänzungssatz, Angabensatz, Relativsatz) lässt sich auf  die Frage zurückführen, ob die Sätze und ihre Matrix konzeptuell unabhängig sind.
}


\section{Konstituentenstellung und Feldermodell}
\label{sec:konstituentenstellungundfeldermodell}

\subsection{Konstituentenstellung in unabhängigen Sätzen}
\label{sec:konstituentenstellunginunabhaengigensaetzen}

\index{Verbphrase}
\index{Konstituente}
Die in Abschnitt~\ref{sec:verbphraseundverbkomplex} besprochene VP definiert die Abfolge der Konstituenten innerhalb der VP untereinander nicht exakt.
Dass man die Abfolge auch nicht spezifizieren muss, zeigen die Beispiele (\ref{ex:verbphrase120}) aus Kapitel~\ref{sec:phrasen}, hier als (\ref{ex:konstituentenstellunginunabhaengigensaetzen016}) wiederholt.

\begin{exe}
  \ex\label{ex:konstituentenstellunginunabhaengigensaetzen016}
  \begin{xlist}
    \ex{\ThePhrasenExOne}
    \ex{\ThePhrasenExTwo}
  \end{xlist}
\end{exe}

\index{Verbkomplex}
Die Stellung des Verbkomplexes als Ganzes am rechten Ende der VP ist allerdings festgelegt.
Im unabhängigen Aussagesatz gilt dies nicht in der gleichen Form.
In (\ref{ex:konstituentenstellunginunabhaengigensaetzen018}) sieht man an den beispielhaften Umformungen des eingeleiteten Nebensatzes (\ref{ex:konstituentenstellunginunabhaengigensaetzen017}) in verschiedene uneingeleitete Sätze (also Hauptsätze), welche zahlreichen Umstellungen möglich sind.

\begin{exe}
  \ex{\label{ex:konstituentenstellunginunabhaengigensaetzen017} dass Ischariot wahrscheinlich dem Arzt das Bild verkauft hat}
  \ex\label{ex:konstituentenstellunginunabhaengigensaetzen018}
  \begin{xlist}
    \ex{\label{ex:konstituentenstellunginunabhaengigensaetzen019} Ischariot hat wahrscheinlich dem Arzt das Bild verkauft.}
    \ex{\label{ex:konstituentenstellunginunabhaengigensaetzen020} Wahrscheinlich hat Ischariot dem Arzt das Bild verkauft.}
    \ex{\label{ex:konstituentenstellunginunabhaengigensaetzen021} Dem Arzt hat Ischariot wahrscheinlich das Bild verkauft.}
    \ex{\label{ex:konstituentenstellunginunabhaengigensaetzen022} Das Bild hat Ischariot wahrscheinlich dem Arzt verkauft.}
    \ex{\label{ex:konstituentenstellunginunabhaengigensaetzen023} Verkauft hat Ischariot wahrscheinlich dem Arzt das Bild.}
  \end{xlist}
\end{exe}

Wie bereits mehrfach angemerkt, steht hier das finite Verb immer an zweiter Stelle, und davor steht irgendeine andere Konstituente.
Die Optionen der Voranstellung aus (\ref{ex:konstituentenstellunginunabhaengigensaetzen018}) werden durch die Voranstellung von komplexeren Satzteilen erweitert, von denen einige in (\ref{ex:konstituentenstellunginunabhaengigensaetzen024}) gezeigt werden.
Die vor das finite Verb gestellte Konstituente ist eingeklammert.

\begin{exe}
  \ex\label{ex:konstituentenstellunginunabhaengigensaetzen024}
  \begin{xlist}
    \ex{\label{ex:konstituentenstellunginunabhaengigensaetzen025} [Das Bild verkauft] hat Ischariot wahrscheinlich dem Arzt.}
    \ex{\label{ex:konstituentenstellunginunabhaengigensaetzen026} [Dem Arzt das Bild verkauft] hat Ischariot wahrscheinlich gestern.}
  \end{xlist}
\end{exe}

Im Vergleich zur VP in Nebensätzen ergeben sich mindestens zwei Unterschiede.
Einerseits wird das finite Verb alleine (auch wenn es aus einem Verbkomplex mit mehreren Verbformen kommt) nach links gestellt.
Sowohl die infiniten Verbformen als auch eventuelle Verbpartikeln (nicht aber Verbpräfixe) bleiben als Rest eines Verbkomplexes ohne finite Form am rechten Rand zurück.
Außerdem wird eine andere (in erster Näherung beliebige) Konstituente davor gestellt.
Beispiel (\ref{ex:konstituentenstellunginunabhaengigensaetzen027}) zeigt das Zurückbleiben eines infiniten Verbs, und Beispiel (\ref{ex:konstituentenstellunginunabhaengigensaetzen028}) zeigt das Zurückbleiben einer Verbpartikel.

\begin{exe}
  \ex\label{ex:konstituentenstellunginunabhaengigensaetzen027}
  \begin{xlist}
    \ex{dass Ischariot dem Arzt das Bild verkauft hat}
    \ex{Ischariot hat dem Arzt das Bild verkauft.}
    \ex{Das Bild hat Ischariot dem Arzt verkauft.}
  \end{xlist}


  \ex\label{ex:konstituentenstellunginunabhaengigensaetzen028}
  \begin{xlist}
    \ex{dass der Arzt Ischariot das Bild gerne abkauft}
    \ex{Der Arzt kauft Ischariot das Bild gerne ab=.}
    \ex{Gerne kauft der Arzt Ischariot das Bild ab=.}
  \end{xlist}
\end{exe}

\index{Bewegung}

Innerhalb der Nebensatz-VP ist die Reihenfolge der Teilkonstituenten nicht eindeutig festgelegt.
Immerhin ist aber klar, dass alle durch Statusrektion verbundenen Verben einschließlich des finiten Verbs rechts stehen, und dass sämtliche anderen Konstituenten in einer Kette links davon positioniert werden.
Im unabhängigen Satz kommen nun die Schwierigkeiten hinzu, dass das finite Verb zwar eine festgelegte Stellung hat, dafür aber der Verbkomplex auseinandergerissen wird, und dass eine beliebige Konstituente außerhalb des VP-Zusammenhangs positioniert wird.
Trotz der Flexibilität der Konstituentenstellung (\textit{Scrambling}, s.\ Abschnitt~\ref{sec:verbphrase}) ist die Struktur des eingeleiteten Nebensatzes (VP innerhalb einer SP) also einfacher systematisch zu beschreiben als die Struktur des Hauptsatzes.
Deswegen beschreiben wir die Konstituentenstellung des Hauptsatzes als Abweichung von der des Nebensatzes, und zwar der Anschaulichkeit halber als Umstellungen bzw.\ \textit{Bewegungen}.

\begin{figure}[!htbp]
  \centering
  \begin{forest}
    [\textcolor{gray}{SP}, calign=first
      [\textcolor{gray}{\bf K}, tier=preterminal, edge={gray}
        [\textcolor{gray}{\it dass}, edge={gray}]
      ]
      l sep+=2em
      [VP, calign=last, edge={gray}
        [NP, tier=preterminal
          [\it Ischariot, narroof]
        ]
        [AvP, tier=preterminal
          [\it wahrscheinlich, narroof]
        ]
        [NP, tier=preterminal
          [\it dem Arzt, narroof]
        ]
        [NP, tier=preterminal
          [\it das Bild, narroof]
        ]
        [AvP, tier=preterminal
          [\it heimlich, narroof]
        ]
        [\bf V, calign=last
          [\bf V, tier=preterminal
            [\it verkauft]
          ]
          [\bf V, tier=preterminal
            [\it hat]
          ]
        ]
      ]
    ]
  \end{forest}
  \caption{VP mit dreistelliger Valenz und Angaben}
  \label{fig:konstituentenstellunginunabhaengigensaetzen029}
\end{figure}

Nehmen wir also an, wir hätten eine VP wie in Abbildung~\ref{fig:konstituentenstellunginunabhaengigensaetzen029} und sollten angeben, was sich im Vergleich zu dieser im unabhängigen Aussagesatz ändert.%
\footnote{Die einbettende SP wird hier nur zur Illustration eines typischen Kontexts für solche VPs gezeigt und ist daher in Grau gesetzt.}
Wir sprechen wie soeben erwähnt davon, dass Konstituenten \textit{bewegt} werden.
Einige Theorien wie \zB die \textit{Government and Binding Theory} (GB) oder das \textit{Minimalist Program} (MP) nehmen tatsächlich Bewegung im Sinne eines mehrstufigen Umbaus von Strukturen an.
Andere Theorien wie die \textit{Head-Driven Phrase Structure Grammar} (HPSG) modellieren dieselben Phänomene ohne solche Umbau-Operationen, erzielen aber denselben Effekt.%
\footnote{In den Literaturhinweisen finden sich Verweise auf Einführungen in diese Theorien.}
Aus unserer deskriptiven Sicht ist der Begriff der \textit{Bewegung} in jedem Fall als Hilfsvorstellung zu betrachten, und wir benutzen ihn, ohne theoretisch Partei nehmen zu wollen.
Es geht uns im Prinzip nur darum, die Strukturen im Haupt- und Nebensatz zueinander in Beziehung zu setzen.

Wir wissen, dass das finite Verb im Ergebnis an der zweiten Position im Satz stehen soll.
Statt in einem fertigen Satz oder einer fertigen VP die zweite Position zu suchen, gibt es eine einfachere Art, automatisch sicherzustellen, dass das finite Verb am Ende aller Umstellungen an zweiter Position steht und irgendeine andere Konstituente davor positioniert wird.
Man führt in dieser Reihenfolge die folgenden beiden Bewegungsoperationen an einer normalen VP durch:

\begin{enumerate}
  \item Bewege das finite Verb vor die VP.
  \item Bewege dann eine andere Konstituente vor das finite Verb.
\end{enumerate}

Aus Abbildung~\ref{fig:konstituentenstellunginunabhaengigensaetzen029} ergeben sich gemäß diesen Anweisungen die Möglichkeiten in den Abbildungen~\ref{fig:konstituentenstellunginunabhaengigensaetzen030}--\ref{fig:konstituentenstellunginunabhaengigensaetzen034}.
Ein unabhängiger Aussagesatz wird allein dadurch hergestellt, dass erst das finite Verb \textit{hat} und dann eine andere Konstituente \textit{Ischariot}, \textit{wahrscheinlich}, \textit{dem Arzt}, \textit{das Bild}, \textit{heimlich} nach links gestellt wurde.%
\footnote{Es besteht zusätzlich die Möglichkeit, den Rest des Verbalkomplexes \textit{verkauft} herauszustellen.}

\begin{figure}[!htbp]
  \centering
  \begin{forest}
    [, phantom, l sep+=2em
      [NP, tier=preterminal
        [\it Ischariot, narroof, name=BeweK]
      ]
      [\bf V\Sub{1}, tier=preterminal
        [\it hat, name=BeweV]
      ]
      [VP, calign=last, s sep+=0.5em
        [\Tii, tier=preterminal]
        {\draw[dotted, thick, ->] (.south) |- ++(0,-5.5em) -| (BeweK.south);}
        [AvP, tier=preterminal
          [\it wahrscheinlich, narroof]
        ]
        [NP, tier=preterminal
          [\it dem Arzt, narroof]
        ]
        [NP, tier=preterminal
          [\it das Bild, narroof]
        ]
        [AvP, tier=preterminal
          [\it heimlich, narroof]
        ]
        [\bf V, calign=last
          [\bf V, tier=preterminal
            [\it verkauft]
          ]
          [\Ti]
          {\draw[dotted, thick, ->] (.south) |- ++(0,-4.5em) -| (BeweV.south);}
        ]
      ]
    ]
  \end{forest}
  \caption{VP mit hinausbewegten Konstituenten}
  \label{fig:konstituentenstellunginunabhaengigensaetzen030}
\end{figure}

\begin{figure}[!htbp]
  \centering
  \begin{forest}
    [, phantom, l sep+=2em
      [AvP, tier=preterminal
        [\it wahrscheinlich, narroof, name=BeweK]
      ]
      [\bf V\Sub{1}, tier=preterminal
        [\it hat, name=BeweV]
      ]
      [VP, calign=last, s sep+=0.5em
        [NP, tier=preterminal
          [\it Ischariot, narroof]
        ]
        [\Tii, tier=preterminal]
        {\draw[dotted, thick, ->] (.south) |- ++(0,-5.5em) -| (BeweK.south);}
        [NP, tier=preterminal
          [\it dem Arzt, narroof]
        ]
        [NP, tier=preterminal
          [\it das Bild, narroof]
        ]
        [AvP, tier=preterminal
          [\it heimlich, narroof]
        ]
        [\bf V, calign=last
          [\bf V, tier=preterminal
            [\it verkauft]
          ]
          [\Ti]
          {\draw[dotted, thick, ->] (.south) |- ++(0,-4.5em) -| (BeweV.south);}
        ]
      ]
    ]
  \end{forest}
  \caption{VP mit hinausbewegten Konstituenten}
  \label{fig:konstituentenstellunginunabhaengigensaetzen031}
\end{figure}

\begin{figure}[!htbp]
  \centering
  \begin{forest}
    [, phantom, l sep+=2em
      [NP, tier=preterminal
        [\it dem Arzt, narroof, name=BeweK]
      ]
      [\bf V\Sub{1}, tier=preterminal
        [\it hat, name=BeweV]
      ]
      [VP, calign=last, s sep+=0.5em
        [NP, tier=preterminal
          [\it Ischariot, narroof]
        ]
        [AvP, tier=preterminal
          [\it wahrscheinlich, narroof]
        ]
        [\Tii, tier=preterminal]
        {\draw[dotted, thick, ->] (.south) |- ++(0,-5.5em) -| (BeweK.south);}
        [NP, tier=preterminal
          [\it das Bild, narroof]
        ]
        [AvP, tier=preterminal
          [\it heimlich, narroof]
        ]
        [\bf V, calign=last
          [\bf V, tier=preterminal
            [\it verkauft]
          ]
          [\Ti]
          {\draw[dotted, thick, ->] (.south) |- ++(0,-4.5em) -| (BeweV.south);}
        ]
      ]
    ]
  \end{forest}
  \caption{VP mit hinausbewegten Konstituenten}
  \label{fig:konstituentenstellunginunabhaengigensaetzen032}
\end{figure}

\begin{figure}[!htbp]
  \centering
  \begin{forest}
    [, phantom, l sep+=2em
      [NP, tier=preterminal
        [\it das Bild, narroof, name=BeweK]
      ]
      [\bf V\Sub{1}, tier=preterminal
        [\it hat, name=BeweV]
      ]
      [VP, calign=last, s sep+=0.5em
        [NP, tier=preterminal
          [\it Ischariot, narroof]
        ]
        [AvP, tier=preterminal
          [\it wahrscheinlich, narroof]
        ]
        [NP, tier=preterminal
          [\it dem Arzt, narroof]
        ]
        [\Tii, tier=preterminal]
        {\draw[dotted, thick, ->] (.south) |- ++(0,-5.5em) -| (BeweK.south);}
        [AvP, tier=preterminal
          [\it heimlich, narroof]
        ]
        [\bf V, calign=last
          [\bf V, tier=preterminal
            [\it verkauft]
          ]
          [\Ti]
          {\draw[dotted, thick, ->] (.south) |- ++(0,-4.5em) -| (BeweV.south);}
        ]
      ]
    ]
  \end{forest}
  \caption{VP mit hinausbewegten Konstituenten}
  \label{fig:konstituentenstellunginunabhaengigensaetzen033}
\end{figure}

\begin{figure}[!htbp]
  \centering
  \begin{forest}
    [, phantom, l sep+=2em
      [AvP, tier=preterminal
        [\it heimlich, narroof, name=BeweK]
      ]
      [\bf V\Sub{1}, tier=preterminal
        [\it hat, name=BeweV]
      ]
      [VP, calign=last, s sep+=0.5em
        [NP, tier=preterminal
          [\it Ischariot, narroof]
        ]
        [AvP, tier=preterminal
          [\it wahrscheinlich, narroof]
        ]
        [NP, tier=preterminal
          [\it dem Arzt, narroof]
        ]
        [NP, tier=preterminal
          [\it das Bild, narroof]
        ]
        [\Tii, tier=preterminal]
        {\draw[dotted, thick, ->] (.south) |- ++(0,-5.5em) -| (BeweK.south);}
        [\bf V, calign=last
          [\bf V, tier=preterminal
            [\it verkauft]
          ]
          [\Ti]
          {\draw[dotted, thick, ->] (.south) |- ++(0,-4.5em) -| (BeweV.south);}
        ]
      ]
    ]
  \end{forest}
  \caption{VP mit hinausbewegten Konstituenten}
  \label{fig:konstituentenstellunginunabhaengigensaetzen034}
\end{figure}

\index{Spur}

Wir verstehen die Stellung im Verb"=Zweit"=Satz (V2) also als das Ergebnis zweier Umstellungsoperationen bzw.\ Bewegungen.
Es bleibt eine VP mit zwei Lücken zurück, wobei diese Lücken in Theorien oft als \textit{Trace} (engl.\ für \textit{Spur}) bezeichnet und daher meist mit \textit{t} symbolisiert werden.
Wenn man die Lücken bzw.\ Spuren notiert und mit den dazugehörigen bewegten Konstituenten durchnumeriert, sind die Bewegungsoperationen auch am fertigen Umstellungsprodukt eindeutig nachvollziehbar.
Die gepunkteten Pfeile, die die Bewegung andeuten, sind dann im Prinzip nicht nötig und dienen hier nur der Verdeutlichung.

\subsection{Das Feldermodell}
\label{sec:dasfeldermodell}

Unser Ziel ist es nun, diese Strukturen möglichst auch mit Phrasenschemata zu beschreiben, denn in den Abbildungen~\ref{fig:konstituentenstellunginunabhaengigensaetzen030}--\ref{fig:konstituentenstellunginunabhaengigensaetzen034} stehen die bewegten Konstituenten jeweils vor der eigentlichen Struktur im syntaktischen Nichts.
Diese Beschreibung ist insofern problematisch, weil sie keine Baumstruktur ist (vgl.\ Abschnitt~\ref{sec:terminologiefuerbaumdiagramme}) und wir sie in unserem Strukturformat daher nicht formulieren können.
In Abschnitt~\ref{sec:schematafuersaetze} werden Satzschemata vorgestellt.

Vorher wird jetzt aber ein anderes Beschreibungsmodell eingeführt, das oft verwendet wird, um die Regularitäten der Konstituentenstellung im Deutschen zu verdeutlichen.
Dieses sogenannte \textit{Feldermodell} liefert eine einfache Terminologie zur Beschreibung der sich durch den Bau der VP und die gerade besprochenen Umsortierungen der Konstituenten im unabhängigen Aussagesatz ergebenden Varianten der Konstituentenstellung.
Das Modell bezieht sich dabei gemäß Satz~\ref{satz:felder} nicht auf Konstituentenstrukturen, sondern nur auf die lineare Abfolge der Satzteile.

\Satz{Feldermodell}{\label{satz:felder}%
Das Feldermodell ist ein Beschreibungsmodell, das ohne Bezug auf die Phrasenstruktur die lineare Abfolge von Satzteilen beschreibt.
\index{Feldermodell}
}


\index{Satzklammer}
\index{Vorfeld}
\index{Mittelfeld}
\index{Subjunktion}

Die erste wichtige Idee des Feldermodells ist es, dass der Verbkomplex in allen Arten von Sätzen wegen seiner Stellung am rechten Rand (der VP) eine gut erkennbare rechte Grenze, die \textit{rechte Satzklammer} (RSK), bildet.
Zusätzlich gibt es in allen Arten von (abhängigen und unabhängigen) Sätzen eine gut erkennbare linke Begrenzung.
Im eingeleiteten Nebensatz (SP) steht die Subjunktion ganz links, und keine Konstituente des Nebensatzes darf links davon stehen.
Im unabhängigen Aussagesatz (ohne Subjunktion) steht das finite Verb links an zweiter Stelle (in unserer Terminologie links von der VP).
Wegen ihrer markanten Position im linken Satzbereich werden die Subjunktion und das links stehende finite Verb im unabhängigen Aussagesatz in der Terminologie des Feldermodells die sogenannte \textit{linke Satzklammer} (LSK) genannt.

Anhand der beiden Satzklammern kann man dann den Rest des Satzes stellungsmäßig aufteilen:
Das \textit{Vorfeld} (Vf) ist der Bereich links von der linken Satzklammer.
Das \textit{Mittelfeld} (Mf) ist der Bereich zwischen den Satzklammern.
Für den durch eine Subjunktion eingeleiteten Nebensatz und den unabhängigen Aussagesatz ergeben sich also die Einteilungen in Felder wie in Abbildung~\ref{fig:dasfeldermodell035}.

\begin{figure}[!htbp]
  \resizebox{\textwidth}{!}{
    \begin{tabular}{lp{0.1em}cp{0.1em}cp{0.1em}cp{0.1em}c}
      \textbf{Satztyp} && \textbf{Vf} && \textbf{LSK} && \textbf{Mf} && \textbf{RSK} \\
      \cmidrule{1-1}\cmidrule{3-3}\cmidrule{5-5}\cmidrule{7-7}\cmidrule{9-9}
      \textbf{unabh.\ Aussagesatz} && \textit{das Bild} && \textit{hat} && \textit{Ischariot} \textit{wahrscheinlich} && \textit{verkauft} \\
      \textbf{eingel.\ Nebensatz} &&&& \textit{dass} && \textit{Ischariot das Bild wahrscheinlich} && \textit{verkauft hat} \\
    \end{tabular}
  }
  \caption{Felder im unabhängigen Aussagesatz und im Nebensatz}
  \label{fig:dasfeldermodell035}
\end{figure}

\index{Satz!Echofrage}
\index{Satz!Frage--}
\index{Satz!w-Frage--}

Das Feldermodell kann auch auf andere Satztypen angewendet werden.
Besonders sind hier der \textit{w}"=Fragesatz (\ref{ex:dasfeldermodell037}), der Entscheidungsfragesatz (\ref{ex:dasfeldermodell038}) und der Relativsatz (\ref{ex:dasfeldermodell039}) bzw.\ der eingebettete \textit{w}"=Fragesatz (\ref{ex:dasfeldermodell040}) zu behandeln.

\begin{exe}
  \ex\label{ex:dasfeldermodell036}
  \begin{xlist}
    \ex{\label{ex:dasfeldermodell037} Wem hat Ischariot das Bild verkauft?}
    \ex{\label{ex:dasfeldermodell038} Hat Ischariot das Bild verkauft?}
    \ex{\label{ex:dasfeldermodell039} (Das ist der Arzt,) dem Ischariot das Bild verkauft hat.}
    \ex{\label{ex:dasfeldermodell040} (Ischariot weiß,) wer die guten Bilder verkauft.}
  \end{xlist}
\end{exe}

\label{abs:dasfeldermodell041} Der \textit{w}"=Fragesatz stellt sich im Grunde wie ein unabhängiger Aussagesatz dar, wobei aber das Fragepronomen (\textit{w}"=Pronomen) oder eine größere Frage"=Konstituente (wie \textit{welchem dubiosen Arzt}) und nicht irgendeine frei wählbare Konstituente obligatorisch im Vorfeld steht.
Wenn die Frage"=Konstituente nicht im Vorfeld steht, ergibt sich eine sogenannte \textit{In-Situ-Frage} oder auch \textit{Echofrage} wie in (\ref{ex:dasfeldermodell043}).
Der zugehörige Aussagesatz wird in (\ref{ex:dasfeldermodell044}) zum Vergleich angegeben.

\begin{exe}
  \ex\label{ex:dasfeldermodell042}
  \begin{xlist}
    \ex{\label{ex:dasfeldermodell043} Ischariot hat wem das Bild verkauft?}
    \ex{\label{ex:dasfeldermodell044} Ischariot hat [dem dubiosen Arzt] das Bild verkauft.}
  \end{xlist}
\end{exe}

Bei einer solchen Frage bleibt das \textit{w}"=Pronomen an der Stelle, an der die korrespondierende Phrase im zugehörigen Aussagesatz stehen würde.
Ins Vorfeld wird dann in der In-Situ-Frage eine andere Konstituente gestellt (hier \zB \textit{Ischariot}).
Echofragen sind typisch in Kontexten, in denen der Fragende eine Verständnisfrage stellt, weil er die betreffende Konstituente akustisch nicht verstanden hat.

Falls mehrere \textit{w}"=Pronomina (oder komplexe Frage"=Konstituenten) im \textit{w}"=Fragesatz vorkommen, muss (In-Situ-Fragen ausgenommen) eines von diesen in das Vorfeld gestellt werden, die anderen verbleiben in der VP.
Dies ist in (\ref{ex:dasfeldermodell045}) dargestellt.


\begin{exe}
  \ex\label{ex:dasfeldermodell045}
  \begin{xlist}
    \ex{Wem hat Ischariot was wie verkauft?}
    \ex{Wie hat Ischariot wem was verkauft?}
    \ex{Was hat Ischariot wem wie verkauft?}
  \end{xlist}
\end{exe}

Die \textit{Entscheidungsfrage} in (\ref{ex:dasfeldermodell038}) ist nur teilweise dem unabhängigen Aussagesatz ähnlich.
Das finite Verb wird ebenfalls nach links bewegt, allerdings entfällt die Besetzung des Vorfelds, und die linke Satzklammer (in Form des finiten Verbs) bildet die linke Grenze des Satzes.

\index{Relativsatz}
\index{Satz!Frage--!eingebettet}

Der Relativsatz und der eingebettete \textit{w}"=Fragesatz werden hier gemeinsam behandelt.
Dabei wird exemplarisch nur der Relativsatz besprochen.
Der eingebettete \textit{w}"=Fragesatz ist dann strukturell völlig identisch.
Zur Verwendung des eingebetteten \textit{w}"=Fragesatzes s.\ Abschnitt~\ref{sec:ergaenzungssaetze}.
Ein Relativsatz wie in (\ref{ex:dasfeldermodell039}) ähnelt dem durch eine Subjunktion eingeleiteten Fragesatz insofern, als der Verbkomplex am rechten Rand intakt bleibt und das finite Verb nicht nach links bewegt wird.
Dafür wird das Relativpronomen (hier \textit{wem}) obligatorisch nach links bewegt und steht im Vorfeld.
Man kann nun die Satztypen wie in den Abbildungen~\ref{fig:dasfeldermodell046} bis~\ref{fig:dasfeldermodell049} zusammenfassen.

\begin{figure}[!htbp]
  \resizebox{\textwidth}{!}{
    \begin{tabular}{cp{0.1em}cp{0.1em}cp{0.1em}c}
      \textbf{Vf} && \textbf{LSK} && \textbf{Mf} && \textbf{RSK} \\
      \cmidrule{1-1}\cmidrule{3-3}\cmidrule{5-5}\cmidrule{7-7}
      eine Konstituente && finites Verb && (Rest) && infinite Verben \\
      \textit{das Bild} && \textit{hat} && \textit{Ischariot wahrscheinlich} && \textit{verkauft} \\
    \end{tabular}
  }
  \caption{Feldermodell: unabhängiger Aussagesatz (V2)}
  \label{fig:dasfeldermodell046}
\end{figure}

\begin{figure}[!htbp]
  \resizebox{\textwidth}{!}{
    \begin{tabular}{cp{0.1em}cp{0.1em}cp{0.1em}c}
      \textbf{Vf} && \textbf{LSK} && \textbf{Mf} && \textbf{RSK} \\
      \cmidrule{1-1}\cmidrule{3-3}\cmidrule{5-5}\cmidrule{7-7}
      (leer) && Subjunktion && (Rest) && Verbkomplex \\
      && \textit{dass} && \textit{Ischariot das Bild wahrscheinlich} && \textit{verkauft hat} \\
    \end{tabular}
  }
  \caption{Feldermodell: Nebensatz mit Subjunktion (VL)}
  \label{fig:dasfeldermodell047}
\end{figure}

\begin{figure}[!htbp]
    \begin{tabular}{cp{0.1em}cp{0.1em}cp{0.1em}c}
      \textbf{Vf} && \textbf{LSK} && \textbf{Mf} && \textbf{RSK} \\
      \cmidrule{1-1}\cmidrule{3-3}\cmidrule{5-5}\cmidrule{7-7}
      (leer) && finites Verb && (Rest) && infinite Verben \\
      && \textit{hat} && \textit{Ischariot das Bild} && \textit{verkauft} \\
    \end{tabular}
  \caption{Feldermodell: Entscheidungsfragesatz (V1)}
  \label{fig:dasfeldermodell048}
\end{figure}

\begin{figure}[!htbp]
  \resizebox{\textwidth}{!}{
    \begin{tabular}{cp{0.1em}cp{0.1em}cp{0.1em}c}
      \textbf{Vf} && \textbf{LSK} && \textbf{Mf} && \textbf{RSK} \\
      \cmidrule{1-1}\cmidrule{3-3}\cmidrule{5-5}\cmidrule{7-7}
      Relativpronomen && (leer) && (Rest) && Verbkomplex \\
      \textit{dem} &&&& \textit{Ischariot das Bild wahrscheinlich} && \textit{verkauft hat} \\
    \end{tabular}
  }
  \caption{Feldermodell: Relativsatz (VL)}
  \label{fig:dasfeldermodell049}
\end{figure}

\index{Satz!Verb-Erst--}
\index{Satz!Verb-Letzt--}
\index{Satz!Verb-Zweit--}

Die Satzarten werden nach der Stellung des finiten Verbs bezeichnet.
Man spricht daher vom \textit{Verb"=Erst"=Satz} oder \textit{V1-Satz} (Entscheidungsfragesatz) sowie vom \textit{Verb"=Zweit"=Satz} oder \textit{V2-Satz} (unabhängiger Aussagesatz und \textit{w}"=Fragesatz) und vom \textit{Verb"=Letzt"=Satz} oder \textit{VL-Satz} (eingeleiteter Nebensatz und Relativsatz).
All diese Bezeichnungen kategorisieren die Sätze nach der Art, wie die vier primären Positionen Vorfeld, linke Satzklammer, Mittelfeld und rechte Satzklammer gefüllt werden.
Neben diesen vier werden noch mindestens zwei weitere Felder angenommen.
Zunächst betrachten wir Sätze wie die in (\ref{ex:dasfeldermodell050}).

\begin{exe}
  \ex\label{ex:dasfeldermodell050}
  \begin{xlist}
    \ex{\label{ex:dasfeldermodell051} Ischariot hat dem Arzt das Bild verkauft, das er selber gemalt hatte.}
    \ex{\label{ex:dasfeldermodell052} Der Arzt hat Ischariot nicht geglaubt, dass das Bild echt war.}
  \end{xlist}
\end{exe}

\index{Relativsatz}
\index{Ergänzungssatz}

In diesen Sätzen stehen einmal ein Relativsatz (\ref{ex:dasfeldermodell051}) und einmal ein Ergänzungssatz (\ref{ex:dasfeldermodell052}) nach dem infiniten Verb.
Im Fall des Relativsatzes kann man besonders gut erkennen, dass dieser nach rechts bewegt wurde, denn die NP, zu der er strukturell gehört (\textit{das Bild}), befindet sich im Mittelfeld, und dadurch sind die NP und der zugehörige Relativsatz durch die rechte Satzklammer (\textit{verkauft}) voneinander getrennt.
Man geht im Falle solcher rechts von der rechten Satzklammer positionierten Konstituenten davon aus, dass sie wegen ihrer Länge aus dem Mittelfeld herausbewegt (\textit{rechtsversetzt}) werden.
Im Rahmen des Feldermodells nennt man die entsprechende Position das \textit{Nachfeld} (Nf).
Eine Analyse wird in Abbildung~\ref{fig:dasfeldermodell053} gegeben.

\index{Nachfeld}

\begin{figure}[!htbp]
  \centering
  \resizebox{\textwidth}{!}{
    \begin{tabular}{cp{0.1em}cp{0.1em}cp{0.1em}cp{0.1em}c}
      \textbf{Vf} && \textbf{LSK} && \textbf{Mf} && \textbf{RSK} && \textbf{Nf} \\
      \cmidrule{1-1}\cmidrule{3-3}\cmidrule{5-5}\cmidrule{7-7}\cmidrule{9-9}
      \textit{Ischariot} && \textit{hat} && \textit{dem Arzt das Bild} && \textit{verkauft} && \textit{das er selber gemalt hatte} \\
    \end{tabular}
  }
  \caption{Felderanalyse mit Nachfeld}
  \label{fig:dasfeldermodell053}
\end{figure}

Außerdem gibt es vermeintliche Subjunktionen wie \textit{denn}, die sich aber anders als echte Subjunktionen verhalten, vgl.\ (\ref{ex:dasfeldermodell054}).

\begin{exe}
  \ex\label{ex:dasfeldermodell054}
  \begin{xlist}
    \ex{\label{ex:dasfeldermodell055} Der Arzt ist froh, weil Ischariot ihm das Bild verkauft hat.}
    \ex{\label{ex:dasfeldermodell056} Der Arzt ist froh, denn Ischariot hat ihm das Bild verkauft.}
  \end{xlist}
\end{exe}

\index{Konnektor}
\index{Konnektorfeld}

Das Wort \textit{denn} kann gemäß unserer Wortklassifikation (s.\ Kapitel~\ref{sec:wortklassen}) als Partikel oder Konjunktion, aber auf keinen Fall als Subjunktion klassifiziert werden, denn es bettet keinen Nebensatz mit Verb-Letzt-Stellung ein.
Nach \textit{denn} steht ein Satz, der wie ein unabhängiger Aussagesatz (V2) strukturiert ist.
Solche Partikeln nennt man auch \textit{Konnektoren}, und man kann innerhalb des Feldermodells für sie ein \textit{Konnektorfeld} (Kf) oder \textit{Vor-Vorfeld} ansetzen, das noch vor dem Vorfeld positioniert ist.
Eine solche Analyse ist in Abbildung~\ref{fig:dasfeldermodell057} angegeben.

\begin{figure}[!htbp]
  \centering
  \begin{tabular}{cp{0.1em}cp{0.1em}cp{0.1em}cp{0.1em}c}
    \textbf{Kf} && \textbf{Vf} && \textbf{LSK} && \textbf{Mf} && \textbf{RSK} \\
    \cmidrule{1-1}\cmidrule{3-3}\cmidrule{5-5}\cmidrule{7-7}\cmidrule{9-9}
    \textit{denn} && \textit{Ischariot} && \textit{hat} && \textit{ihm das Bild} && \textit{verkauft} \\
  \end{tabular}
  \caption{Felderanalyse mit Konnektorfeld}
  \label{fig:dasfeldermodell057}
\end{figure}

Zwei typische Strukturen werden abschließend in Abbildung~\ref{fig:dasfeldermodell058} und Abbildung~\ref{fig:dasfeldermodell059} gezeigt.
Abbildung~\ref{fig:dasfeldermodell058} stellt einen kurzen Hauptsatz dar, in dem die meisten Felder leer bleiben.
Abbildung~\ref{fig:dasfeldermodell059} illustriert eine abgetrennte Verbpartikel, die in der rechten Satzklammer steht, während das finite Verb in der linken Satzklammer steht.

\begin{figure}[!htbp]
  \centering
  \begin{tabular}{cp{0.1em}cp{0.1em}cp{0.1em}cp{0.1em}cp{0.1em}c}
    \textbf{Kf} && \textbf{Vf} && \textbf{LSK} && \textbf{Mf} && \textbf{RSK} && \textbf{Nf} \\
    \cmidrule{1-1}\cmidrule{3-3}\cmidrule{5-5}\cmidrule{7-7}\cmidrule{9-9}\cmidrule{11-11}
    && \textit{Ischariot} && \textit{malt} &&&&&& \\
  \end{tabular}
  \caption{Felderanalyse eines V2-Satzes mit leeren Feldern}
  \label{fig:dasfeldermodell058}
\end{figure}

\begin{figure}[!htbp]
  \centering
  \begin{tabular}{cp{0.1em}cp{0.1em}cp{0.1em}cp{0.1em}cp{0.1em}c}
    \textbf{Kf} && \textbf{Vf} && \textbf{LSK} && \textbf{Mf} && \textbf{RSK} && \textbf{Nf} \\
    \cmidrule{1-1}\cmidrule{3-3}\cmidrule{5-5}\cmidrule{7-7}\cmidrule{9-9}\cmidrule{11-11}
    && \textit{Ischariot} && \textit{fährt} && \textit{den Pfosten} && \textit{um=} & \\
  \end{tabular}
  \caption{Felderanalyse eines V2-Satzes mit Verbpartikel}
  \label{fig:dasfeldermodell059}
\end{figure}

Abschließend fasst Tabelle~\ref{tab:dasfeldermodell060} die Konstituentenstellungen der primären Satztypen (ohne Konnektorfeld und Nachfeld) gemäß dem Feldermodell nochmals zusammen.
Die Tabelle beschreibt damit den Kern des Feldermodells.

\begin{table}
  \centering
  \resizebox{1\textwidth}{!}{
    \begin{tabular}{lp{0.3cm}llll}
    \lsptoprule
    \textbf{Satztyp} && \textbf{Vorfeld} & \textbf{LSK} & \textbf{Mittelfeld} & \textbf{RSK} \\
    \midrule
    \textbf{V2} && bel.\ Satzglied & finites Verb    & Rest der VP & infinite Verben \\
    \textbf{V1} && ---                  & finites Verb    & Rest der VP & infinite Verben \\
    \textbf{VL} && ---                  & Subjunktion & Rest der VP & Verbkomplex \\
    \lspbottomrule
  \end{tabular}
  }
  \caption{Besetzung der Felder in primären Satztypen laut Feldermodell}
  \label{tab:dasfeldermodell060}
\end{table}

\subsection{Eingebettete Nebensätze und der LSK-Test}
\label{sec:eingebettetenebensaetzeundderlsktest}

In Abschnitt~\ref{sec:konstituententests} wurde im Zusammenhang mit dem Vorfeldtest darauf verwiesen, dass es nicht immer trivial ist, die linke Satzklammer (und damit das Vorfeld) zu identifizieren.
Das Problem rührt daher, dass je nach Satzstruktur das erste finite Verb auch das Verb eines eingebetteten Nebensatzes sein kann, wenn dieser Nebensatz \zB im Vorfeld eines anderen Satzes steht.
Die Beispiele in (\ref{ex:eingebettetenebensaetzeundderlsktest061}) illustrieren das Problem.
In (\ref{ex:eingebettetenebensaetzeundderlsktest062}) sind sowohl \textit{glaubt} als auch \textit{haben} finit, in (\ref{ex:eingebettetenebensaetzeundderlsktest063}) kommt \textit{irrt} hinzu.

\begin{exe}
  \ex\label{ex:eingebettetenebensaetzeundderlsktest061}
  \begin{xlist}
    \ex{\label{ex:eingebettetenebensaetzeundderlsktest062} [Wer] glaubt, dass Tiere im Tierheim ein schönes Leben haben?}
    \ex{\label{ex:eingebettetenebensaetzeundderlsktest063} [Wer glaubt, dass Tiere im Tierheim ein schönes Leben haben], irrt.}
  \end{xlist}
\end{exe}

\index{Satz!w-Frage--}
Im Feldermodell müssen für solche Sätze eine Analyse der Struktur des Matrixsatzes und Analysen aller eingebetteten Nebensätze \textit{getrennt} durchgeführt werden.
Dazu muss zunächst die linke Satzklammer des Matrixsatzes gefunden werden.
Die hier jetzt vorgestellte Testprozedur zur Ermittlung der linken Satzklammer des Matrixsatzes macht es sich zunutze, dass in der Entscheidungsfrage immer das finite Verb des Matrixsatzes an erster Stelle steht.
Entweder ist ein Satz also bereits eine Entscheidungsfrage (oder \textit{w}"=Frage, s.\,u.) und wir müssen ihn nur noch als solche erkennen.
Oder der Satz muss in eine Entscheidungsfrage umformuliert werden.
In beiden Fällen steht dann das finite Verb des Matrixsatzes an erster Stelle.
Nehmen wir zunächst einen einfacheren Satz wie (\ref{ex:eingebettetenebensaetzeundderlsktest064}).

\begin{exe}
  \ex{\label{ex:eingebettetenebensaetzeundderlsktest064} Der Maler hat dem Arzt ein Bild geschenkt, das jetzt in der Praxis hängt.}
\end{exe}

Es kann entweder \textit{hat} oder \textit{hängt} das finite Verb des Matrixsatzes sein.
Da jede Satzstruktur (ob unabhängig oder abhängig) genau ein finites Verb enthält, zu dem alle weiteren Konstituenten dependent sind, muss das andere zu einem tiefer eingebetteten Nebensatz gehören.
Formuliert man (\ref{ex:eingebettetenebensaetzeundderlsktest064}) in eine Entscheidungsfrage um, erkennt man, dass \textit{hat} das finite Verb des unabhängigen Satzes sein muss, s.\ (\ref{ex:eingebettetenebensaetzeundderlsktest065}).

\begin{exe}
  \ex{\label{ex:eingebettetenebensaetzeundderlsktest065} Hat der Maler dem Arzt ein Bild geschenkt, das jetzt in der Praxis hängt?}
\end{exe}

In der Umformung steht \textit{hat} am Satzanfang, und es kann daher geschlossen werden, dass es in (\ref{ex:eingebettetenebensaetzeundderlsktest064}) die linke Satzklammer besetzt.
Dass man mit dem Test die richtige Frage produziert hat, erkennt man daran, dass der ursprüngliche Satz mit vorangestelltem \textit{Ja} eine adäquate (wenn auch umständliche) positive Antwort wäre, hier also (\ref{ex:eingebettetenebensaetzeundderlsktest066}).

\begin{exe}
  \ex{\label{ex:eingebettetenebensaetzeundderlsktest066} Ja, der Maler hat dem Arzt ein Bild geschenkt, das jetzt in der Praxis hängt.}
\end{exe}

Wenn wir nun auf (\ref{ex:eingebettetenebensaetzeundderlsktest061}) zurückkommen, gilt es zunächst zu beachten, dass (\ref{ex:eingebettetenebensaetzeundderlsktest062}) bereits eine \textit{w}"=Frage ist.
Insofern ist \textit{glaubt} prinzipiell ohne Umstellung als finites Verb (LSK) zu identifizieren, denn im \textit{w}"=Fragesatz ist das Vorfeld immer mit dem \textit{w}"=Pro\-no\-men (hier \textit{wer}) besetzt.
Die Umformung in eine Entscheidungsfrage ergibt das gleiche Ergebnis, wobei allerdings ein Pronomen ausgetauscht werden muss, nämlich \textit{wer} zu einem Pronomen wie \textit{irgendjemand}, s.\ (\ref{ex:eingebettetenebensaetzeundderlsktest067}).

\begin{exe}
  \ex{\label{ex:eingebettetenebensaetzeundderlsktest067} Glaubt irgendjemand, dass Tiere im Tierheim ein schönes Leben haben?}
\end{exe}

Satz (\ref{ex:eingebettetenebensaetzeundderlsktest063}) ist insofern schwierig, als im Vorfeld ein sogenannter \textit{freier Relativsatz} [\textit{wer \ldots\ haben}] (vgl.\ Abschnitt~\ref{sec:relativsaetze}) steht, der wiederum einen Ergänzungssatz [\textit{dass \ldots\ haben}] (vgl.\ Abschnitt~\ref{sec:ergaenzungssaetze}) enthält.
Das finite Verb des Matrixsatzes ist dadurch das insgesamt dritte, nämlich \textit{irrt}.
Bei der Umformung in eine Entscheidungsfrage müssen nun Pronomina ausgetauscht und hinzugefügt werden, um den Satz völlig akzeptabel zu machen.
Die einfache Umstellung (ohne Austausch und Ergänzung von Pronomina), die bezüglich ihrer Grammatikalität etwas fragwürdig ist, findet sich in (\ref{ex:eingebettetenebensaetzeundderlsktest069}), die völlig akzeptable Version (mit Austausch\slash Ergänzung von Pronomina) in (\ref{ex:eingebettetenebensaetzeundderlsktest070}).
Mit dieser Umformung in eine Entscheidungsfrage ist also auch hier das richtige finite Verb zu identifizieren.

\begin{exe}
  \ex\label{ex:eingebettetenebensaetzeundderlsktest068}
  \begin{xlist}
    \ex{\label{ex:eingebettetenebensaetzeundderlsktest069} Irrt, wer glaubt, dass Tiere im Tierheim ein schönes Leben haben?}
    \ex{\label{ex:eingebettetenebensaetzeundderlsktest070} Irrt derjenige, der glaubt, dass Tiere im Tierheim ein schönes Leben haben?}
  \end{xlist}
\end{exe}

Wie bereits erwähnt, muss für den unabhängigen Satz (Matrixsatz) und die abhängigen Sätze (Nebensätze) je eine Felderanalyse durchgeführt werden.
Im Fall von Nebensätzen ist sozusagen eine vollständige Felderstruktur in eine andere eingebettet.
Für (\ref{ex:eingebettetenebensaetzeundderlsktest062}) sieht das aus wie in den Abbildung~\ref{fig:eingebettetenebensaetzeundderlsktest071} (Matrixsatz) und Abbildung~\ref{fig:eingebettetenebensaetzeundderlsktest072} (Nebensatz).
Für (\ref{ex:eingebettetenebensaetzeundderlsktest063}) erhalten wir Abbildung~\ref{fig:eingebettetenebensaetzeundderlsktest073} (Matrixsatz) und Abbildung~\ref{fig:eingebettetenebensaetzeundderlsktest074} (Nebensatz).

\begin{figure}[!htbp]
  \centering
  \resizebox{\textwidth}{!}{
  \begin{tabular}{cp{0.1em}cp{0.1em}cp{0.1em}cp{0.1em}c}
    \textbf{Vf} && \textbf{LSK} && \textbf{Mf} && \textbf{RSK} && \textbf{Nf} \\
    \cmidrule{1-1}\cmidrule{3-3}\cmidrule{5-5}\cmidrule{7-7}\cmidrule{9-9}
    \textit{Wer} && \textit{glaubt} && && && \textit{dass Tiere im Tierheim ein schönes Leben haben} \\
  \end{tabular}
  }
  \caption{Felderanalyse eines V2-Satzes mit Nebensatz}
  \label{fig:eingebettetenebensaetzeundderlsktest071}
\end{figure}

\begin{figure}[!htbp]
  \centering
  \begin{tabular}{cp{0.1em}cp{0.1em}cp{0.1em}c}
    \textbf{Vf} && \textbf{LSK} && \textbf{Mf} && \textbf{RSK} \\
    \cmidrule{1-1}\cmidrule{3-3}\cmidrule{5-5}\cmidrule{7-7}
    && \textit{dass} && \textit{Tiere im Tierheim ein schönes Leben} && \textit{haben} \\
  \end{tabular}
  \caption{Felderanalyse für den Nebensatz aus Abbildung~\ref{fig:eingebettetenebensaetzeundderlsktest071}}
  \label{fig:eingebettetenebensaetzeundderlsktest072}
\end{figure}

\begin{figure}[!htbp]
  \centering
  \resizebox{\textwidth}{!}{
    \begin{tabular}{cp{0.1em}cp{0.1em}cp{0.1em}c}
      \textbf{Vf} && \textbf{LSK} && \textbf{Mf} && \textbf{RSK} \\
      \cmidrule{1-1}\cmidrule{3-3}\cmidrule{5-5}\cmidrule{7-7}
      \textit{Wer glaubt, dass Tiere im Tierheim ein schönes Leben haben} && \textit{irrt} &&&& \\
    \end{tabular}
  }
  \caption{Felderanalyse eines V2-Satzes mit komplexem Vorfeld}
  \label{fig:eingebettetenebensaetzeundderlsktest073}
\end{figure}

\begin{figure}[!htbp]
  \centering
  \resizebox{\textwidth}{!}{
    \begin{tabular}{cp{0.1em}cp{0.1em}cp{0.1em}cp{0.1em}c}
      \textbf{Vf} && \textbf{LSK} && \textbf{Mf} && \textbf{RSK} && \textbf{Nf} \\
      \cmidrule{1-1}\cmidrule{3-3}\cmidrule{5-5}\cmidrule{7-7}\cmidrule{9-9}
      \textit{wer} &&&&&& \textit{glaubt} && \textit{dass Tiere im Tierheim ein schönes Leben haben} \\
    \end{tabular}
  }
  \caption{Felderanalyse des komplexen Vorfelds aus Abbildung~\ref{fig:eingebettetenebensaetzeundderlsktest073}}
  \label{fig:eingebettetenebensaetzeundderlsktest074}
\end{figure}


\Zusammenfassung{%
Im unabhängigen Aussagesatz steht das finite Verb nicht im Verbkomplex, sondern an zweiter Stelle nach einer (fast) beliebig wählbaren anderen Konstituente (V2-Satz).
Ein unabhängiger Aussagesatz kann als VP betrachtet werden, aus dem zuerst das finite Verb und dann eine andere Konstituente nach links hinausbewegt wurde.
Das Feldermodell bietet für diese und andere Satzstrukturen eine Oberflächenbeschreibung an, die mit unserer phrasenstrukturellen Darstellung aber nicht direkt kompatibel ist.
In Entscheidungsfragesätzen steht das finite Verb an erster Stelle (V1-Satz).
}

\section{Schemata für Sätze}
\label{sec:schematafuersaetze}

\subsection{Verb-Zweit-Sätze}
\label{sec:verbzweitsaetze}

Der Bau der Phrasen (Kapitel~\ref{sec:phrasen}) ist geprägt von einer reichen internen Struktur und von Valenz- und Rektionsbeziehungen.
Das Feldermodell hingegen ist ein von diesem Phrasenbau unabhängiges reines Linearisierungsmodell, also eine Beschreibung der Abfolge von Satzteilen, ohne dass deren Struktur weiter betrachtet wird.
Das ist der Grund, warum das Feldermodell die üblicherweise angenommene Konstituentenstruktur nicht direkt nachbilden kann.
Die Beziehung zwischen Feldermodell und Konstituentenstruktur wird daher jetzt verdeutlicht.
Dabei muss berücksichtigt werden, dass die beiden Beschreibungsmodelle (Feldermodell und Phrasenstruktur) zwar denselben Gegenstand beschreiben (deutsche Satzsyntax), aber eigentlich konkurrierende und nicht vereinbare Modelle darstellen.
Beide sind ausgesprochen populär, und man kann sie (wie es hier geschieht) miteinander vergleichen.
Jedoch haben in einem Phrasenstrukturbaum Bezeichnungen von Feldern nichts verloren, genauso wie in einer Felderanalyse Bezeichnungen von Phrasen nichts verloren haben.

Beginnen wir damit, unter einer Konstituentenanalyse eines V2-Satzes (inklusive Bewegung) zu notieren, welche Felder den Konstituenten entsprechen.
In Abbildung~\ref{fig:verbzweitsaetze075} geschieht dies durch die Boxen mit den Namen der Felder unter dem Baum mit den herausbewegten Konstituenten (vgl.\ schon die Abbildungen~\ref{fig:konstituentenstellunginunabhaengigensaetzen030}--\ref{fig:konstituentenstellunginunabhaengigensaetzen034}).
Offensichtlich können bestimmte Knoten im Strukturbaum der VP und die herausgestellten Konstituenten bestimmten Feldern des Feldermodells zugeordnet werden.
Das Vorfeld und die linke Satzklammer entsprechen den herausbewegten Konstituenten, das Mittelfeld entspricht der VP ohne Verbkomplex, und die rechte Satzklammer entspricht dem Verbkomplex.
Weil der Rest-Verbkomplex aber eben eine Teilkonstituente der VP ist, können wir das Feldermodell phrasenstrukturell nicht genau nachbilden.
Sobald wir sagen, \textit{die VP entspreche dem Mittelfeld}, machen wir den Verbkomplex zum Teil des Mittelfelds, obwohl er eigentlich ein eigenes Feld bildet.
Eine hierarchische Phrasenstruktur und das Feldermodell passen nicht wirklich zueinander, und wir versuchen daher jetzt ein rein phrasenstrukturelles Modell des unabhängigen Aussagesatzes zu erarbeiten, das die Grammatik aus Kapitel~\ref{sec:konstituentenstruktur} und Kapitel~\ref{sec:phrasen} vervollständigt.

\begin{figure}[!htbp]
  \centering
  \begin{forest}
    [, phantom, l sep+=2em
      [AvP\Sub{2}, tier=preterminal, name=Vfnode
        [\it wahrscheinlich, narroof, name=Vfterm]
      ]
      [\bf V\Sub{1}, tier=preterminal
        [\it hat, name=Lskterm]
      ]
      [VP, calign=last, s sep+=0.5em
        [NP, tier=preterminal
          [\it Ischariot, narroof, name=Mffirstterm]
        ]
        [\Tii, tier=preterminal]
        [NP, tier=preterminal
          [\it dem Arzt, narroof]
        ]
        [NP, tier=preterminal
          [\it das Bild, narroof]
        ]
        [AvP, tier=preterminal
          [\it heimlich, narroof, name=Mflastterm]
        ]
        [\bf V, calign=last, name=Rsknode
          [\bf V, tier=preterminal
            [\it verkauft, name=Rskfirstterm]
          ]
          [\Ti, tier=preterminal, name=Rsklastterm]
        ]
      ]
      {\draw ($(Vfterm.west |- Vfnode.north) + (-0.2,0.3)$) -- ($(Vfterm.east |- Vfnode.north) + (0.15,0.3)$) -- ($(Vfterm.south east) + (0.15,0)$) -- node [midway, below] {Vf} ($(Vfterm.south west) + (-0.2,0)$) -- cycle;}
      {\draw ($(Lskterm.west |- Vfnode.north) + (0.05,0.3)$) -- ($(Lskterm.east |- Vfnode.north) + (-0.05,0.3)$) -- ($(Lskterm.east |- Vfterm.south) + (-0.05,0)$) -- node [midway, below] {LSK} ($(Lskterm.west |- Vfterm.south) + (0.05,0)$) -- cycle;}
      {\draw ($(Mffirstterm.west |- Vfnode.north) + (-0.15,0.3)$) -- ($(Mflastterm.east |- Vfnode.north) + (0.25,0.3)$) -- ($(Mflastterm.east |- Vfterm.south) + (0.25,0)$) -- node [midway, below] {Mf} ($(Mffirstterm.west |- Vfterm.south) + (-0.15,0)$) -- cycle;}
      {\draw ($(Rskfirstterm.west |- Rsknode.north) + (-0.025,0.3)$) -- ($(Rsklastterm.east |- Rsknode.north) + (0,0.3)$) -- ($(Rsklastterm.east |- Vfterm.south) + (0,0)$) -- node [midway, below] {RSK} ($(Rskfirstterm.west |- Vfterm.south) + (-0.025,0)$) -- cycle;}
    ]
  \end{forest}
  \caption{Zuordnung der Felder zu Konstituenten (V2)}
  \label{fig:verbzweitsaetze075}
\end{figure}

\begin{figure}[!htbp]
  \centering
  \begin{forest}
    [S, calign=child, calign child=2
      [NP\Sub{2}, tier=preterminal
        [\it das Bild, narroof, name=BeweBild]
      ]
      [\bf V\Sub{1}, tier=preterminal
        [\it hat, name=BeweHat]
      ]
      [VP, calign=last
        [NP, tier=preterminal
          [\it Ischariot, narroof]
        ]
        [AvP, tier=preterminal
          [\it wahrscheinlich, narroof]
        ]
        [t\Sub{2}, tier=preterminal
        ]
        {\draw[dotted, thick, ->] (.south) |- ++(0,-5em) -| (BeweBild.south);}
        [\bf V, calign=last
          [\bf V, tier=preterminal
            [\it verkauft]
          ]
          [t\Sub{1}]
          {\draw[dotted, thick, ->] (.south) |- ++(0,-4em) -| (BeweHat.south);}
        ]
      ]
    ]
  \end{forest}
  \caption{V2-Satz}
  \label{fig:verbzweitsaetze076}
\end{figure}

\index{Satz!Verb-Zweit--}
\index{Bewegung}
\index{Spur}

Die angestrebte Analyse der Konstituentenstruktur eines V2-Satzes sieht aus wie in Abbildung~\ref{fig:verbzweitsaetze076}.
Ein unabhängiger Aussagesatz (Symbol S) wird hier als eine zusammenhängende Konstituente analysiert.
Die Konstituenten, die man im Feldermodell dem Mittelfeld und der rechten Satzklammer zuordnet, entsprechen den Resten der VP und des Verbkomplexes.
Die Bewegung des finiten Verbs in die zweite Position in S entspricht der Besetzung der linken Satzklammer.
Die Bewegung einer beliebigen Phrase (wobei für \textit{beliebige Phrase} üblicherweise XP geschrieben wird) in die linke Position von S entspricht der Besetzung des Vorfelds.
Das Schema, das diese Konstituentenstruktur erzeugt, muss die Anforderungen kodieren, dass eine VP mit zwei Spuren (der Spur des finiten Verbs und der des Vorfeldbesetzers) sich mit den Konstituenten verbindet, die in einer Nebensatz-VP an der Stelle der Spuren stünden.


\Phrasenschema{V2-Satz (Aussage- und w"=Fragesätze)}{\label{str:v2}
  \centering
  \begin{forest}
    [S, calign=child, calign child=2, Ephr
      [XP\Sub{2}, Eobl]
      [V\UpSub{finit}{1}, Eobl]
      [VP\\{[\ldots\Tii\ldots\Ti]}, Eobl]
    ]
  \end{forest}
}

Im Schema~\ref{str:v2} wird die Notation VP[~\ldots~\Tii~\ldots~\Ti~] verwendet, um anzuzeigen, dass eine VP mit zwei Spuren eingesetzt werden muss, egal was die VP sonst noch enthält.%
\footnote{Die Abkürzung \ldots\ zeigt an, dass an ihrer Stelle beliebiges Material stehen kann, aber nicht muss.}
Die aus den Positionen der Spuren herausbewegten Konstituenten werden vorne in die S-Struktur eingefügt.
Über die Konstituente, die zu Spur \Ti\ gehört, wird gesagt, dass sie ein finites Verb sein muss, was wir mit V\Up{finit} abkürzen.
Das Feldermodell kann also vollständig durch eine sehr einfache phrasenstrukturelle Analyse ersetzt werden.

Abschließend sei angemerkt, dass nicht immer davon ausgegangen wird, dass alle Vorfeldbesetzer aus dem Mittelfeld herausbewegt werden.
Angaben wie \textit{erfreulicherweise} \zB könnten auch ohne Weiteres direkt in S eingefügt werden, sofern die VP nur die Spur \Ti\ mit dem finiten Verb enthält.
Das sähe dann so aus wie in Abbildung~\ref{fig:verbzweitsaetze077}.%
\footnote{Das Schema für S müsste angepasst werden, um auch diesen Fall zu beschreiben.}
Wir gehen hier in den besprochenen Sätzen immer von Bewegung aus, nicht ohne darauf hinzuweisen, dass dies eine zu starke Simplifizierung sein könnte.

\begin{figure}[!htbp]
  \centering
  \begin{forest}
    [S, calign=child, calign child=2
      [AvP, tier=preterminal
        [\it erfreulicherweise, narroof]
      ]
      [\bf V\Sub{1}, tier=preterminal
        [\it hat, name=BeweHat]
      ]
      [VP, calign=last
        [NP, tier=preterminal
          [\it Ischariot, narroof]
        ]
        [NP, tier=preterminal
          [\it das Bild, narroof]
        ]
        [\bf V, calign=last
          [\bf V, tier=preterminal
            [\it verkauft]
          ]
          [t\Sub{1}]
          {\draw[dotted, thick, ->] (.south) |- ++(0,-4em) -| (BeweHat.south);}
        ]
      ]
    ]
  \end{forest}
  \caption{Konstituentenanalyse bei direkter Vorfeldbesetzung}
  \label{fig:verbzweitsaetze077}
\end{figure}

Damit haben wir ein Modell der Konstituentenstellung im eingeleiteten Nebensatz (normale SP, vgl.\ Abschnitt~\ref{sec:subjunktionsphrase}) und im V2-Satz (unabhängiger Aussagesatz, Schema~\ref{str:v2}).
Der \textit{w}"=Fragesatz benötigt kein eigenes Schema, denn er ist lediglich eine Variante des V2-Aussagesatzes.
Die vor das finite Verb bewegte Konstituente muss dabei immer eine Phrase mit einem \textit{w}"=Pronomen sein.
Eine Analyse zeigt Abbildung~\ref{fig:verbzweitsaetze078}.

\begin{figure}[!htbp]
  \centering
  \begin{forest}
    [S, calign=child, calign child=2
      [NP\Sub{2}, tier=preterminal
        [\it was, narroof, name=BeweBild]
      ]
      [\bf V\Sub{1}, tier=preterminal
        [\it hat, name=BeweHat]
      ]
      [VP, calign=last
        [NP, tier=preterminal
          [\it Ischariot, narroof]
        ]
        [AvP, tier=preterminal
          [\it wahrscheinlich, narroof]
        ]
        [t\Sub{2}, tier=preterminal
        ]
        {\draw[dotted, thick, ->] (.south) |- ++(0,-5em) -| (BeweBild.south);}
        [\bf V, calign=last
          [\bf V, tier=preterminal
            [\it verkauft]
          ]
          [t\Sub{1}]
          {\draw[dotted, thick, ->] (.south) |- ++(0,-4em) -| (BeweHat.south);}
        ]
      ]
    ]
  \end{forest}
  \caption{V2-\textit{w}"=Fragesatz}
  \label{fig:verbzweitsaetze078}
\end{figure}


\subsection{Verb-Erst-Sätze}
\label{sec:verberstsaetze}

\index{Satz!Verb-Erst--}
\index{Satz!Entscheidungsfrage--}

\begin{sloppypar}
V1-Fragesätze (FS) sind denkbar einfach zu beschreiben, nachdem wir bereits V2-Sätze analysiert haben.
Sätze wie (\ref{ex:verberstsaetze079}) werden durch Schema~\ref{str:v1} abgebildet, s.\ Abbildung~\ref{fig:verberstsaetze080}.
\end{sloppypar}

\begin{exe}
  \ex{\label{ex:verberstsaetze079} Hat Ischariot tatsächlich das Bild verkauft?}
\end{exe}

\Phrasenschema{V1-Satz (Fragesatz)}{\label{str:v1}
  \centering
  \begin{forest}
    [FS, calign=first, Ephr
      [V\UpSub{finit}{1}, Eobl]
      [VP\\{[\ldots\Ti]}, Eobl]
    ]
  \end{forest}
}

\begin{figure}[!htbp]
  \centering
  \begin{forest}
    [FS, calign=first
      [\bf V\Sub{1}, tier=preterminal
        [\it hat, name=BeweHat]
      ]
      [VP, calign=last
        [NP, tier=preterminal
          [\it Ischariot, narroof]
        ]
        [AvP, tier=preterminal
          [\it wahrscheinlich, narroof]
        ]
        [NP, tier=preterminal
          [\it das Bild, narroof]
        ]
        [\bf V, calign=last
          [\bf V, tier=preterminal
            [\it verkauft]
          ]
          [t\Sub{1}]
          {\draw[dotted, thick, ->] (.south) |- ++(0,-4em) -| (BeweHat.south);}
        ]
      ]
    ]
  \end{forest}
  \caption{Entscheidungsfragesatz}
  \label{fig:verberstsaetze080}
\end{figure}

Es entfällt bei dem V1-Satz lediglich die Bewegung der zweiten Konstituente nach der Bewegung des finiten Verbs.
Nur das finite Verb muss nach links gestellt werden, und das Schema ist damit einfacher als das V2-Schema.
Es bleibt anzumerken, dass wir hier die Bezeichnung FS mehr oder weniger informell benutzen.
Mit der Beschriftung FS wird die Information kodiert, dass es sich um einen Fragesatz handelt.%
\footnote{Man könnte den Status des Satzes als Frage- oder Aussagesatz alternativ (und besser) mittels Merkmalen abbilden.
Insofern ist die Bezeichnung FS nur eine Abkürzung für eine bestimmte Merkmalskonfiguration.}

\index{Imperativ!Satz}
Imperativsätze wie (\ref{ex:verberstsaetze081}) sind im Prinzip wie V1-Sätze strukturiert.

\begin{exe}
  \ex{\label{ex:verberstsaetze081} Verkauf das Bild.}
\end{exe}

Es sei allerdings darauf hingewiesen, dass wir in Abschnitt~\ref{sec:formendesimperativs} morphologisch argumentiert haben, dass imperativische Verbformen nicht finit sind.
Wenn man dies annimmt, wird in Imperativsätzen eine infinite Verbform herausbewegt.
Das müsste in einem Schema für den Imperativsatz ausbuchstabiert werden, was hier aus Platzgründen nicht erfolgt.
Alternativ könnten wir Finitheit so definieren, dass Imperative zu den finiten Verbformen zählen und eine angepasste Version von Schema~\ref{str:v1} verwenden.
Beide Lösungen haben Vor- und Nachteile, und wir legen uns nicht auf eine fest.

\index{Verb!Partikel--}

Damit sind alle Stellungstypen prinzipiell erklärt.
Zu Nebensätzen kann und sollte man allerdings mehr sagen, als einfach ihre Konstituentenstruktur anzugeben.
Über Verwendung, Anschluss und Stellung der drei wichtigen Nebensatztypen folgen (nach Bemerkungen zu Partikelverben in Abschnitt~\ref{sec:syntaxderpartikelverben} und zu Kopulasätzen in Abschnitt~\ref{sec:kopulasaetze}) weitere Überlegungen in Abschnitt~\ref{sec:nebensaetze}.

\subsection{Syntax der Partikelverben}
\label{sec:syntaxderpartikelverben}

Durch die Bewegung von finiten Verben ergibt sich eine Besonderheit, wenn wir Partikelverben als eine Wortform analysieren.
In einer V2-Struktur bleibt die Partikel zurück, und die Bewegung müsste aus einer Wortform heraus geschehen, vgl. (\ref{ex:syntaxderpartikelverben082}).

\begin{exe}
  \ex{\label{ex:syntaxderpartikelverben082} Sarah isst den Kuchen alleine auf=.}
\end{exe}

\index{Verbkomplex}

Das syntaktische Herausbewegen aus einer Wortform ist problematisch, denn Wortformen sollen auf der Ebene der Syntax als atomare Konstituenten gelten.
Die Lösung besteht darin, Kombinationen aus Partikel und Verb als syntaktische Struktur zu analysieren wie in Abbildung~\ref{fig:syntaxderpartikelverben083}.
Damit ist es möglich, die Bewegung des finiten Verbs durchzuführen.
Eigentlich müsste das Phrasenschema für den Verbkomplex für diesen Zweck erweitert werden.
Außerdem würden sich evtl.\ Änderungen an den Wortklassen bzw.\ den Aussagen zur Verbalmorphologie (\zB Bildung der Partizipien) ergeben.

\begin{figure}[!htbp]
  \centering
  \begin{forest}
    [S, calign=child, calign child=2
      [NP\Sub{2}, tier=preterminal
        [\it Sarah, narroof, name=BeweSarah]
      ]
      [\bf V\Sub{1}, tier=preterminal
        [\it isst, name=BeweIsst]
      ]
      [VP, calign=last
        [\Tii, tier=preterminal]
        {\draw[dotted, thick, ->] (.south) |- ++(0,-5em) -| (BeweSarah.south);}
        [NP, tier=preterminal
          [\it den Kuchen, narroof]
        ]
        [AvP, tier=preterminal
          [\it alleine, narroof]
        ]
        [\bf V, calign=last
          [Ptkl, tier=preterminal
            [\it auf{=}]
          ]
          [\Ti, tier=preterminal]
          {\draw[dotted, thick, ->] (.south) |- ++(0,-4em) -| (BeweIsst.south);}
        ]
      ]
    ]
  \end{forest}
  \caption{V2-Satz mit Partikelverb}
  \label{fig:syntaxderpartikelverben083}
\end{figure}

\subsection{Kopulasätze}
\label{sec:kopulasaetze}

\index{Kopula}
\index{Kopula!Satz}

Für die Beschreibung von Kopulasätzen wie (\ref{ex:kopulasaetze085}) müssen keine besonderen Satzstrukturen eingeführt werden.
Wir können sie als Ergebnis der üblichen Bewegungsoperationen betrachten und Strukturen wie (\ref{ex:kopulasaetze086}) zugrundelegen.

\begin{exe}
  \ex\label{ex:kopulasaetze084}
  \begin{xlist}
    \ex{\label{ex:kopulasaetze085} Die Frau ist stolz auf ihre Tochter.}
    \ex{\label{ex:kopulasaetze086} dass die Frau auf ihre Tochter stolz ist}
  \end{xlist}
\end{exe}

Die AP ist hier strukturell etwas anders gebaut als eine attributive AP innerhalb einer NP.
Die in der NP prototypische Abfolge [[\textit{auf ihre Tochter}] \textit{stolze}] wird (zumindest optional) umgekehrt zu [\textit{stolz} [\textit{auf ihre Tochter}]].
Außerdem besteht keine Kongruenz des Adjektivs zu irgendeinem Bezugsnomen, und das Adjektiv steht in der unflektierten Kurzform (s.\ Abschnitt~\ref{sec:klassifikation}).
Ansonsten fällt auf, dass die Nominativ-NP \textit{die Frau} mit der Kopula in Person und Numerus kongruiert und frei im Satz bewegt werden kann.
Eine besondere morphosyntaktische Beziehung zum Adjektiv hat das Subjekt nicht.
Die Konstituente [\textit{stolz auf ihre Tochter}] kann außerdem auch frei bewegt werden wie in (\ref{ex:kopulasaetze087}).

\begin{exe}
  \ex{\label{ex:kopulasaetze087} [Stolz auf ihre Tochter] ist die Frau.}
\end{exe}

Die Analyse in Abbildung~\ref{fig:kopulasaetze088} bietet sich daher an.
Dabei regiert die Kopula eine AP, die zwar eine andere Abfolge ihrer Konstituenten realisiert als die AP innerhalb einer NP, die aber aus denselben Konstituenten besteht.
Die Kopula regiert außerdem eine NP im Nominativ.

\begin{figure}[!htbp]
  \centering
  \begin{forest}
    s sep=3em
    [S, calign=child, calign child=2
      [NP\Sub{2}, tier=preterminal
        [\it die Frau, narroof, name=BeweDiefrau]
      ]
      [\bf V\Sub{1}, tier=preterminal
        [\it ist]
      ]
      [VP, calign=last, s sep=2em
        [\Tii]
        {\draw[dotted, thick, ->] (.south) |- ++(0,-9em) -| (BeweDiefrau.south);}
        [AP, calign=first
          [\bf A, tier=preterminal
            [\it stolz]
          ]
          [PP, tier=preterminal
            [\it auf ihre Tochter, narroof]
          ]
        ]
        [\Ti, fit=band]
        {\draw[dotted, thick, ->] (.south) |- ++(0,-8em) -| (BeweIsst.south);}
      ]
    ]
  \end{forest}
  \caption{Analyse eines Kopulasatzes mit AP}
  \label{fig:kopulasaetze088}
\end{figure}

\Zusammenfassung{%
Wir stellen Sätze durch Phrasenschemata dar.
Sätze haben in der hier vertretenen Analyse aber keinen Kopf.
Unabhängige Aussagesätze werden als Umstellung einer VP (wie sie im eingeleiteten Nebensatz vorkommt) abgebildet.
Zuerst wird das finite Verb ganz nach links herausgestellt, dann eine beliebige Konstituente aus der VP vor das finite Verb gestellt (Verb-Zweit-Satz).
In der Entscheidungsfrage wird nur das finite Verb nach links herausgestellt (Verb-Erst-Satz).
}


\section{Nebensätze}
\label{sec:nebensaetze}

In diesem Abschnitt werden die verschiedenen Typen von Nebensätzen und ihre Besonderheiten im internen Aufbau und in ihrem externen syntaktischen Verhalten besprochen.
Die Definition des Nebensatzes aus Kapitel~\ref{sec:wortklassen} (Definition~\ref{def:nebensatz} auf Seite~\pageref{def:nebensatz}) kann unverändert zugrundegelegt werden.
Es handelt sich also um eine Konstituente, die ein finites Verb enthält, in der alle Valenzen gesättigt sind, und die nicht alleine stehen kann.
Fälle wie (\ref{ex:nebensaetze090}), in denen ein Nebensatz scheinbar alleine steht, analysieren wir als Ellipsen, also Strukturen, in denen eine hauptsatzartige Struktur getilgt wurde, s.\ (\ref{ex:nebensaetze091}).

\begin{exe}
  \ex\label{ex:nebensaetze089}
  \begin{xlist}
    \ex{\label{ex:nebensaetze090} Ob das wohl stimmt!}
    \ex{\label{ex:nebensaetze091} Ich frage mich\slash Ich bin nicht sicher\slash\ldots, ob das wohl stimmt!}
  \end{xlist}
\end{exe}

\subsection{Relativsätze}
\label{sec:relativsaetze}

Ein Relativsatz (RS) wie in (\ref{ex:relativsaetze092}) ist im prototypischen Fall ein Attribut zu einem nominalen Kopf, dem Bezugsnomen (vgl.\ Abschnitt~\ref{sec:nominalphrase}).
Der Sonderfall des \textit{freien Relativsatzes} wird weiter unten besprochen.

\begin{exe}
  \ex{\label{ex:relativsaetze092} Einen Tofu, der mir nicht geschmeckt hat, habe ich noch nie gegessen.}
\end{exe}

Wie schon in Abschnitt~\ref{sec:konstituentenstellungundfeldermodell} angedeutet, ist der Relativsatz unter den satzförmigen Strukturen ein Sonderfall bezüglich seiner internen Konstituentenstellung.
Das Verb bleibt im Verbkomplex stehen (VL-Satz), und das Relativpronomen -- genauer gesagt die \textit{Relativphrase} -- wird nach links (in das Vorfeld) bewegt.

Man kann sich die Struktur eines RS wie (\ref{ex:relativsaetze094}) verdeutlichen, indem man aus dem Relativsatz und seinem Bezugsnomen einen unabhängigen Satz baut.
Man ersetzt das Relativpronomen durch das Bezugsnomen und setzt dabei das Bezugsnomen in den Kasus, den das Relativpronomen hatte, s.\ (\ref{ex:relativsaetze095}).
Dann stellt man durch Umstellung des finiten Verbs eine V2-Stellung her, vgl.\ (\ref{ex:relativsaetze096}).

\begin{exe}
  \ex\label{ex:relativsaetze093}
  \begin{xlist}
    \ex{\label{ex:relativsaetze094} einen Tofu, der mir nicht geschmeckt hat}
    \ex{\label{ex:relativsaetze095} ein Tofu mir nicht geschmeckt hat}
    \ex{\label{ex:relativsaetze096} Ein Tofu hat mir nicht geschmeckt.}
  \end{xlist}
\end{exe}

Der Relativsatz wird durch Schema~\ref{str:rs} beschrieben, ein Analysebeispiel liefert Abbildung~\ref{fig:relativsaetze097}.
In diesem Satz ist die Relativphrase -- die wir im Schema mit XP\Up{relativ} abkürzen -- eine NP aus einem einfachen maskulinen Relativpronomen im Nominativ Singular.
Für ein Relativpronomen können wir annehmen, dass es im Lexikon bereits durch ein besonderes Merkmal als Relativelement gekennzeichnet ist, und dass die Relativphrase, deren Kopf es ist, dieses Merkmal erbt.


\index{Relativsatz}

\Phrasenschema{Relativsatz}{\label{str:rs}
  \centering
  \begin{forest}
    [RS, Ephr
      [XP\UpSub{relativ}{1}, Eobl]
      [VP\\{[\ldots\Ti\ldots]}, Eobl]
    ]
  \end{forest}

}

\begin{figure}[!htbp]
  \centering
  \begin{forest}
    [NP, calign=child, calign child=2
      [Art, tier=preterminal
        [\it einen]
      ]
      [\bf N, tier=preterminal
        [\it Tofu]
      ]
      [RS, calign=first
        [NP\Sub{1}, tier=preterminal
          [\it der, narroof, name=BeweDer]
        ]
        [VP, calign=last
          [NP, tier=preterminal
            [\it mir, narroof]
          ]
          [\Ti, tier=preterminal]
          {\draw[dotted, thick, ->] (.south) |- ++(0,-4.5em) -| (BeweDer.south);}
          [Ptkl, tier=preterminal
            [nicht]
          ]
          [\bf V, calign=last
            [\bf V, tier=preterminal
              [\it geschmeckt]
            ]
            [\bf V, tier=preterminal
              [\it hat]
            ]
          ]
        ]
      ]
    ]
  \end{forest}
  \caption{NP mit Relativsatz}
  \label{fig:relativsaetze097}
\end{figure}

\index{Relativphrase}

Wir müssen uns nun fragen, welche Form bzw.\ welche Merkmale die Relativphrase genau hat.
Es gelten zwei wichtige Beschränkungen:
Erstens kongruiert die Relativphrase mit dem Bezugsnomen in \textsc{Genus} und \textsc{Numerus}.
Zweitens erhält sie ihren Wert für \textsc{Kasus} innerhalb des Relativsatzes.
In Satz (\ref{ex:relativsaetze092}) hat \textit{der} das Merkmal [\textsc{Kasus}: \textit{nom}] dank der Rektion durch \textit{geschmeckt}.
Dass \textit{der} aber [\textsc{Genus}: \textit{mask}, \textsc{Numerus}: \textit{sg}] ist, kommt durch Kongruenz mit \textit{Tofu} zustande.

Die zweite Bedingung ist etwas zu eng gefasst, weil die Relativphrase nicht unbedingt eine einfache NP sein muss, deren Kasus vom Verb des Relativsatzes regiert wird.
Es gibt auch Relativsätze wie in (\ref{ex:relativsaetze099}), in denen die Relativphrase komplexer ist.
Wenn wir diesen Relativsatz genauso wie in (\ref{ex:relativsaetze093}) in einen unabhängigen Satz umwandeln, erhalten wir über (\ref{ex:relativsaetze100}) schließlich (\ref{ex:relativsaetze101}).


\begin{exe}
  \ex\label{ex:relativsaetze098}
    \begin{xlist}
      \ex{\label{ex:relativsaetze099} der Tofu, [auf den] ich mich freue}
      \ex{\label{ex:relativsaetze100} [auf den Tofu] ich mich freue}
      \ex{\label{ex:relativsaetze101} [Auf den Tofu] freue ich mich.}
    \end{xlist}
\end{exe}


Die Relativphrase (im Beispiel eingeklammert) hat die Form einer PP mit einem Relativpronomen als Kopf der von der Präposition regierten NP.
Die Präposition bleibt bei der Umwandlung erhalten, die Relativphrase (die PP mit dem eingebetteten Relativpronomen) wird also nur teilweise ersetzt.
Da \textit{freue} eine PP mit \textit{auf} regiert, wird der Relativphrase hier offensichtlich nicht einfach ein Kasus per Rektion innerhalb des Relativsatzes zugewiesen, sondern eine präpositionale Form.
Eine Relativ-PP muss aber nicht einmal regiert sein.
Die PP [\textit{auf der Straße}] (bzw.\ die Relativphrase [\textit{auf der}]) ist  in (\ref{ex:relativsaetze103}) keine Ergänzung von \textit{laufen}, sondern eine Angabe.


\begin{exe}
  \ex\label{ex:relativsaetze102}
  \begin{xlist}
    \ex{\label{ex:relativsaetze103} Die Straße, [auf der] wir den Marathon laufen, ist eine Autobahn.}
    \ex{\label{ex:relativsaetze104} [Auf der Straße] laufen wir den Marathon.}
  \end{xlist}
\end{exe}


In (\ref{ex:relativsaetze106}) haben wir es mit einer weiteren Art von Relativphrase zu tun.
Dazu zeigt (\ref{ex:relativsaetze107}) die Hauptsatz-Umformung.


\begin{exe}
  \ex\label{ex:relativsaetze105}
    \begin{xlist}
      \ex{\label{ex:relativsaetze106} Der Tofu, [dessen Geschmack] ich mag, ist ausverkauft.}
      \ex{\label{ex:relativsaetze107} [Den Geschmack [des Tofus]] mag ich.}
    \end{xlist}
\end{exe}


Das Pronomen \textit{dessen} ist ein pränominaler Genitiv innerhalb der NP [\textit{dessen Ge\-schmack}].
Die Relativphrase ist hier die gesamte NP, innerhalb derer das Pronomen den Kasus (Genitiv) erhält, den auch eine korrespondierende Genitiv-NP in einer unabhängigen NP erhalten würde, vgl.\ (\ref{ex:relativsaetze107}).
Hier kongruiert nicht die gesamte Relativphrase (die NP im Akkusativ) in Genus und Numerus mit dem Bezugsnomen, sondern nur die pränominale Genitiv-NP.

\index{Genitiv!pränominal}

In Abbildung~\ref{fig:relativsaetze108} wird die Struktur dieser Konstruktion abgebildet.
Die Rektionsanforderung des Verbs \textit{mag} (Akkusativ) wird durch die NP erfüllt, deren Kopf \textit{Geschmack} ist.
Der Kasus von \textit{dessen} ist ein Attributsgenitiv und abhängig von \textit{Geschmack}.
Das Relativpronomen \textit{dessen} kongruiert mit \textit{Tofu} in Genus und Numerus.%
\footnote{Der Bewegungspfeil wird der Übersicht wegen weggelassen.}

\begin{figure}[!htbp]
  \centering
  \begin{forest}
    [NP, calign=child, calign child=2
      [Art, tier=preterminal
        [\it der]
      ]
      [\bf N, tier=preterminal
        [\it Tofu]
        {\draw [->, bend right=30] (.south) to node [below, near start] {\footnotesize\textsc{Genus,Numerus}} (RekDessen.south);}
      ]
      [RS, calign=first
        [NP\Sub{1}, calign=first
          [NP, tier=preterminal
            [\it dessen, narroof, name=RekDessen]
          ]
          [\bf N, tier=preterminal
            [\it Geschmack, name=RekGeschmack]
            {\draw [->, bend left=25] (.south) to node [below, near start] {\footnotesize\textsc{Kasus}} (RekDessen.south);}
          ]
        ]
        [VP, calign=last
          [NP, tier=preterminal
            [\it ich, narroof]
          ]
          [\Ti]
          [\bf V, tier=preterminal
            [\it mag]
            {\draw [->, bend left=15] (.south) to node [below, near start] {\footnotesize\textsc{Kasus}} (RekGeschmack.south);}
          ]
        ]
      ]
    ]
  \end{forest}

  \caption{NP mit Relativsatz mit genitivischer Relativphrase}
  \label{fig:relativsaetze108}
\end{figure}

\index{Relativadverb}

Neben den normalen Relativpronomina gibt es noch eine Reihe von sogenannten \textit{Relativadverben} wie \textit{womit}, \textit{worin}, \textit{worauf} usw., die für sich alleine eine Relativphrase bilden.%
\footnote{Diese Relativadverben sind im Grunde \textit{Pronominaladverben} mit Relativfunktion.
Vgl.\ auch Übung~\ref{exc:wortklassen07} auf Seite~\pageref{exc:wortklassen07}.}
Wie geben hier nur ein Beispiel in (\ref{ex:relativsaetze109}).


\begin{exe}
  \ex{\label{ex:relativsaetze109} Alles, womit man rechnet, tritt auch ein.}
\end{exe}


Eine Sonderklasse von Relativsätzen sind die sogenannten \textit{freien Relativsätze}.
Freie Relativsätze sind intern wie jeder andere Relativsatz aufgebaut, beziehen sich aber nicht auf ein Bezugsnomen, sondern nehmen für sich allein den Platz einer NP ein.
Diese Art von Relativsatz wird in (\ref{ex:relativsaetze110}) illustriert.

\begin{exe}
  \ex\label{ex:relativsaetze110}
  \begin{xlist}
    \ex{\label{ex:relativsaetze111} Wer Klaviermusik mag, mag oft auch Chopin.}
    \ex{\label{ex:relativsaetze112} Wen man mag, beschenkt man.}
    \ex{\label{ex:relativsaetze113} Wir glauben, wem wir Vertrauen schenken.}
  \end{xlist}
\end{exe}

\index{Relativsatz!frei}

Im Normalfall muss die Relativphrase den Kasus haben, den auch eine NP an der Position des RS im einbettenden Satz hätte.
Dies hat zur Folge, dass der Kasus der Relativphrase im Relativsatz gleich dem externen Kasus sein muss.
Abbildung~\ref{fig:relativsaetze114} zeigt die Kasusanforderungen.%
\footnote{Es ist eine theoretisch gesehen problematische Annahme, dass eine NP (hier \textit{wen}) von zwei verschiedenen Verben regiert wird.
Insofern sind die Pfeile als Veranschaulichung zu verstehen, nicht als theoretische Modellierung im engen Sinn.}
Die Ungrammatikalität von (\ref{ex:relativsaetze115}) rührt aus einer Verletzung dieser speziellen Kasusanforderung her.

\begin{figure}[!htbp]
  \centering
  \begin{forest}
    [S, calign=child, calign child=2
      [RS\Sub{3}, calign=first
        [NP\Sub{1}, tier=preterminal
          [\it wen, narroof, name=RekWen]
        ]
        [VP, calign=last
          [NP, tier=preterminal
            [\it man, narroof]
          ]
          [\Ti, tier=preterminal]
          [\bf V, tier=preterminal
            [\it mag]
            {\draw [->, bend left=30] (.south) to node [below, near start] {\footnotesize\textsc{Kasus}} (RekWen.south);}
          ]
        ]
      ]
      [\bf V\Sub{2}, tier=preterminal
        [\it beschenkt]
        {\draw [->, bend left=45] (.south) to node [below, near start] {\footnotesize\textsc{Kasus}} (RekWen.south);}
      ]
      [VP, calign=last
        [NP, tier=preterminal
          [\it man, narroof]
        ]
        [\Tiii, tier=preterminal]
        [\Tii, tier=preterminal]
      ]
    ]
  \end{forest}
  \caption{Satz mit freiem Relativsatz}
  \label{fig:relativsaetze114}
\end{figure}

\begin{exe}
  \ex[*]{\label{ex:relativsaetze115} Wer Klaviermusik mag, beschenkt man mit Konzertkarten.}
\end{exe}

Um einen Satz wie (\ref{ex:relativsaetze115}) zu reparieren, muss der Relativsatz an einen pronominalen Kopf als Bezugsnomen angeschlossen werden, der die Kasusanforderung des einbettenden Satzes erfüllen kann.
In (\ref{ex:relativsaetze116}) ist ein solches Pronomen in Form von \textit{denjenigen} eingesetzt.
Es erfüllt als Akkusativ die Rektionsanforderung von \textit{beschenkt}, während die Relativphrase \textit{der} die Rektionsanforderung (Nominativ) von \textit{mag} innerhalb des Relativsatzes erfüllt.

\begin{exe}
  \ex{\label{ex:relativsaetze116} Denjenigen, der Klaviermusik mag, beschenkt man mit Konzertkarten.}
\end{exe}

Eine andere Möglichkeit ist es, den freien Relativsatz mit \textit{w}"=Pronomen vor das Vorfeld zu stellen und ein im Kasus angepasstes korrelierendes Pronomen ins Vorfeld zu stellen, wie in (\ref{ex:relativsaetze117}).

\begin{exe}
  \ex{\label{ex:relativsaetze117} Wer Klaviermusik mag, den beschenkt man mit Konzertkarten.}
\end{exe}

Manche Sprecher akzeptieren Strukturen wie in (\ref{ex:relativsaetze115}) allerdings doch, wenn der Kasus der Relativphrase obliquer ist als der durch Rektion im Matrixsatz geforderte, vgl.\ (\ref{ex:relativsaetze119}).
Wenn die Relativphrase in einem weniger obliquen Kasus steht, funktioniert das allerdings nicht, wie in (\ref{ex:relativsaetze120}).


\begin{exe}
  \ex\label{ex:relativsaetze118}
  \begin{xlist}
    \ex[?]{\label{ex:relativsaetze119} Wen es stört, kann gehen.}
    \ex[*]{\label{ex:relativsaetze120} Wer hier stört, beschenkt man.}
  \end{xlist}
\end{exe}


Abschließend müssen einige Stellungsbesonderheiten der Relativsätze diskutiert werden.
Bezüglich der Stellung der Relativsätze im einbettenden Satz müssen zwei Fälle unterschieden werden.
Die Fälle sind in (\ref{ex:relativsaetze121}) und (\ref{ex:relativsaetze124}) illustriert.


\begin{exe}
  \ex\label{ex:relativsaetze121}
  \begin{xlist}
    \ex{\label{ex:relativsaetze122} Die Gavotte, die ich am liebsten mag, hat Ariel gespielt.}
    \ex{\label{ex:relativsaetze123} Ariel hat die Gavotte, die ich am liebsten mag, gespielt.}
  \end{xlist}
  \ex\label{ex:relativsaetze124}
  \begin{xlist}
    \ex{\label{ex:relativsaetze125} Ariel hat die Gavotte gespielt, die ich am liebsten mag.}
    \ex{\label{ex:relativsaetze126} Ich glaube, dass Ariel die Gavotte gespielt hat, die ich am liebsten mag.}
  \end{xlist}
\end{exe}
\index{Nachfeld}


In (\ref{ex:relativsaetze121}) ist der Relativsatz innerhalb der NP rechts vom Kopf positioniert, also genau dort, wo er gemäß Schema~\ref{str:ngr} (Seite~\pageref{str:ngr}) stehen soll.
Dabei ist es egal, ob die NP mit dem Relativsatz ins Vorfeld bewegt wird wie in (\ref{ex:relativsaetze122}), oder ob die NP im Mittelfeld verbleibt wie in (\ref{ex:relativsaetze123}).
Bereits in Abschnitt~\ref{sec:dasfeldermodell} (s.\ vor allem Abbildung~\ref{fig:dasfeldermodell053}) wurden aber Sätze wie die in (\ref{ex:relativsaetze124}) gezeigt.
Hier wird der Relativsatz nach rechts ins Nachfeld herausgestellt und damit von der NP getrennt.
Dies kann sowohl aus unabhängigen Sätzen wie in (\ref{ex:relativsaetze125}) als auch aus eingebetteten Sätzen wie in (\ref{ex:relativsaetze126}) geschehen.
In (\ref{ex:relativsaetze126}) wurde der Relativsatz aus dem \textit{dass}-Satz über die rechte Satzklammer seines Matrixsatzes nach rechts ins Nachfeld verschoben.%
\footnote{Wir geben hier keine Strukturen dafür an.
Übung \ref{exc:saetze04} auf Seite~\pageref{exc:saetze04} beschäftigt sich mit der Frage von Konstituentenstrukturen bei Bewegung ins Nachfeld.}

\subsection{Ergänzungssätze}
\label{sec:ergaenzungssaetze}

\textit{Ergänzungssätze} (oft auch als \textit{Komplementsätze} bezeichnet) sind Sätze, die als Ergänzung zu Verben fungieren, die also eine Valenzanforderung saturieren.%
\footnote{Die Begriffe \textit{Komplement} und \textit{Ergänzung} sind weitgehend synonym, vgl.\ Abschnitt~\ref{sec:valenz}.}
Dabei unterscheidet man gemäß Definition~\ref{def:komplementsatz} \textit{Subjektsätze} und \textit{Objektsätze}.


\Definition{Ergänzungssatz}{\label{def:komplementsatz}%
Ein \textit{Ergänzungssatz} ist eine Ergänzung in Form eines Nebensatzes.
Der Untertyp des \textit{Subjektsatzes} nimmt die Stelle ein, die auch von einer NP im Nominativ eingenommen werden könnte.
Alle anderen Ergänzungssätze sind \textit{Objektsätze}.
\index{Ergänzungssatz}
\index{Subjekt!Satz}
\index{Objekt!Satz}
}

Wenn wir uns zunächst den Objektsätzen zuwenden, dann müssen wir formal zwei Typen unterscheiden.
Das Verb \textit{wissen} in (\ref{ex:ergaenzungssaetze127}) regiert einen \textit{dass}-Satz.
In (\ref{ex:ergaenzungssaetze129}) regiert es einen \textit{w}"=Fragesatz und in (\ref{ex:komplementsaetze130}) einen Fragesatz mit der Frage"=Subjunktion \textit{ob}.

\begin{exe}
  \ex{\label{ex:ergaenzungssaetze127} Michelle weiß, [dass die Corvette nicht anspringen wird].}
  \ex\label{ex:ergaenzungssaetze128}
  \begin{xlist}
    \ex{\label{ex:ergaenzungssaetze129} Michelle will wissen, [wer die Corvette gewartet hat].}
    \ex{\label{ex:ergaenzungssaetze130} Michelle will wissen, [ob die Corvette gewartet wurde].}
  \end{xlist}
\end{exe}

Verben, die Objektsätze fordern, folgen drei Mustern, je nachdem, mit welchen Arten von Objektsätzen sie stehen können.
Entweder stehen sie nur mit \textit{dass}-Sätzen wie \textit{behaupten} in (\ref{ex:ergaenzungssaetze131}), nur mit Fragesätzen wie \textit{untersuchen} in (\ref{ex:komplementsaetze134}) oder mit beidem wie \textit{hören} in (\ref{ex:komplementsaetze137}) oder \textit{wissen} in (\ref{ex:komplementsaetze127}) und (\ref{ex:komplementsaetze128}).

\begin{exe}
  \ex\label{ex:ergaenzungssaetze131}
  \begin{xlist}
    \ex[]{\label{ex:ergaenzungssaetze132} Michelle behauptet, dass die Corvette nicht anspringt.}
    \ex[*]{\label{ex:ergaenzungssaetze133} Michelle behauptet, wie\slash ob die Corvette nicht anspringt.}
  \end{xlist}
  \ex\label{ex:ergaenzungssaetze134}
  \begin{xlist}
    \ex[*]{\label{ex:ergaenzungssaetze135} Michelle untersucht, dass der Vergaser funktioniert.}
    \ex[]{\label{ex:ergaenzungssaetze136} Michelle untersucht, wie\slash ob der Vergaser funktioniert.}
  \end{xlist}
  \ex\label{ex:ergaenzungssaetze137}
  \begin{xlist}
    \ex[]{\label{ex:ergaenzungssaetze138} Michelle hört, dass die Nockenwelle läuft.}
    \ex[]{\label{ex:ergaenzungssaetze139} Michelle hört, wie\slash ob die Nockenwelle läuft.}
  \end{xlist}
\end{exe}

Bei \textit{dass}-Sätzen gibt es Alternationen mit Infinitivkonstruktionen mit \textit{zu} (selbständigen infiniten VP) wie in (\ref{ex:ergaenzungssaetze140}).
Diese Infinitive werden in Abschnitt~\ref{sec:infinitivkontrolle} genauer besprochen.

\begin{exe}
  \ex\label{ex:ergaenzungssaetze140}
  \begin{xlist}
    \ex{\label{ex:ergaenzungssaetze141} Michelle glaubt, [dass sie das Geräusch erkennt].}
    \ex{\label{ex:ergaenzungssaetze142} Michelle glaubt, [das Geräusch zu erkennen].}
  \end{xlist}
\end{exe}

Nach Definition~\ref{def:komplementsatz} nehmen Subjektsätze die Position einer NP im Nominativ ein.
Ein Subjektsatz ist in (\ref{ex:ergaenzungssaetze144}) illustriert.
In (\ref{ex:ergaenzungssaetze145}) ersetzt ein Nominativ den Subjektsatz.
Eine solche Ersetzung funktioniert nicht immer, und der Status eines Nebensatzes als Subjektsatz ist nicht an die praktische Möglichkeit der Ersetzung durch eine NP gebunden.


\begin{exe}
  \ex\label{ex:ergaenzungssaetze143}
  \begin{xlist}
    \ex{\label{ex:ergaenzungssaetze144} [Dass die Sonne scheint], freut uns.}
    \ex{\label{ex:ergaenzungssaetze145} [Der Sonnenschein] freut uns.}
  \end{xlist}
\end{exe}


\index{Mittelfeld}
\index{Korrelat}
\index{Nebensatz}
Die bisher besprochenen Ergänzungssätze standen alle entweder im Vorfeld oder im Nachfeld.
Tatsächlich ist es ungewöhnlich (wenn auch je nach Sprecher und Gestalt des Satzes nicht ganz ausgeschlossen), dass Ergänzungssätze im Mittelfeld stehen, wo sie als Ergänzungen des Verbs eigentlich zu erwarten wären.
Die (c)-Sätze in (\ref{ex:ergaenzungssaetze146})--(\ref{ex:komplementsaetze154}) illustrieren dies.


\begin{exe}
  \ex\label{ex:ergaenzungssaetze146}
  \begin{xlist}
    \ex[]{\label{ex:ergaenzungssaetze147} [Dass sie unseren Kuchen mag], hat Sarah uns nun doch eröffnet.}
    \ex[]{\label{ex:ergaenzungssaetze148} Sarah hat uns nun doch eröffnet, [dass sie unseren Kuchen mag].}
    \ex[?]{\label{ex:ergaenzungssaetze149} Sarah hat uns, [dass sie unseren Kuchen mag], nun doch eröffnet.}
  \end{xlist}

  \ex\label{ex:ergaenzungssaetze150}
  \begin{xlist}
    \ex[]{\label{ex:ergaenzungssaetze151} [Ob Pavel unseren Kuchen mag], haben wir uns oft gefragt.}
    \ex[]{\label{ex:ergaenzungssaetze152} Wir haben uns oft gefragt, [ob Pavel unseren Kuchen mag].}
    \ex[?]{\label{ex:ergaenzungssaetze153} Wir haben uns, [ob Pavel unseren Kuchen mag], oft gefragt.}
  \end{xlist}
  \ex\label{ex:ergaenzungssaetze154}
  \begin{xlist}
    \ex[]{\label{ex:ergaenzungssaetze155} [Wer die Rosinen geklaut hat], wollen wir endlich wissen.}
    \ex[]{\label{ex:ergaenzungssaetze156} Wir wollen endlich wissen, [wer die Rosinen geklaut hat].}
    \ex[?]{\label{ex:ergaenzungssaetze157} Wir wollen, [wer die Rosinen geklaut hat], endlich wissen.}
  \end{xlist}
\end{exe}


Die Ergänzungssätze werden also überwiegend aus dem Mittelfeld herausbewegt.
Wenn ein Objektsatz ins Nachfeld gestellt wird, dann kann wie eine sichtbare Spur ein sogenanntes \textit{Korrelat} im Mittelfeld stehen.\index{Spur}
Die (b)-Sätze aus (\ref{ex:ergaenzungssaetze146})--(\ref{ex:komplementsaetze154}) werden in (\ref{ex:komplementsaetze158}) mit dem Korrelat \textit{es} wiederholt.


\begin{exe}
  \ex\label{ex:ergaenzungssaetze158}
  \begin{xlist}
    \ex{\label{ex:ergaenzungssaetze159} Sarah hat es uns eröffnet, [dass sie unseren Kuchen mag].}
    \ex{\label{ex:ergaenzungssaetze160} Wir haben es uns gefragt, [ob Pavel unseren Kuchen mag].}
    \ex{\label{ex:ergaenzungssaetze161} Wir wollen es wissen, [wer die Rosinen aus dem Kuchen geklaut hat].}
  \end{xlist}
\end{exe}


Das Korrelat \textit{es} ist hier optional, muss also nicht stehen.
Wenn der Ergänzungssatz ein Präpositionalobjekt vertritt, wird das Korrelat bei vielen Verben wie \textit{hinweisen} obligatorisch, s.\ (\ref{ex:ergaenzungssaetze162}).
Das Verb \textit{hinweisen} fordert eine NP im Nominativ und eine PP mit \textit{auf}, vgl.\ (\ref{ex:ergaenzungssaetze163}).
Wenn ein Ergänzungssatz vorliegt, wird die Valenzanforderung formal durch \textit{darauf} im Mittelfeld gesättigt, das in (\ref{ex:ergaenzungssaetze164}) als Korrelat zum Ergänzungssatz fungiert.
Satz (\ref{ex:ergaenzungssaetze165}) zeigt, dass das Korrelat nicht fehlen darf.


\begin{exe}
  \ex\label{ex:ergaenzungssaetze162}
  \begin{xlist}
    \ex[]{\label{ex:ergaenzungssaetze163} Ich weise [auf den leckeren Kuchen] hin.}
    \ex[]{\label{ex:ergaenzungssaetze164} Ich weise darauf hin, [dass der Kuchen lecker ist].}
    \ex[*]{\label{ex:ergaenzungssaetze165} Ich weise hin, [dass der Kuchen lecker ist].}
  \end{xlist}
\end{exe}


Auch Subjektsätze können in Konstruktionen mit Korrelaten stehen wie in (\ref{ex:ergaenzungssaetze166}).
Das Korrelat muss dabei immer vor dem Nebensatz stehen.

\begin{exe}
  \ex\label{ex:ergaenzungssaetze166}
  \begin{xlist}
    \ex[ ]{\label{ex:ergaenzungssaetze167} Es hat uns gefreut, [dass Sarah unseren Kuchen mochte].}
    \ex[ ]{\label{ex:ergaenzungssaetze168} Uns hat es gefreut, [dass Sarah unseren Kuchen mochte].}
    \ex[ ]{\label{ex:ergaenzungssaetze169} Uns hat gefreut, [dass Sarah unseren Kuchen mochte].}
    \ex[*]{\label{ex:ergaenzungssaetze170} [Dass Sarah unseren Kuchen mochte], hat es uns gefreut.}
  \end{xlist}
\end{exe}

Damit beenden wir die sehr knappe Darstellung der Ergänzungssätze.
Mit den Ergänzungssätzen sind die nun folgenden Angabensätze bezüglich ihres internen syntaktischen Aufbaus verwandt.
Sie unterscheiden sich im Wesentlichen dadurch, dass sie keine Valenzstelle saturieren.


\subsection{Angabensätze}
\label{sec:angabensaetze}

\index{Subjunktion}

Angabensätze (oft auch als \textit{Adverbialsätze} bezeichnet) sind VL-Sätze, die von einer Subjunktion eingeleitet werden.
Sie werden normalerweise nach der semantischen Funktion ihrer jeweiligen Subjunktion unterklassifiziert.%
\footnote{Siehe Abschnitt~\ref{sec:funktionenvonsatzartigenkonstituenten} für eine kurze funktionale Betrachtung.}
Definition~\ref{def:angabensatz} klärt den Begriff, Beispiele werden dann in (\ref{ex:angabensaetze171}) gegeben.

\Definition{Angabensatz}{\label{def:angabensatz}%
Ein \textit{Angabensatz} ist ein von einer Subjunktion eingeleiteter VL-Nebensatz, der keine Valenzstelle im Matrixsatz saturiert.
\index{Angabensatz}
}

\begin{exe}
  \ex\label{ex:angabensaetze171}
  \begin{xlist}
    \ex{\label{ex:angabensaetze172} [Weil es regnet], bleibe ich lieber zuhause.}
    \ex{\label{ex:angabensaetze173} Wir haben Kaffee getrunken, [nachdem der Kuchen aufgegessen war].}
    \ex{\label{ex:angabensaetze174} [Obwohl das Buch interessant ist], ignorieren wir es.}
  \end{xlist}
\end{exe}

Angabensätze lassen sich oft als eine nicht-satzförmige Angabe umformulieren, \zB als PP.
Parallel zu (\ref{ex:angabensaetze171}) sind in (\ref{ex:angabensaetze175}) solche PPs realisiert.

\begin{exe}
  \ex\label{ex:angabensaetze175}
  \begin{xlist}
    \ex{\label{ex:angabensaetze176} [Wegen des Regens] bleibe ich lieber zuhause.}
    \ex{\label{ex:angabensaetze177} Wir haben [nach dem Kuchenessen] Kaffee getrunken.}
    \ex{\label{ex:angabensaetze178} [Trotz unseres Interesses an dem Buch] ignorieren wir es.}
  \end{xlist}
\end{exe}

\index{Nachfeld}
\index{Mittelfeld}

Wie die Beispiele in (\ref{ex:angabensaetze171}) zeigen, stehen Angabensätze genauso wie Ergänzungssätze gerne im Vorfeld oder Nachfeld.
Ob sie aus dem Mittelfeld herausbewegt werden, oder ob sie direkt in diese Positionen gestellt werden, kann und muss hier nicht entschieden werden (vgl.\ zu dieser Frage auch schon Abschnitt~\ref{sec:verbzweitsaetze}, besonders Abbildung~\ref{fig:verbzweitsaetze077}).
Satz~\ref{satz:eigenschaftangabensatz} fasst die Erkenntnisse zusammen.

\Satz{Eigenschaften von Angabensätzen}{\label{satz:eigenschaftangabensatz}%
Angabensätze lassen sich (im Gegensatz zu Ergänzungssätzen) oft unter Beibehaltung der Bedeutung in nicht-satzförmige Angaben (\zB PPs) umformen.
Sie stehen i.\,d.\,R.\ im Vorfeld oder Nachfeld.
\index{Angabensatz}
}

\index{Konditionalsatz}
\index{Satz!Verb-Erst--}

Einen Sonderfall bilden die Konditionalsätze, die normalerweise mit Subjunktionen wie \textit{wenn}, \textit{falls}, \textit{sofern} eingeleitet werden, s.\ (\ref{ex:angabensaetze180}).
Die Subjunktion kann entfallen.
Der Konditionalsatz wird dann als V1-Satz realisiert, der im Vorfeld seines Matrixsatzes stehen muss, vgl.\ (\ref{ex:angabensaetze181}).

\begin{exe}
  \ex\label{ex:angabensaetze179}
  \begin{xlist}
    \ex{\label{ex:angabensaetze180} [Wenn der Kuchen aufgegessen ist], stürzen wir uns auf die Kekse.}
    \ex{\label{ex:angabensaetze181} [Ist der Kuchen aufgegessen], stürzen wir uns auf die Kekse.}
  \end{xlist}
\end{exe}

\Zusammenfassung{%
In einem Relativsatz bezieht sich das Relativpronomen als Teil der Relativphrase auf das Bezugsnomen und kongruiert mit ihm in Numerus und Genus.
Ihren Kasus bzw.\ ihre Form (\zB als PP mit einer bestimmten Präposition) erhält die Relativphrase innerhalb des Relativsatzes (per Rektion oder als Angabe).
Relativsätze können auch ohne Bezugsnomen als freie Relativsätze auftreten und verhalten sich dann (mit einigen Einschränkungen) wie eine NP.
Ergänzungssätze sind Sätze, die eine Valenzstelle (Subjekt oder Objekt) des Matrixverbs füllen.
Angabensätze sind Angaben zum Matrixverb.
Nebensätze werden typischerweise nach rechts aus dem Satz versetzt (ggf.\ unter Einsetzung eines Korrelats) oder stehen (stets ohne Korrelat) im Vorfeld.
}

\Uebungen

\Uebung{saetze01} \label{exc:saetze01} Analysieren Sie die eingeklammerten Strukturen im Rahmen des Feldermodells nach dem Muster des ersten Beispiels.
Bei den Sätzen \ref{it:angabensaetze183} und \ref{it:angabensaetze184} handelt es sich um Transferaufgaben.

\begin{enumerate}
  \item{[Sarah isst den Kuchen alleine auf.]}
    \begin{itemize}
      \item Kf: ---
      \item Vf: Sarah
      \item LSK: isst
      \item Mf: den Kuchen alleine
      \item RSK: auf
      \item Nf: ---
    \end{itemize}
  \item{[Man sollte den Tag genießen.]}
  \item{[Kann mal jemand das Fenster aufmachen?]}
  \item\label{it:angabensaetze182} Das ist das Eis, [das wir selber gemacht haben].
  \item{[Was hat Ischariot gemalt?]}
  \item{[Gehst du?]}
  \item{\label{it:angabensaetze183} [Geh!]}
  \item\label{it:angabensaetze184} Es ist eine tolle Sommernacht, [denn der Mond scheint hell].
  \item{[Den leckeren Kuchen auf dem Tisch hatte Rigmor sofort entdeckt.]}
  \item{[Obwohl Liv einkaufen wollte], ist nichts im Haus.}
  \item Kann man feststellen, [wer den Kuchen gegessen hat]?
\end{enumerate}

\Uebung{saetze02} \label{exc:saetze02} Analysieren Sie die folgenden komplexen Sätzen im Rahmen des Feldermodells nach dem Muster des ersten Beispiels.
Dabei sind von eingebetteten Nebensätzen keine Analysen durchzuführen.

\begin{enumerate}
  \item Dass der Kuchen gegessen wurde, bedauern alle sehr, die es erfahren haben.
    \begin{itemize}
      \item Kf: ---
      \item Vf: Dass der Kuchen gegessen wurde
      \item LSK: bedauern
      \item Mf: alle sehr
      \item RSK: ---
      \item Nf: die es erfahren haben
    \end{itemize}
  \item Wohin man auch blickt, kann man die Bäume kaum erkennen, denn der Schnee bedeckt alles.
  \item Geht derjenige, der kommt, auch wieder?
  \item Die Kollegen, denen wir nichts vom Kuchen gegeben haben, schimpfen.
  \item Denn ob es Eis gibt, kann nur einer wissen, der Zugang zur Eismaschine hat.
  \item Liv will, dass Rigmor ihr von dem Eis abgibt.
\end{enumerate}

\Uebung{saetze03} \label{exc:saetze03} Führen Sie Konstituentenanalysen der folgenden Auswahl einfacher Sätze aus Übung \ref{exc:saetze01} durch (ohne Bewegungspfeile).
Für ein Beispiel (erster Satz) vgl.\ Abbildung~\ref{fig:syntaxderpartikelverben083}.

\begin{enumerate}
  \item{Sarah isst den Kuchen alleine auf.}
  \item{Man sollte den Tag genießen.}
  \item{Kann mal jemand das Fenster aufmachen?}
  \item{Was hat Ischariot gemalt?}
  \item{Gehst du?}
  \item{Den leckeren Kuchen auf dem Tisch hatte Rigmor sofort entdeckt.}
\end{enumerate}

\Uebung[\tristar]{saetze04} \label{exc:saetze04} Führen Sie Konstituentenanalysen für die folgende Auswahl aus den komplexen Sätzen aus Übung \ref{exc:saetze02} durch.
Es handelt sich überwiegend um eine Transferaufgabe:
Überlegen Sie, wie das Nachfeld in den Konstituentenstrukturen abgebildet werden kann.

\begin{enumerate}
  \item Dass der Kuchen gegessen wurde, bedauern alle sehr, die es erfahren haben.
  \item Die Kollegen, denen wir nichts vom Kuchen gegeben haben, schimpfen.
  \item Liv will, dass Rigmor ihr von dem Eis abgibt.
\end{enumerate}

\Uebung[\tristar]{saetze05} \label{exc:saetze05} Analysieren Sie die folgenden NPs mit Relativsatz nach dem Muster von Abbildung~\ref{fig:relativsaetze097} (s.\ Seite~\pageref{fig:relativsaetze097}), aber ohne Kongruenz- und Rektionspfeile.

\begin{enumerate}
  \item{[Das Buch, das ich lese], gehört nicht mir.}
  \item Wir mögen [Menschen, auf die wir vertrauen können].
  \item Wir treffen [die Kommilitoninnen, deren Kuchen wir gegessen haben].
\end{enumerate}

\Uebung[\tristar]{saetze06} \label{exc:saetze06} Dialektal gibt es Relativsätze bzw.\ eingebettete \textit{w}"=Sätze wie in (\ref{ex:angabensaetze185}).

\begin{exe}
  \ex{\label{ex:angabensaetze185} Ich weiß, [wer dass kommt].}
\end{exe}

Überlegen Sie, was hier anders ist als im Standard und geben Sie eine Felderanalyse und eine Konstituentenstruktur an.

\Uebung[\tristar]{saetze07} \label{exc:saetze07} Die deutsche Orthographie zeigt viele interessante grammatische Beziehungen auf.
Überlegen Sie, warum die Form des Verbs \textit{zurückbleiben} in (\ref{ex:angabensaetze187}) zusammengeschrieben, aber in (\ref{ex:angabensaetze188}) auseinandergeschrieben wird.

\begin{exe}
  \ex\label{ex:angabensaetze186}
  \begin{xlist}
    \ex{\label{ex:angabensaetze187} Es ist in Ordnung, wenn der große Schreibtisch zurückbleibt.}
    \ex{\label{ex:angabensaetze188} Zurück bleibt der Schreibtisch nur, wenn der LKW randvoll ist.}
  \end{xlist}
\end{exe}

