
%%%%%%%%%%%%%%%%%%%%%%%%%%%%%%%%%%%%%%%%%%%%%%%%%%%%%%%%%%%%%%%%%%%%%%%
\paragraph*{\citet{Feilke2012} -- Bildungssprache}

Bildungssprache setzt Verstehen und abstraktes Bedeutungswissen voraus (4)

Sprache als zentraler Gegenstand und Medium von Bildung (4, zitiert Roth 2010:574)

Bildungssprache werde zwar in vieler Hinsicht vorausgesetzt, aber nicht gelehrt. (4)

[Bildungssprache] sind die besonderen sprachlichen Formate und Prozeduren einer auf Texthandlungen wie Beschreiben, Vergleichen, Erklären, Analysieren, Erörtern etc.\ bezogenen Sprachkompetenz, wie man sie im schulischen und akademischen Bereich findet. (5)

[Die sprachlichen Mittel findet man auch] auch in Texten mit Alltagsthemen, die sachlich komplexe Verhältnisse darstellen. (5)

Vielmehr ist es die „Hauptaufgabe der Bildungssprache, zwischen Wissenschaft bzw. speziellem Sphärenwissen und Alltag zu vermitteln“ (Ortner 2009, S. 2232) (6)

Bildungssprache sei primär auf die Schriftsprache (bzw.\ schriftliche Situationen) bezogen, damit einher gehe eine Dekontextualisierung und damit erhebliche Abstraktion. (6)

Auf solche komplexen Verstehens- bzw. Formulierungsaufgaben bereitet ein rein formenorientierter Grammatikunterricht kaum vor. (8)

Explizieren (8): Sachverhalte für Lesende nachvollziehbar explizieren und fokussieren; komplexe Adverbiale, Attribute und Sätze, explizite Konnexion (konditional, final, adversativ usw.)

Verdichten (8): Sachverhalte ohne finites Verb ausdrücken; Nominalisierungen, Komposita, Partizipialattribute, präpositionale Adverbiale, FVG

Verallgemeinern (9): dekontextualisiert und generalisierte Darstellung; generische dritte Person, generischer Artikel, generisches Präsens

Diskutieren (9): Sachverhalte als Gegenstände mit variierendem Wahrheitsgehalt; Modalverben, Konjunktiv, Konzessiva

[Die Bildungssprache] ist deshalb gleichermaßen ein Bildungskapital, wie sie eine Hürde für das Verstehen sein kann. Auch bezogen auf die Produktion gibt es besondere Ansprüche der Bildungssprache: Erwartet wird, dass die Schüler das für das Verstehen Notwendige grammatisch, lexikalisch und textlich explizit machen. Diese Explizitformenerwartung (Maas 2010, Feilke 2012) ist nicht nur kommunikativ begründet. Gerade im Kontext der Schule hat das Explizieren nicht nur mit Verständigung zu tun, sondern verweist auf die kognitive Funktionalisierung der Sprache für Zwecke des Lernens: einen Begriff definieren, ein Konzept strukturieren, einen Text zusammenfassen usw. (11)

Die Situierung der Sprachanforderungen ist ein wichtiges Merkmal bildungssprachlicher Didaktik. Es sollten Kontexte konstruiert werden, in denen der Formengebrauch ebenso wie deren Reflexion pragmatisch motiviert ist. (12)

%%%%%%%%%%%%%%%%%%%%%%%%%%%%%%%%%%%%%%%%%%%%%%%%%%%%%%%%%%%%%%%%%%%%%%%
\paragraph*{\citet{Schroeterbrauss2013}}

Bildungssprachliche Kompetenzen im Fachunterricht stärken und global entwickeln.
Deutschunterricht kann dann mehr zu den Grammatik-Kernkompetenzen zurückkehren.
Benachteiligung nicht-erstsprachlicher Lerner in Deutschland laut PISA.

%%%%%%%%%%%%%%%%%%%%%%%%%%%%%%%%%%%%%%%%%%%%%%%%%%%%%%%%%%%%%%%%%%%%%%%
\paragraph*{\citet{Eisenberg2013c}}

Grammatik nicht als Selbstzweck (7)



Transferproblem: Grammatik -> Sprache können (8)
Sprachwissenschaft in Hochschulen oft wenig an Schulrelevanz interessiert (8)

Im Vorwort von \citet[2]{KoepckeZiegler2013}: Inhalte des Grammatikunterrichts unverändert, empirisch nicht geklärt, was hilft und was nicht (9)

Beherrschung der Orthographie setzt voraus, dass die Anwendenden grammatisch versiert sind (9--10); und \citet{Duerscheid2013} sagt korrekt Sprachbetrachtung ist Schriftbetrachtung (13)

Sprachreflexion relevant für die Beherrschung des Standards als einer von vielen Varietäten (10), ebenso Textsortenspezifik (10), aber das funktioniert nur, wenn man die verschiedenen Varietäten (aller Art) als System begreift, das auf zugrundeliegende Möglichkeiten eines grammatischen Systems aufbaut (11)

Modellunabhängige Formulierung von grammatischen Regularitäten (11)

Orientierung am Gebrauch nicht so einschlägig, wenn es um Normbeherrschung geht (12); deswegen (RS) auch Unterschied zwischen meiner Forschung und meiner Lehre bzw. dem Buch

Sekundarstufe II ist grammatikfreie Zone (12), Tendenz von Schulgrammatiken ist die Reduktion von Theorie und Methode

BA-Bücher handeln vor allem von Sprachwissenschaft statt von Sprache (13)

%%%%%%%%%%%%%%%%%%%%%%%%%%%%%%%%%%%%%%%%%%%%%%%%%%%%%%%%%%%%%%%%%%%%%%%
\paragraph*{\citet{Eisenberg2004}}

Sprachliches Verhalten zu beurteilen heißt, es verstehen zu müssen. (4)

Literalisierung von Lernenden führt zu einer Restrukturierung ihrer unbewussten Grammatik (4) Siehe auch fast wörtlich so \citet[78]{Portmanntselikas2011}.

KMK formuliert keine konkreten Grammatikkenntnisse für Schüler, dafür aber hohes Niveau in sprachlichem Wissen für Lehrer (6)

Unter anderem wird von Schülern registersicheres Sprechen und Schreiben sowie "`Standardsprache"' verlangt (6)

Mit Bezug auf \citet{Braun1979}, eine Untersuchung von Zweifelsfällen in der Bewertung durch Studierende und Lehrpersonen ist festzustellen, dass diese meist mehr Fehler finden, als sie sollten.
Dabei besteht zwischen der untersuchten Gruppe nicht immer Einigung, welche \textit{eine} Lösung die richtige sein soll.
Das entspricht der Beobachtung des Autors (RS), dass oft statt einer differenzierten grammatischen\slash graphematischen Analyse und einem Abgleich mit dem Standard, wo er denn definiert ist, persönliche Richtigkeitsurteile vehement vertreten werden, (angehende) Lehrpersonen also ihren eigenen Stil als Bewertungsgrundlage hernehmen.

Neben ausufernden Formenpluralismus und Regulierungswut (bzw. die Einheitslösung) stellt Eisenberg die Einordnung sprachlicher Formen ins System und die genaue Analyse der Bedeutungs- und Gebrauchsunterschiede von Zweifelsfällen. (8-9) Genau das tut die Analyse von Zweifelsfällen als Alternationen. (RS)

Beispiel für interne Funktion: Schwa-Tilgung in der ersten Singular (9--11)

Grammatik für das Lehramtsstudium: KEINE Fixierung auf Zweifelsfälle (11), denn (RS) die Zweifelsfälle bleiben ohne das System eine nicht verstehbare Liste von Einzelfällen. Außerdem (PE): Methodik statt Fakten! (11)

Kein Spielen mit der Norm beim Orthographieunterricht! Das kann eine Hauptquelle von Rechtschreibschwächen. (12) Es gibt reichlich Fehlertypologien, die Lehrpersonen kennen sollten, aber (RS) ihr Verständnis erfordert die Kenntnis des gesamten Systems.

Eine methodisch-didaktisch erfolgreiche Aufarbeitung setzt explizites und (konzeptuell) vollständiges Wissen bei Lehrpersonen voraus. (14, stark RS-formuliert)

Schriftspracherwerb stellt eine Rekonstruktion der Grammatik dar und darf nicht behindert werden. (12, 14)
Die Schrift führt zur Ausbildung eines metasprachlichen Bewusstseins. (14)
Damit einher geht die Fähigkeit zu dekontextualisierter Sprache. (15)
Kinder lernen in der Schule eine neue Sprache, nicht bloß das Schreiben einer bereits erworbenen Sprache. (15)
Erlernen von neuen Stilen (eher: Registern, RS), die mit spezifischen grammatischen Mitteln verknüpft sind (z.B. der explizierende Stil mit kausalen und konditionalen Konstruktionen, komplexen Tempusbezügen usw.) (15)

Ein (ungelöstes, RS) Problem beim Schulunterricht ist die Frage, wie das explizite (Grammatik-)Wissen in tatsächliches automatisches Sprachverhalten umgesetzt werden kann.
Aber das befreit die Lehrpersonen nicht von der Pflicht, das explizite Wissen zu haben.
Es geht nicht darum, im Studium Wissen zu vermitteln, das später in der Schule so weitergegeben wird, sondern um die Vermittlung von Wissen, das für einen erfolgreichen Sprachunterricht notwendig ist. (19)
>>
Nur das hilft auch bei der Einschätzung des sprachlichen Hintergrunds von lernenden Kindern. (22)
Erst auf dem Hintergrund umfassenden Wissens kann nach der methodisch-didaktischen Aufbereitung gefragt werden. (22)


Der Linguistik werde zu viel und der Sprache zu wenig Raum eingeräumt. 
\textit{Unser Gegenstand ist die Sprache, nicht die Sprachwissenschaft.} (22)

\textit{Das Verhältnis zur Sprache, das als '`metasprachliche Kompetenz'', durch '`Reflexion über Sprache'' und `'Transfer von explizitem zu implizitem Wissen'' von den meisten Lehrplänen gefordert wird, können die Schüler nicht entwickeln, wenn es die Lehrer nicht haben.} (23)


%%%%%%%%%%%%%%%%%%%%%%%%%%%%%%%%%%%%%%%%%%%%%%%%%%%%%%%%%%%%%%%%%%%%%%%
\paragraph{\citet{Portmanntselikas2011}}

Der späte Spracherwerb dreht sich um literale Kompetenzen (Schrift, Hochsprache, Fachsprachen, literale Register, Unterschied zwischen Bedeutung und pragmatischer Funktion, Textkompetenzen \zB in Form von Kohärenz) (71)

Auseinandersetzung mit Orthographie erstreckt sich über die Schulzeit hinaus. (72)

Orale Korrektheit ist im Bezug auf Literalität zu werten. (72)

Grammatikunterricht steuert diesen Erwerb und entdeckt/korrigiert Probleme und muss daher fundamentale Regularitäten vermitteln, da sonst der systematische Charakter von Sprache nicht zu Geltung kommt. (72)

Problem: Wissen über Sprache kann nicht in automatische und spontan funktionierende Mechanismen der Sprachproduktion umgesetzt werden. (73)

\citet[75--79]{Portmanntselikas2011} beschreibt ein Sprachproduktions- und Sprachverarbeitungsmodell, das primär unbewusste spontane Mechanismen annimmt.
Es gibt jedoch Rückkopplungsschleifen, die die bewusste Kontrolle des produzierten sprachlichen Materials erlauben (76), und die in vielen bildungssprachlichen Kontexten standardmäßig zum Einsatz kommen (die meisten schriftsprachlichen Kontexte [auch 77], Normdruck, explizite Sprachreflexion).

Im Laufe der Erwerbskarriere verschieben sich die Schwerpunkte von orthographischer und grammatischer Korrektheit zu kontextueller und funktionaler und stilistischer Angemessenheit, "`Grammatisches spielt aber fast überall mit eine Rolle"'. (77--78)

Im Laufe der Restrukturierung der Sprachkompetenz durch Schriftspracherwerb setzt der Zweifel an der Angemessenheit der eigenen produzierten Sprache ein. (78--79)
Der Zweck davon ist die Produktion von informativerer, für die Hörenden bzw.\ Lesenden optimalerer Sprache. (79)

Grammatisches Wissen\ldots

Zunächst ist für die Beherrschung von Sprache immer das Erkennen und Unterscheiden von Formen relevant, auch wenn es meist implizit geschieht. (79)
Inwiefern dabei Explizierungen helfen, ist nicht abschließend geklärt. (80)

(Frei paraphrasiert, RS:) Für einen Sprachunterricht, der eine systematische und nicht bloß inselhaft-episodische Wirkung hat, ist muss auf begrifflich fassbare Regularitäten zurückgreifen können, und die relevanten Probleme müssen gezielt in den Aufmerksamkeitsfokus gebracht werden. (80)

Der Sprachunterricht schafft eine Kultur der \textit{habituellen Aufmerksamkeit} (84) und versetzt Lernende dadurch in die Lage, ihre eigene Sprache zu reflektieren, (RS) wo dies nötig wird, um bildungssprachlich angemessene Sprache zu produzieren.
Grammatische Begriffe sind in der Sprachwirklichkeit der Lernenden durchaus entbehrlich, aber das explizite Wissen um die Sprache erschließt erst die Sichtweise auf Sprache, die das kontrollierte Beherrschen von Standard- und Bildungssprache erst möglich macht. (85, frei interpretiert)

Es ist nicht einzusehen, warum wir eine analytische Einsicht in die Chemie oder Physik schulisch Lernenden fordern sollten, aber keine analytische Einsicht in Sprache.
Die analytisch-systematische Sicht auf die Welt prägt die Kultur, die unsere schulischen Inhalte bestimmt, und sie darf vor Sprache nicht halt machen. (85, sehr frei)


%%%%%%%%%%%%%%%%%%%%%%%%%%%%%%%%%%%%%%%%%%%%%%%%%%%%%%%%%%%%%%%%%%%%%%%
\paragraph{\citet{Menzel2017}}

Formaler Grammatikunterricht ohne Bezug zu sprachlichen Anwendungen gegenüber funktionalem Unterricht, der semantische, textuelle oder kommunikative Funktionen zu den Formen stellt. (8)

Orthogonal dazu steht die Achse zwischen systematisch (Teilsysteme konzeptuell vollständig im Blick) und funktionsorientiert (Einzelkategorien in situativen Zusammenhängen. (8)

Deduktiv gegenüber induktiv. (8)

Forderungen an Grammatiken: systematisch, induktiv, funktional, integrativ (nicht ohne Bezug zu Situation, Verwendungskontext, Inhalt). (9)

Wert eines systematischen Grammatikunterrichts nicht einfach nur ein positiver Beitrag zur Sprachfähigkeit, sondern auch zu den allgemeinen Fähigkeiten im Abstrahieren und Kategorisieren. (9) Setzt aber (RS) systematischen und induktiven Ansatz voraus.

Grammatik kommt oft zu früh (Grundschule).
In der Sekundarstufe dann mit dem Anspruch der Wiederholung und dem Anspruch der Verwaltung von deklarativem Wissen.
Wenn echter systematischer Grammatikunterricht gut möglich wäre (in der Sekundarstufe II), wird Grammatik als bekannt vorausgesetzt. (9)

Das Problem des rein situationsorientierten Ansatzes ist, dass nur sporadisches Inselwissen angesichts ganz bestimmter eingegrenzter Aufgaben vermittelt werden kann. (10)

Sprachprobleme von Kindern werden immer korrigierend, selten aber analytisch (mit Bezug auf das gesamte System der Grammatik) aufgegriffen. (10)

Lernende haben meist kein Interesse an der Form, sondern daran, zu verstehen, warum manche Formen in bestimmten Kontexten akzeptiert werden und andere nicht. (11)

Grammatiken sind menschengemacht. (13)



%%%%%%%%%%%%%%%%%%%%%%%%%%%%%%%%%%%%%%%%%%%%%%%%%%%%%%%%%%%%%%%%%%%%%%%
\paragraph{\citet{Bredel2013}}

Wenn es nur um das Können von Sprache geht, ist die Reflexion nicht unbedingt erforderlich. (14)

Fachdidaktik ist keine praktische Anleitung zur Unterrichtsvorbereitung, sondern die wissenschaftliche Ausbildung für das richtige Handeln im Unterricht. (15)


Kinder haben vor dem Schriftspracherwerb keine Konzepte von Wörtern oder Sätzen und identifizieren Laute, Silben oder Sätze als Wörter (38, aus \citealt{SteinigHuneke2002}).
Sätze werden ebenfalls nicht identifiziert, vielmehr aber Äußerungen. (39)


Der Verlust der Sprechen-Hören-Situation führt zu einer stärkeren Fokussierung der formalen Eigenschaften. (41)

Schreiben erlaubt bessere und bewusstere Planungs- und Revisionsprozesse (Deautomatisierung). (42--43)

Lernender brauchen eine Anleitung für Textplanungsprozesse. (44--45)


Das Ziel der expliziten Sprachbetrachtung in der Schule ist vor allem ein späteres primärsprachliches Handeln. (77)
Das Sprechen über eine Sprache erfordert aber Wissen über diese Sprache, nicht nur Können der Sprache. (94)
Wie die Umsetzung von explizitem Wissen in Anwendungswissen durch Schulunterricht genau geschehen soll, und ob die überhaupt möglich ist, ist eine offene Frage. (Abschnitt 1.5) Aber (RS) niemand wird riskieren wollen, den systematischen erstsprachlichen Unterricht abzuschaffen oder durch rein situativ vermitteltes Inselwissen zu ersetzen. Vernachlässigt wird dabei auch (RS) die Frage, auf welchen Niveaustufen der explizite Unterricht am besten wirkt. Vermutlich täte er dies vor allem später, als die Vermittlung grammatischen bzw.\ linguistischen Wissens in der Schule endet (Sekundarstufe II). Als Linguist profitiere ich selber \zB beim Fremdsprachenlernen erheblich von expliziter systematischer Vermittlung linguistischen Wissens.

=== Didaktische Ansätze der Sprachbetrachtung ===

Die Methode \textit{vom Laut zur Schrift} scheitert am phonematischen Schreibprinzip.
Außerdem setzt die induktive Ableitung der Schreibprinzipien aus den phonologischen Prinzipien eine standardkonforme Aussprache voraus. (171)
(RS) Die Didaktik hat eine Reihe von funktionierenden Methoden an der Hand, die bei Lehrenden allerdings ein tieferes Verständnis des Systems voraussetzen. (vgl. \citealt[Abschnit~2.1]{Bredel2013}).

Auf jeden Fall wissen wir, dass falsche Vermittlungsmethoden wie semantisierende Definitionen von Wortarten zu falschem Kategorienwissen führen (178--179). (RS) Lehrende müssen daher in der Lage sein, auch Lehrmethoden, die in Schulbüchern eingesetzt werden, kritisch zu bewerten. Dazu ist sowohl Wissen um die Sache als auch didaktisches Wissen unabdinglich.




%%%%%%%%%%%%%%%%%%%%%%%%%%%%%%%%%%%%%%%%%%%%%%%%%%%%%%%%%%%%%%%%%%%%%%%
\paragraph{Meine Notizen}

Meine Beobachtung: Sprachlich kompetente angehende Lehrpersonen können oft sehr gut Bedeutungen und Funktionen von bestimmten Formen benennen, aber sie haben einen schlechten Überblick über das Formenangebot und die Formalternativen insgesamt.


















% \section{Ziele dieses Kapitels}
% 
% Dieses Kapitel ist insofern ungewöhnlich, als es in der dritten Auflage des Buches komplett neu hinzugekommen ist, und als es in ihm zumindest vordergründig nicht um Sprache (und Personen, die Sprache benutzen) geht, sondern um Menschen, die mit Sprache reflektierend umgehen, nämlich Lehr- und Lernpersonen in Schulen und Universitäten.
% Nachdem die wesentlichen Grundlagen der wissenschaftliche Grammatikforschung in den vorangegangenen Kapiteln dargelegt wurden, folgt also scheinbar eine Digression.
% Während das Kapitel ohne jegliche Folgen für das weitere Verständnis übersprungen werden kann, ist es gerade für Lehrpersonen im Fach Deutsch an Schulen wichtig, über den Stellenwert einer gründlichen grammatischen Beschreibung im Deutschunterricht nachzudenken.
% Der mangelnde \textit{Anwendungsbezug} wird schließlich immer wieder in Evaluationen von Lehrveranstaltungen zur Linguistik insbesondere von angehenden Lehrpersonen bemängelt.
% Wir diskutieren hier, worin der Anwendungsbezug der Grammatik an sich für die schulische Lehre besteht, und warum es nicht zielführend ist, bei jedem linguistischen Detail, das im Studium besprochen wird, die Rundumschlagsfrage zu stellen: "`Wofür brauche ich das denn später im Lehramt?"'
% 
% Es geht damit mittelbar auch darum, den Zweck des gesamten Buches und den Nutzen einer linguistischen bzw.\ grammatischen Ausbildung für einen Großteil des Publikums dieses Buches (nämlich angehende Lehrpersonen) zu motivieren.
% Ich argumentiere dafür, dass das, was hier auf vielen hundert Seiten dargelegt wird, einen sehr genau definierten und wichtigen Platz in der Ausbildung von Lehrpersonen für das Fach Deutsch hat.
% Dies setzt natürlich auch voraus, dass die Ziele des Schulfachs Deutsch, zumindest was die Grammatik angeht, angegeben werden.
% 
% Das Kapitel ist wie folgt aufgebaut.
% Zunächst wird in Abschnitt~\ref{sec:formundfunktionindergrammatik} genauer über die Beziehung von Form und Funktion zu sprechen, weil diese ein Hauptthema in der Diskussion um Grammatik in Schule und Lehramtsstudium ist.
% Ich gehe in diesem Abschnitt dabei sowohl von einer bestimmten Sorte der Kritik an diesem Buch als auch von spezifischen Problemen, die in der Lehramtsausbildung zu beobachten sind, aus.
% In Abschnitt~\ref{sec:kenntnisseerwartungenundmotivationvonstudierenden} werden dann empirische Daten präsentiert, die illustrieren, welchen Wissensstand Studierende in Schulgrammatik haben, und welche Funktionen des schulischen Grammatikunterrichts für Studierende wichtig sind.
% In Abschnitt~\ref{sec:grammatikinderschule} werden kompakt Konzepte des schulischen Grammatikunterrichts diskutiert, die zwischen Fachdidaktik und Fachwissenschaft diskutiert werden.
% In Abschnitt~\ref{sec:grammatikimlehramtsstudium} wird schließlich ein Konzept der Grammatikausbildung im Lehramtsstudium skizziert.
% Stilistisch bewegen wir uns hier übrigens etwas ungewöhnlich zwischen einer Diskussion, die die germanistische Linguistik und die germanistische Fachdidaktik schon länger um die Sache führen, und einer Vermittlung von Inhalten und Anregungen für Studierende des Fachs.

